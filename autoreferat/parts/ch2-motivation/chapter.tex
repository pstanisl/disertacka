% !TEX root = ../../autoreferat.tex
\ifdefined\CELE
\else
\documentclass[a4paper,12pt]{scrreprt}

\usepackage[utf8]{inputenc}
\usepackage[czech]{babel}
\usepackage[T1]{fontenc}
\usepackage{amsmath}
\usepackage{amsfonts}
\usepackage{amssymb}
\usepackage{graphicx}
\usepackage{longtable}
\author{Jan Bartošek}
\usepackage{cite}
\usepackage{url}
\usepackage{subfig}
\usepackage{multibib}
\usepackage[hidelinks,unicode]{hyperref}

%\usepackage{epsfig}
%\usepackage{epstopdf}

\linespread{1.1}
\usepackage{a4wide}

%\usepackage[raggedright]{titlesec} % zamezeni deleni nadpisu

%\newcites{all}{Seznam literatury}
\newcites{my}{Soupis publikací}

\newcommand*{\tabh}[1]{\multicolumn{1}{|c|}{#1}}

\captionsetup{justification=centering}

\begin{document}

\fi

\chapter{Motivace a cíle disertační práce}
\label{chap:mot}

\begin{enumerate}
  \item Seznamte se s~přístupy, které umožňují alespoň částečnou obnovu schopnosti řečové komunikace u pacientů po totální laryngektomii (TL).
  \item Pro účely konstrukce systému automatického rozpoznávání řeči u lidí po totální laryngektomii využívajících pro komunikaci elektrolarynx navrhněte a pořiďte vhodný korpus řečových nahrávek.
  \item Natrénujte základní systém rozpoznávání řeči pro jednoho řečníka - pacienta po totální laryngektomii mluvícího pomocí elektrolarynxu - a porovnejte funkcionalitu systému (zejména jeho přesnost) se systémem rozpoznávajícím řeč zdravých lidí. Ke konstrukci systému využijte state-of-the-art metody.
  \item Analyzujte základní příčiny případné zvýšené chybovosti realizovaného systému rozpoznávání řeči a pokuste se navrhnout vhodné úpravy v~jeho konstrukci, které chybovost sníží. Diskutujte vhodnost navrženého řešení.
\end{enumerate}

\ifdefined\CELE
\else
%\bibliographystyle{plain}
%\bibliography{literatura}

%\bibliographystyleall{plain}
%\bibliographyall{literatura.bib}
%\bibliographymy{literatura}
\end{document}

\fi
