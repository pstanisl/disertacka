% !TEX root = ../thesis.tex
\section{Analýza akustického signálu a jeho parametrizace}
\label{chap:construction:analysis}

% Rozpoznávání řeči se věnuje nemalé úsilí již od 50. let 20. století a v~současné době nikoho nepřekvapí téměř bezchybně fungující obecný rozpoznávač souvislé řeči v~mobilních zařízeních.
% Pro obecné systémy dokonce existují korpusy s~desítkami, stovkami i více hodinami promluv, které je možné využít při vytváření těchto systémů.
% Tyto korpusy ve většině případů obsahují pouze \uv{standardní}\footnote{Slovním spojením \uv{standardní řeč} je myšlena řeč neobsahující výrazné řečové vady, případně jiné formy produkce a často v~nepříliš akusticky náročném prostředí.} řeč.
% Pokud se objeví snaha využít systém za specifických podmínek vytvořit nebo ověřit jeho funkčnost (ať už se jedná o rušné prostředí či speciální typy promluv), tak je nezbytné získat potřebná data pořízená za srovnatelných podmínek.

\subsection{Analýza získaných dat}
\label{chap:construction:analysis:data}

% Získaný korpus obsahuje přes 10 hodin akustických záznamů promluv a více či méně přesných přepisů\footnote{I přes nemalou snahu a několikastupňovou kontrolu, je téměř jisté, že by nebylo obtížné najít přepis, který obsahuje chybu například ve formě překlepu.}.
% V momentě, kdy jsou  k~dispozici data, je možné se zaměřit na specifika EL řeči a případně porovnat se zdravým řečníkem.

% Pro potřeby porovnání byl použit začátek promluvy \textit{\uv{Akcie Komerční banky...}}.
% Tato promluva je součástí standardní množiny vět používaných při vytváření řečových korpusů na katedře kybernetiky při ZČU.
% Tím pádem je  k~dispozici v~relativně velkém množství příkladů pro zdravé řečníky. Tato věta je součástí také korpusu EL řeči.

Na obr. \ref{fig:construction:analysis:comparison} je zobrazen průběh amplitudy a spektrogram vybrané promluvy pro zdravého řečníka (obr. \ref{fig:construction:analysis:comparison:normal}) a EL řečníka (obr. \ref{fig:construction:analysis:comparison:el}).
Už na první pohled je možné zaznamenat určité rozdíly i přesto, že obsah obou promluv je identický.
Prvním rozdílem je délka promluvy.
% V~případě zdravého řečníka je v~průměru\footnote{Hodnota odvozena na základě 10 náhodně vybraných promluv ze standardně používaného korpusu na Katedře kybernetiky ZČU.} o celou $1$ vteřinu kratší než v~případě EL řeči.
% Tempo řeči je samozřejmě velmi individuální, ale z principu je EL řeč pomalejší.
% Navíc
Z~průběhu signálu na obr. \ref{fig:construction:analysis:comparison:el} je patrné, že EL řečník dělá výraznější pauzy mezi jednotlivými slovy promluvy.
% To je často způsobené potřebou naplnit jícen vzduchem.
% Po TL je dýchání realizováno přes tracheu a pokud nebyl voperován shunt (více v~\ref{chap:cause:treatment:tracheo}), tak je trvale oddělen hrtan a hltan.
% I přesto je potřeba, pro produkci některých neznělých fonémů exhalovat vzduch z dutiny ústní.
% Zkušený EL řečník to dělá naprosto automaticky, nicméně \uv{polykání} vzduchu zabere nějaký čas.
% Nevyhnutelným důsledkem je pak velmi častý výskyt samovolného říhání v~průběhu promluvy\footnote{Fakt, že je říhání jako neřečová událost běžnou součástí téměř každé promluvy, vedl  k~ignorování těchto událostí během anotace.}.

\begin{figure}[htpb]
  \centering
  \begin{subfigure}[b]{0.45\textwidth}
    \includegraphics[width=\textwidth]{./parts/ch5-construction/img/energy_spec_normal.png}
    \caption{Zdravý řečník}
    \label{fig:construction:analysis:comparison:normal}
  \end{subfigure}
  %
  \begin{subfigure}[b]{0.45\textwidth}
    \includegraphics[width=\textwidth]{./parts/ch5-construction/img/energy_spec_el.png}
    \caption{EL řečník}
    \label{fig:construction:analysis:comparison:el}
  \end{subfigure}
  \caption[Průběh a spektrogram promluvy zdravého a EL řečníka.]{Průběh a spektrogram promluvy a vyznačenou energií promluvy zdravého a EL řečníka.}
  \label{fig:construction:analysis:comparison}
\end{figure}

% Svou roli může hrát i snaha správně artikulovat.
% Při používání EL je nezbytné, aby bylo produkované řeči alespoň trochu rozumět.
% Správná artikulace si žádá svůj čas a není snadné mluvit rychle.
% Při nahrávání bylo velmi běžné, že v~průběhu promluvy řečník udělal pauzu, aby mohl lépe umístit EL, protože jeho umístění má velký vliv na kvalitu produkované řeči.
% Nicméně je třeba říci, že tempo není pro ASR systémy problém.

% Dalším způsobem, jak ukázat rozdíly mezi promluvou zdravého řečníka a řečníka s~EL, je porovnat oba signály ve frekvenční oblasti.
Zajímavé je také porovnání signálů od obou řečníků ve frekvenční oblasti.
Obsah obou promluv je identický.
% Pro větší názornost jsou na obr.~\ref{fig:construction:spectrogram} zobrazena společně spektra ukázkové promluvy zdravého řečníka (\ref{fig:construction:spectrogram:normal}) a toho s~EL (\ref{fig:construction:spectrogram:el}), přičemž obsah obou promluv je identický.
% Prvním markantním rozdílem je mnohem
U EL řečníka je významnější zastoupení šumu v~promluvě
% řečníka s~EL
v~úsecích \uv{ticha}, viz obr. \ref{fig:construction:spectrogram:el}.
% To je nepochybně způsobeno samotným EL, který řečník mezi jednotlivými slovy nevypíná.
% Přítomnost šumu se projeví zejména na průběhu energie (viz obr. obr. \ref{fig:construction:el_speech}), zejména před prvním a druhým slovem promluvy.
% Zajímavá je přítomnost šumu v~celém frekvenčním spektru, přestože EL produkuje konstantní buzení, které je ve spektru (obr. \ref{fig:construction:spectrogram:el}) reprezentováno výraznou souvislou linií na nízkých frekvencích.
% Přítomnost šumu na vyšších frekvencích je způsobena umístěním mikrofonu, který je nalepen přímo na pokožku, a tím pádem snímá namodulované vibrace přenášené měkkou tkání.
% Tato hypotéza byla potvrzena v~dalších etapách nahrávání, kde byl použit studiový mikrofon vzdálený od úst minimálně 15 cm, toto bude podrobněji bude popsáno v~části \ref{chap:realisation:corpus}.
% Nicméně z pohledu použitelnosti nějakého budoucího systému je nezbytné počítat i se situací, kdy mikrofon zaznamenává i vibrace přenášené tkání.
Dalším rozdílem v~promluvě EL řečníka je absence vyšších frekvencí u většiny produkovaných fonémů.
% Absence vyšších frekvencí se dá vysvětlit nejen použitím EL, kde samotný EL má vždy konstantní frekvenci buzení, ale i tím, že nedochází  k~modulaci ve všech dutinách vokálního traktu.
% Výjimku tvoří afrikáty $/c/$ a $/\check{c}/$, u kterých jsou hlasivky (u zdravého jedince) v~klidu, a vznikají uvolněním nahromaděného vzduchu v~dutině ústní\footnote{Nahromadění vzduchu je realizováno přitisknutím jazyka  k~přední/zadní části horního patra.} \cite{Psutka2006}.
% U řečníka po TL není mechanismus produkce těchto fonémů žádným způsobem ovlivněn.
% Problémem teoreticky může být zdroj vzduchu, jelikož jej z plic není možné dostat do dutiny ústní, ale jak už bylo zmíněno (a spektrogram to potvrzuje), zkušený uživatel EL se dokáže adaptovat.

% Nicméně nejdůležitější frekvenční složky, zajišťující srozumitelnost promluvy, se vyskytují ve frekvenčním pásmu od $1\ kHz$ do $3\ kHz$.
% Vyšší frekvence se podílejí a priori na zabarvení hlasu.

\begin{figure}[htpb]
  \centering
  \begin{subfigure}[b]{0.4\textwidth}
    \includegraphics[width=\textwidth]{./parts/ch5-construction/img/spectrogram_normal.png}
    \caption{Zdravý řečník}
    \label{fig:construction:spectrogram:normal}
  \end{subfigure}
  %
  \begin{subfigure}[b]{0.4\textwidth}
    \includegraphics[width=\textwidth]{./parts/ch5-construction/img/spectrogram_el.png}
    \caption{EL řečník}
    \label{fig:construction:spectrogram:el}
  \end{subfigure}
  \caption{Spektrogram promluvy \uv{Akcie Komerční banky} dvou řečníků.}
  \label{fig:construction:spectrogram}
\end{figure}

Dalším způsobem, jak porovnat promluvy zdravého řečníka a EL řečníka, je na úrovni analýzy jednotlivých fonémů.
Na obr. \ref{fig:construction:phonemes:k} - \ref{fig:construction:phonemes:c} jsou zobrazeny průběhy amplitud jednotlivých fonémů v~čase\footnote{Hodnoty času odpovídají časům výskytu v~původní promluvě.} pro fonémy $/k/$, $/g/$ a $/\check{c}/$.
% V případě $/k/$ a $/g/$ (obr. \ref{fig:construction:phonemes:k} a \ref{fig:construction:phonemes:g}) se jedná o okluzivy, konkrétně $/k/$ je kategorizováno jako neznělá ploziva a $/g/$ jako znělá ploziva.
% Tyto fonémy obecně vznikají v~důsledku uzavření vydechovaného proudu vzduchu pomocí artikulačních orgánů, což se projeví jako krátká pauza (tzv. okluze).
% Po té následuje náhlé jednorázové uvolnění překážky a únik nahromaděného vzduchu, tzv. exploze \cite{Psutka2006}.
% Takto popsáno to samozřejmě funguje u zdravého jedince.
% V případě EL řečníka je pro jejich produkci využíván stejný mechanizmus, ale vydechovaný vzduch pochází z hltanu.
% Dalším rozdílem je samozřejmě absence hlasivek.

Foném $/k/$ je tedy zástupcem skupiny neznělých fonémů.
% Ty se vyznačují tím, že do jejich produkce nevstupují hlasivky, jsou v~klidu.
% Zdrojem buzení je tedy šum, viz část \ref{chap:asr}.
Pokud se podíváme na průběh amplitudy v~čase u zdravého řečníka (obr. \ref{fig:construction:phonemes:k:normal}), není zde vidět žádný periodický signál.
% Hlasivky jsou tedy opravdu v~klidu.
Oproti tomu u EL řečníka (obr. \ref{fig:construction:phonemes:k:el}) je jasně patrné, že je zde přítomno aktivní buzení vytvořené EL.
% Ve frekvenční oblasti je zobrazeno tzv. amplitudové spektrum, které znázorňuje závislost amplitudy signálu na frekvenci.
% V případě zdravého řečníka odpovídá vývoj předpokladům.
% Není zde žádná výrazná frekvence a také nedochází k~výraznému útlumu.
% Přestože se v~obou případech jedná o stejný foném, je z časového i frekvenčního průběhu amplitudy zřejmé, že parametry signálu se u obou řečníků diametrálně liší.

\begin{figure}[htpb]
  \centering
  \begin{subfigure}[b]{0.45\textwidth}
    \includegraphics[width=\textwidth]{./parts/ch5-construction/img/signal-normal_k.png}
    \caption{Zdravý řečník}
    \label{fig:construction:phonemes:k:normal}
  \end{subfigure}
  %
  \begin{subfigure}[b]{0.45\textwidth}
    \includegraphics[width=\textwidth]{./parts/ch5-construction/img/signal-el_k.png}
    \caption{EL řečník}
    \label{fig:construction:phonemes:k:el}
  \end{subfigure}
  \caption[Průběh amplitudy fonému $/k/$ zdravého a EL řečníka.]{Průběh amplitudy $/k/$ v~časové a frekvenční oblasti fonému u~zdravého a EL řečníka.}
  \label{fig:construction:phonemes:k}
\end{figure}

Jako druhý ukázkový foném byl vybrán $/g/$.
Jedná se o znělou plozivu.
% Při produkci znělých fonémů hrají velký vliv hlasivky, protože jsou zdrojem buzení.
Z~obr. \ref{fig:construction:phonemes:g:normal} je patrné buzení zřetelné ve formě periodického průběhu amplitudy.
U EL řečníka (obr. \ref{fig:construction:phonemes:g:el}) je také vidět periodický signál, ale úplně jiného charakteru.
% Svým způsobem dost podobný tomu, který je zřetelný u fonému $/k/$.
Rozdíl je zřetelný i ve frekvenční oblasti, kdy u EL řečníka nedochází  k~útlumu ve střední oblasti frekvenčního spektra.

\begin{figure}[htpb]
  \centering
  \begin{subfigure}[b]{0.45\textwidth}
    \includegraphics[width=\textwidth]{./parts/ch5-construction/img/signal-normal_g.png}
    \caption{Zdravý řečník}
    \label{fig:construction:phonemes:g:normal}
  \end{subfigure}
  %
  \begin{subfigure}[b]{0.45\textwidth}
    \includegraphics[width=\textwidth]{./parts/ch5-construction/img/signal-el_g.png}
    \caption{EL řečník}
    \label{fig:construction:phonemes:g:el}
  \end{subfigure}
  \caption[Průběh amplitudy fonému $/g/$ zdravého a EL řečníka.]{Průběh amplitudy fonému $/g/$ v~časové a frekvenční oblasti fonému u~zdravého a EL řečníka.}
  \label{fig:construction:phonemes:g}
\end{figure}

Posledním ukázkovým fonémem je již zmiňované $/\check{c}/$.
Jedná se o neznělý foném , který vzniká přiložením jazyku k~zadní části horního patra.
% Tím je zadržen vzduch v~dutině ústní a vzniká krátká pauza.
% Uvolněním pak dochází  k~explozi a vytvoření zvuku \cite{Psutka2006}.
% Do produkce se nezapojují hlasivky a produkovaný zvuk by měl být dostatečně intenzivní, aby nebyl v~případě EL řečníka tolik neovlivněn případným EL.
Tím pádem by měl být průběh signálu u obou řečníků podobný a to jak v~časové, tak i ve frekvenční oblasti, viz obr. \ref{fig:construction:phonemes:c}.

\begin{figure}[htpb]
  \centering
  \begin{subfigure}[b]{0.45\textwidth}
    \includegraphics[width=\textwidth]{./parts/ch5-construction/img/signal-normal_c.png}
    \caption{Zdravý řečník}
    \label{fig:construction:phonemes:c:normal}
  \end{subfigure}
  %
  \begin{subfigure}[b]{0.45\textwidth}
    \includegraphics[width=\textwidth]{./parts/ch5-construction/img/signal-el_c.png}
    \caption{EL řečník}
    \label{fig:construction:phonemes:c:el}
  \end{subfigure}
  \caption[Průběh amplitudy fonému $/\check{c}/$ zdravého a EL řečníka.]{Průběh amplitudy fonému $/\check{c}/$ v~časové a frekvenční oblasti fonému u~zdravého a EL řečníka.}
  \label{fig:construction:phonemes:c}
\end{figure}

Z doposud provedené analýzy plyne, že EL řeč je v~mnoha charakteristikách odlišná od té produkované zdravým řečníkem.
Zejména ve frekvenční oblasti (obr. \ref{fig:construction:phonemes:k} a \ref{fig:construction:phonemes:g}) jsou výše uvedené rozdíly patrné.
% Tento fakt nepochybně přispívá  k~tomu, že standardní obecné modely rozpoznávání řeči nedosahují takové přesnosti jako v~případě běžné promluvy, viz dále.

% \csvautotabular{./ch5-construction/test.csv}
