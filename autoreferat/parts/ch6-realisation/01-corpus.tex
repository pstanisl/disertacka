% !TEX root = ../../autoreferat.tex
\section{Doplnění řečového korpusu o specifická data - vliv nových dat na kvalitu akustického modelu}
\label{chap:realisation:corpus}

Před samotným pořízením nahrávek promluv bylo nezbytné vybrat co možná nejvíce dvojic slov, které se liší významem a znělostí právě jednoho fonému.
Příkladem takovýchto slov může být dvojice slov \textit{kosa} + \textit{koza} nebo \textit{přibít} + \textit{připít}.
% Pro tento účel byl vyvinut algoritmus výběru slov, který provede výběr dílčích kroků pomocí

% \begin{enumerate}
%   \item načtení dat (slovník a hledané párové fonémy);
%   \item shluknutí všech slov vedoucích ke stejné fonetické transkripci;
%   \item vytvoření všech možných kombinací dvojic slovních transkripcí;
%   \item nalezení dvojic transkripcí, které se liší právě ve znělosti jednoho fonému\footnote{Algoritmus vzájemně porovná obě slova a najde rozdílné fonémy. Pokud tyto rozdíly odpovídají některé z dvojic párových fonémů, tak je dvojice přijata.};
%   \item výběr dvojic slov na základě vybraných fonetických transkripcí.
% \end{enumerate}

% \noindent Jako slovník byl použit seznam slov s~fonetickými přepisy, které pocházejí z jazykového modelu obsahujícího 1,2 milionu slov.
% Pomocí výše zmíněného algoritmu se podařilo nalézt $160$ párů slov lišících se znělostí právě jednoho fonému, celkem tedy $320$ slov.
% Ke každému nalezenému slovu se následně vybrala minimálně jedna věta obsahující toto slovo (ale nikoli druhé slovo z dvojice).
% Těchto vět je pak $418$. Příklad vybraných vět je uveden níže:

% \begin{verbatim}
%   Zkoušel jsem to několikrát, ale pokaždé padla kosa na kámen.
%   Do basy nemusí, vlk žere, koza žije.
% \end{verbatim}

Vybraná slova a věty se staly základem pro 2. etapu nahrávání.
Oproti 1. etapě nahrávání byla eliminována role anotátorů, v~důsledku mnohem širších možností nahrávacího softwaru, protože umožňoval větší možnosti automatické kontroly promluv v průběhu nahrávání.
Celkem se podařilo získat přibližně 2 hodiny řeči (každá nahrávka obsahuje $0,5\ s$ ticha na začátku a konci).
% Ta se uskutečnila během dvou sezení v~červenci roku 2016 se stejným řečníkem jako v~1. etapě.
% Jde tedy o relativně velký časový odstup od 1. etapy.
% Jednotlivá nahrávací sezení měla mezi sebou týdenní rozestup.
% Oproti 1. etapě probíhalo nahrávání v~odhlučněné nahrávací komoře za pomocí profesionálního nahrávacího zařízení.
% Studiový mikrofon byl od úst řečníka vzdálen přibližně 15 cm.
% K nahrávání byl použit speciální software, který kontroloval, zda každá nahrávka splňuje určité parametry.
% Každá akceptovaná nahrávka musela mít na svém začátku a konci minimálně $0,5\ s$ ticha a zároveň celá nahrávka nesměla být příliš tichá a zároveň přebuzená (kontrolováno pomocí energie).
% Pokud nahrávka nesplňovala definované parametry, byla zamítnuta a řečník musel promluvu zopakovat.

% Oproti 1. etapě nahrávání byla eliminována role anotátorů, v~důsledku mnohem širších možností nahrávacího softwaru.
% Nahrávací software řečníkovi vždy ukáže text, který je potřeba vyslovit, a následně jej společně s~audio záznamem uloží.
% K dispozici je tedy nahrávka a její \uv{přepis}.
% Nicméně samotný řečník často může udělat chybu aniž by si toho všiml (např. záměnou podobných slov apod.).
% Software ale žádným způsobem nekontroluje co je ve skutečnosti vysloveno, proto je nahrávání přítomen operátor, který poslouchá co bylo řečeno a v~případě potřeby zamítne nahrávku.
% Řečník následně musí promluvu opakovat, dokud nahrávka neodpovídá požadovaným parametrům a zároveň není obsahově správně.

% Na obr. \ref{fig:realisation:corpus:word} a \ref{fig:realisation:corpus:sentence} jsou ukázky audio záznamu slova \uv{kosa} a věty \uv{Zkoušel jsem to několikrát, ale pokaždé padla kosa na kámen.}.
% Pokud se nahrávky porovnají s~nahrávkami získanými v~1. etapě (obr. \ref{fig:construction:el_speech}), je patrná vyšší kvalita nahrávek, zejména vyšší amplituda.
% Ze zobrazených spektrogramů je zřejmé, že šum je přítomen v~menším množství a intenzitě než v~nahrávkách získaných v~průběhu 1. etapy.
% Hlavní vliv na toto má studiový mikrofon, který není přilepen ke tváři řečníka, a tudíž nezaznamenává vibrace přenášené měkkou tkání.
% Další rozdíl je vidět v~oblasti nižších frekvencích spektrogramu, ty jsou výraznější.
% Přestože se jedná o stejného řečníka, zaznamenaná řeč nemá úplně identické parametry.
% Jedním z důvodů bude nepochybně změna nahrávací aparatury a procesu nahrávání.
% Nezanedbatelný vliv má i relativní nestálost parametrů EL řeči.
% Ty jsou závislé na typu a pozici elektrolarynxu.
% Ten se v~době mezi nahráváními navíc změnil, což v~konečném důsledku představuje asi hlavní důvod diference parametrů.

% \begin{figure}[hbpt]
%   \centering
%   \includegraphics[width=0.9\textwidth]{./ch5-construction/img/energy_spec_word.png}
%   \caption[Průběh a spektrogram slova \uv{kosa}.]{Průběh a spektrogram slova \uv{kosa} s~společně s~vyznačenou energií EL promluvy.}
%   \label{fig:realisation:corpus:word}
% \end{figure}

% \begin{figure}[hbpt]
%   \centering
%   \includegraphics[width=0.9\textwidth]{./ch5-construction/img/energy_spec_sentence.png}
%   \caption[Průběh a spektrogram promluvy obsahující slovo \uv{kosa}.]{Průběh a spektrogram promluvy obsahující slovo \uv{kosa} s~vyznačenou energií EL promluvy.}
%   \label{fig:realisation:corpus:sentence}
% \end{figure}

% Tab. \ref{tab:realisation:corpus:recording} přibližuje souhrnné parametry nahrávek pořízených v~2. etapě nahrávání.
% Celkem se podařilo získat přibližně 2 hodiny řeči (každá nahrávka obsahuje $0,5\ s$ ticha na začátku a konci).
% Z toho přibližně jen $10~\%$ představují vybraná izolovaná slova.
% Dohromady s~novými daty obsahuje korpus téměř $14$ hodin audio záznamů a jim odpovídajících přepisů.

% \begin{table}[htpb]
%   \centering
%   \def\arraystretch{1.5}
%   \pgfplotstabletypeset[
%     col sep=comma,
%     string type,
%     columns/phase/.style={column name=Nahrávání, column type={l}},
%     columns/length/.style={column name=Délka \textit{[HH:MM:SS]}, column type={r}},
%     columns/words/.style={column name=Počet slov, column type={r}},
%     columns/sentences/.style={column name=Počet vět, column type={r}},
%     columns/files/.style={column name=Počet souborů, column type={r}},
%     every head row/.style={
%       after row={
%         \cmidrule(r){1-1}
%         \cmidrule(lr){2-2}
%         \cmidrule(lr){3-3}
%         \cmidrule(lr){4-4}
%         \cmidrule(l){5-5}
%       },
%       before row={\toprule}},
%     every last row/.style={
%       after row={\bottomrule}
%     },
%   ]{./ch6-realisation/tabs/0201-recording-stats.csv}
%   \caption{Informace o korpusu nahrávek z 2. etapy nahrávání.}
%   \label{tab:realisation:corpus:recording}
% \end{table}

\subsection{Vliv nových dat na kvalitu modelů}
\label{chap:realisation:corpus:influence}

Rozšíření korpusu umožňuje vytvoření nových modelů, pomocí kterých lze ověřit konzistenci a vliv nových dat na přesnost rozpoznávání.
Na kompletní testovací sadě\footnote{Testovací sada z 1. etapy rozšířena o izolovaná slova z 2. etapy.} bylo dosaženo přesnosti rozpoznávání $Acc_{p}^{GMM} = 54,96~\%$.
V případě, že testovací sada obsahuje pouze nově nahraná slova, tak dokonce jen $Acc_{p}^{GMM} = 42,97~\%$.
To je významné zhoršení oproti výsledkům dosažených u baseline modelu.
Po ověření správnosti trénovacího procesu se ukázalo, že na vině jsou data.
% Oproti baseline modelu jsou všechny následující modely vytvořeny ve frameworku Kaldi.
% Ten se po roce 2015 stal standardem pro vytváření akustických modelů, protože je velmi flexibilní a umožňuje snadné přidávání nových typů akustických modelů \cite{Kaldi2011}.

% Již při vytváření baseline modelu se ukázala lepší funkce DNN modelů.
% Přestože vývoj výpočetních GPU postupuje závratnou rychlostí, tak natrénování HMM-DNN modelu je čásově náročnější než vytvoření HMM-GMM akustického modelu.
% Navíc, jak bylo popsáno v~\ref{chap:asr:acoustic:DNN},  k~natrénování DNN modelu je potřeba zarovnání získané pomocí HMM-GMM modelu.
% Proto je vhodné prvotní validaci nových dat provést na jednodušším modelu.

% Proces vytvoření akustického modelu vychází z předpřipravených Kaldi trénovacích skriptů pro vytvoření modelu pomocí Wall Street Journal korpusu.
% Tyto skripty jsou jen drobně upraveny tak, aby výsledný model mohl být natrénován z EL korpusu. Data jsou parametrizována pomocí PLP s~12 kepstrálními, delta a delta-delta koeficienty\footnote{V rámci úprav Kaldi skriptů se PLP parametrizace ukázala jako vhodnější pro EL řeč. Ověření proběhlo experimentálně. Byly vytvořeny dva identické HMM-GMM modely s~4096 stavy, ale každý byl natrénován na jinak parametrizovaných datech (MFCC a PLP). PLP model dosáhl o~$1,31~\%$ absolutně vyšší přesnost rozpoznávání.}. Nejprve je vytvořen monofónový model, který slouží jako iniciační model pro trifónové  modely, viz obr. \ref{fig:construction:results:baseline:hmm:training}.

% V části \ref{chap:construction:results} bylo popsáno rozdělení korpusu na trénovací a testovací sadu.
% Po rozšíření korpusu je rozdělení dat z 1. etapy ponecháno a nová data  k~jednotlivým sadám přidána.
% Všechny věty nahrané v~2. etapě jsou přidány do trénovací sady a všechna slova naopak do testovací sady.
% Toto rozdělení vychází z impulzu pro rozšíření korpusu o specifická slova.
% Ta tedy a priori mají sloužit  k~otestování nových modelů, a tedy i lepšímu porozumění problematice znělosti u EL řeči.

% Jazykový model je opět fonémový zerogramový.
% Na kompletní testovací sadě bylo dosaženo přesnosti rozpoznávání $Acc_{p}^{GMM} = 54,96~\%$\footnote{Celková délka nahrávek v~testovací sadě složené z vět a slov činila $1h16m42s$.}.
% V případě, že testovací sada obsahuje pouze nově nahraná slova, tak dokonce jen $Acc_{p}^{GMM} = 42,97~\%$\footnote{Celková délka nahrávek v~testovací sadě složené pouze slov činila $15m44s$.}.
% To je významné zhoršení oproti výsledkům dosažených u baseline modelu ($Acc_{p}^{GMM} = 76,64~\%$).
% Pro výpočet přesnosti rozpoznávání je opětovně využit vztah (\ref{eq:asr:decoding:acc}).

% Jelikož došlo ke změně ASR frameworku je potřeba ověřit,že nevznikla chyba při vytváření akustického modelu.
% K ověření je použit křížový test, kdy jsou pomocí stejného procesu natrénovány modely tvořené původními (1. etapa) a novými (2. etapa) daty a křížově otestovány na kompletní, původní a jen nové části testovací sady.
% Trénovaný akustický model má stejné parametry jako v~předchozím případě.
% Vstupem jsou PLP data s~12 kepstrálními, delta a delta-delta koeficienty, výsledný model může mít až 4096 stavů.
% Výsledky testu jsou uvedeny v~tab. \ref{tab:realisation:verification:cross}.
% Z té je jasně patrné, že trénovací proces proběhl správně a vině jsou tedy trénovací data.

% \begin{table}[htpb]
%   \centering
%   \def\arraystretch{1.5}
%   \pgfplotstabletypeset[
%     col sep=semicolon,
%     string type,
%     columns/model/.style={column name=Model, column type={c}},
%     columns/orig/.style={column name={1. etata}, column type={r}},
%     columns/new/.style={column name={2. etapa}, column type={r}},
%     every head row/.style={
%       before row={
%         \toprule & \multicolumn{2}{c}{$Acc_{p}\ [\%]$} \\
%       },
%       after row={
%         \cmidrule(r){1-1}
%         \cmidrule(l){2-3}
%       }
%     },
%     every last row/.style={after row={\bottomrule}},
%   ]{./ch6-realisation/tabs/0202-cross_test.csv}
%   \caption[Křížový test s~CMN.]{Křížový test modelů natrénovaných a otestovaných na datech z 1. a 2. etapy.}
%   \label{tab:realisation:verification:cross}
% \end{table}

\subsection{Eliminace vlivu kanálu}
\label{chap:realisation:corpus:elimination}

% Z prezentovaných výsledků plyne, že nová data jsou příliš odlišná od původních a v~parametrickém prostoru jsou od nich příliš vzdálena.
% Zároveň je těchto dat relativně malé množství na to, aby se mohly modely plně adaptovat.
% Na zmíněný rozdíl v~datech je možné nahlížet jako na změnu kanálu.
% Řečník je totiž stejný.
% V předchozí části bylo zmíněno, že v~rámci 2. etapy došlo ke změně nahrávací procedury a elektrolarynxu.
% Tím byl pozměněn kanál a řeč zaznamenaná v~2. etapě  má jiné parametry než ta původní z 1. etapy.
% Mezi další vlivy, které mohou způsobit změnu kanálu, je např. prostředí, ve kterém je řeč produkována, nebo přítomnost šumu na pozadí.

Na zmíněný rozdíl v~datech je možné nahlížet jako na změnu kanálu.
Řečník je totiž stejný.
K tomu, aby bylo možné použít všechna dostupná data, je tedy potřeba eliminovat vliv kanálu.
Standardně se k~tomuto účelu používá Cepstral Mean Normalisation (CMN). Principem této metody je odstranění vlivu kanálu na základě střední hodnoty kepstrálních koeficientů.

% Předpokládejme, že zaznamenaný signál $y[n]$ je možné popsat jako konvoluci promluvy a vlivu kanálu, tedy

% \begin{equation}
%   y\left[n\right] = x\left[n\right] \circledast h\left[n\right],
%   \label{eq:experiments:normalization:convolution}
% \end{equation}

% \noindent kde $x\left[n\right]$ představuje vstupní signál, tedy řeč, a $h\left[n\right]$ odezvu kanálu na jednotkový impulz.
% Ve frekvenční oblasti lze pak rovnici (\ref{eq:experiments:normalization:convolution}) zapsat ve tvaru

% \begin{equation}
%   Y\left[f\right] = X\left[f\right] \cdot H\left[f\right].
%   \label{eq:experiments:normalization:convolution:reaq}
% \end{equation}

% \noindent Pro přechod do frekvenční oblasti byla využita FFT.
% Dalším krokem je převedení hodnot do kepstrální oblasti.
% Pomocí logaritmu spektra, stejně jako v~případě MFCC parametrizace, viz \ref{chap:asr:parametrization:hearing}. V~kepstrální oblasti má vzorec (\ref{eq:experiments:normalization:convolution}), resp. (\ref{eq:experiments:normalization:convolution:reaq}) následující podobu

% \begin{equation}
%   Y\left[q\right] = \log\left(Y\left[f\right]\right) = \log\left(X\left[f\right] \cdot H\left[f\right]\right) = X\left[q\right] + H\left[q\right],
% \end{equation}

% \noindent kde $q$ představuje kepstrální koeficient.
% V kepstrální oblasti je vliv kanálu aditivní složkou výsledného záznamu.
% Problémem však je, že konkrétní hodnota vlivu kanálu je neznáma.
% K dispozici je pouze výsledný ovlivněný signál.
% Předpokládejme však, že vliv kanálu je stacionární\footnote{Jedná se sice o silný, ale logický předpoklad. Pokud se vztáhne na pořízený řečový korpus, tak v~rámci jedné etapy nahrávání je proces nahrávání neměnný, tzn. že je použita stejná aparatura a  k~nahrávání dochází vždy ve stejné místnosti.}.
% Pak je možné každý frame nahrávky $i$ popsat pomocí vztahu

% \begin{equation}
%   Y_i\left[q\right] = H\left[q\right] + X_i\left[q\right],
% \end{equation}

% \noindent kde $Y_i\left[q\right]$ představuje $i$-tý frame kepstra $q$ nahrávky a $X_i\left[q\right]$ představuje $i$-tý frame kepstra $q$ neovlivněné řeči.
% Z této rovnice je pak možné určit jeho střední hodnotu

% \begin{equation}
%   \frac{1}{N} \sum_i Y_i\left[q\right] = H\left[q\right] + \frac{1}{N} \sum_i X_i\left[q\right].
% \end{equation}

% \noindent Vliv kanálu je následně možné eliminovat odečtením této střední hodnoty kepstra $q$~od aktuální hodnoty kepstra $Y_i\left[q\right]$, konkrétně

% \begin{align}
%   R_i\left[q\right] &= Y_i\left[q\right] - \dfrac{1}{N}\sum_{j} Y_j\left[q\right] \nonumber  \\
%   &= H\left[q\right] + X_i\left[q\right] - \left( H\left[q\right] + \frac{1}{N} \sum_j X_j\left[q\right] \right) \nonumber  \\
%   &= X_i\left[q\right] - \frac{1}{N} \sum_j X_j\left[q\right].
%   \label{eq:realisation:verification:cmn}
% \end{align}

% \noindent S pomocí rovnice (\ref{eq:realisation:verification:cmn}) je možné odfiltrovat vliv kanálu a teoreticky tak získat nezkreslený signál.
% Otázkou je, přes jaký úsek počítat střední hodnotu.
% Je možné ji počítat přes posuvné okénko fixní délky, přes jednotlivé věty/nahrávky, nebo dokonce přes všechny nahrávky konkrétní etapy.
% Optimální úsek pro výpočet stř. hodnoty byl stanoven na základě provedených experimentů.

% \subsubsection{Určení délky úseku pro výpočet CMN}

% Ke stanovení vhodné délky úseku pro výpočet střední hodnoty kepstra $q$~zaznamenaného signálu byla využita stejná trénovací procedura, tj. byl
% trénován HMM-GMM model s~maximálně 4096 stavy, vstupní data byla parametrizována pomocí PLP.
% Celkem jsou uvažovány dva experimenty, a to

% \begin{itemize}
%   \item CMN počítáno pro každou nahrávku,
%   \item CMN počítáno pro celou etapu.
% \end{itemize}

% V tab. \ref{tab:realisation:verification:cmn:file} jsou uvedeny výsledky experimentu s~CMN počítaném přes jednotlivé nahrávky.
% Z dosažených výsledků je patrné, že oproti výsledkům zaznamenaným v~tab. \ref{tab:realisation:verification:cross} je dosaženo určitého zlepšení, zvláště pro případy, kdy je model natrénován na datech z 1. etapy a otestován na datech z 2. etapy.
% Výsledky však nejsou zdaleka tak dobré, jako v~případě trénování a testování modelu na datech ze stejné sady.
% Významnou roli tu hraje fakt, že zejména nahrávky izolovaných slov jsou relativně krátké a vypočtené střední hodnoty, tak nabývají odlišných hodnot.

% \begin{table}[htpb]
%   \centering
%   \def\arraystretch{1.5}
%   \pgfplotstabletypeset[
%     col sep=semicolon,
%     string type,
%     columns/model/.style={column name=Model, column type={c}},
%     columns/orig/.style={column name={1. etata}, column type={r}},
%     columns/new/.style={column name={2. etapa}, column type={r}},
%     every head row/.style={
%       before row={
%         \toprule & \multicolumn{2}{c}{$Acc_{p}\ [\%]$} \\
%       },
%       after row={
%         \cmidrule(r){1-1}
%         \cmidrule(l){2-3}
%       }
%     },
%     every last row/.style={after row={\bottomrule}},
%   ]{./ch6-realisation/tabs/0203-cmn_file.csv}
%   \caption[Křížový test s~CMN přes jednotlivé věty.]{Křížový test modelů natrénovaných a otestovaných na datech z 1. a 2. etapy s~CMN  přes jednotlivé věty.}
%   \label{tab:realisation:verification:cmn:file}
% \end{table}

% Další experiment byl proveden pro případ výpočtu CMN ze všech nahrávek konkrétní etapy.
% V tab. \ref{tab:realisation:verification:cmn:full} je vidět markantní zlepšení výsledků.
% Pokud byl model natrénován na datech z 1. etapy a otestován na datech z libovolné etapy, byly dosažené výsledky velmi podobné.
% Nejhoršího výsledku bylo dosaženo pro případ, kdy byl model natrénován na datech z 2. etapy a otestován na těch z 1.
% V tomto případě se projevil velký vliv relativně malého množství dat (pouhé 2 hodiny).
% Pokud bylo CMN počítáno přes všechny nahrávky v~dané etapě, bylo dosaženo významného zlepšení a vliv kanálu byl v~podstatě eliminován.
% Pro doplnění je nutno zmínit, že pokud byl model natrénován na celé množině všech trénovacích dat (1. a 2. etapa) a otestován pomocí kompletní testovací sady, tak byla dosažena přesnost rozpoznávání fonémovým zerogramovým jazykovým modelem rovna $Acc_{p} = 77,69~\%$.

% \begin{table}[htpb]
%   \centering
%   \def\arraystretch{1.5}
%   \pgfplotstabletypeset[
%     col sep=semicolon,
%     string type,
%     columns/model/.style={column name=Model, column type={c}},
%     columns/orig/.style={column name={1. etata}, column type={r}},
%     columns/new/.style={column name={2. etapa}, column type={r}},
%     every head row/.style={
%       before row={
%         \toprule & \multicolumn{2}{c}{$Acc_{p}\ [\%]$} \\
%       },
%       after row={
%         \cmidrule(r){1-1}
%         \cmidrule(l){2-3}
%       }
%     },
%     every last row/.style={after row={\bottomrule}},
%   ]{./ch6-realisation/tabs/0204-cmn_full.csv}
%   \caption[Křížový test s~CMN přes všechny nahrávky]{Křížový test modelů natrénovaných a otestovaných na datech z 1. a 2. etapy s~CMN  přes všechny nahrávky v~etapě.}
%   \label{tab:realisation:verification:cmn:full}
% \end{table}

% Z výsledků v~tab. \ref{tab:realisation:verification:cmn:file} plyne, že pokud by se CMN počítalo přes posuvné okénko fixní délky, tak by dosažené výsledky nebylo možné považovat za dobré.
% To se i experimentálně potvrdilo, protože výsledná přesnost rozpoznávání dosáhla hodnoty $Acc_{p} = 56,51~\%$ na kompletní trénovací i testovací sadě.
% Samotný framework Kaldi umožňuje aplikování CMVN, což je Cepstral mean and variance normalization.
% Jedná se o upravenou rovnici (\ref{eq:realisation:verification:cmn}), kde je kromě střední hodnoty počítána i variance.
% Kaldi CMVN je počítáno přes okénko fixní délky a výsledná přesnost rozpoznávání HMM-GMM modelu s~CMVN dosáhla hodnoty $Acc_{p} = 76,15~\%$ na kompletní trénovací a testovací sadě.
% Tento výsledek je srovnatelný s~modelem využívajícím výpočet CMN přes všechny nahrávky v~dané etapě.

\subsubsection{Výsledky modelů po eliminaci vlivu kanálu}

Aplikací CMN dosáhl HMM-GMM model srovnatelných výsledků s~výsledky dosaženými v~části \ref{chap:construction:results:baseline}.
Dalším krokem bylo natrénování HMM-DNN modelu.
Trénovaná neuronová FF síť měla 5 skrytých vrstev, výstupní vrstva byla typu softmax s~dimenzí rovnou počtu HMM stavů.
Postupně byla natrénována síť s~1024, 2048 a 4096 neurony v~každé skryté vrstvě.
Vstupní data byla parametrizována pomocí PLP s~12 kepstrálními, delta a delta-delta koeficienty a CMN počítané ze všech nahrávek dané etapy.
Byl využit fonémový zerogramový model\footnote{U LM je většinou základní jednotkou slovo. V tomto případě se však jedná o foném, protože je snahou odhalit problémy AM s určitým typem fonémů.}, který minimalizoval vliv jazykového modelu na přesnost rozpoznávání, a tím byl co nejvíce amplifikován vliv akustického modelu.
V tab. \ref{tab:realisation:verification:dnn} jsou zapsány dosažené výsledky všech natrénovaných variant.
Nejvyšší přesnosti dosáhl model s~4096 neurony v~každé vrstvě, ale rozdíl od ostatních variant s~menším počtem neuronů v~každé vrstvě nebyl významný.
Nejlepší HMM-DNN model dosáhl $Acc_{p} = 84,66~\%$.
% To je zlepšení o~$6,97~\%$ absolutně oproti HMM-GMM na kompletní testovací sadě.

\begin{table}[htpb]
  \centering
  \def\arraystretch{1.5}
  \pgfplotstabletypeset[
    col sep=semicolon,
    string type,
    columns/neurons/.style={column name={Počet neuronů}, column type={c}},
    columns/accuracy/.style={column name={$Acc_{p}\ [\%]$}, column type={r}},
    every head row/.style={
      before row={\toprule},
      after row={
        \cmidrule(r){1-1}
        \cmidrule(l){2-2}
        % \midrule
      }
    },
    every last row/.style={after row={\bottomrule}},
  ]{./parts/ch6-realisation/tabs/0205-dnn.csv}
  \caption[Přesnost neuronové sítě s~monofónovým zerogramovým LM.]{Dosažená přesnost neuronové sítě s~monofónovým zerogramovým jazykovým modelem.}
  \label{tab:realisation:verification:dnn}
\end{table}
