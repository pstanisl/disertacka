% !TEX root = ../../autoreferat.tex
\section{Augmentace dat}
\label{chap:realisation:augmentation}

% Poslechový test jasně ukázal, že správné rozpoznání pronesené EL promluvy není lehký úkol ani pro člověka.
% Naprosto markantní význam má kontext.
% Ten pomáhá v~případě, že některým částem promluvy nebylo dobře rozumět.
% Navíc, ze zkušeností získaných při pořizování řečového korpusu (části \ref{chap:construction:corpus} a \ref{chap:realisation:corpus}) plyne, že EL řečník má tendenci mluvit ve spíše kratších dávkách slov, mezi kterými dělá drobné pauzy.
% Pro člověka není problém udržet v~povědomí kontext, ale stroji to může někdy způsobovat problémy.
% Otázkou tedy je, jak \uv{vylepšit} stroj tak, aby poskytoval lepší výsledky.

Ať se řečník snaží sebevíc, tak i při využití metod rehabilitace hlasu uvedených v~části \ref{chap:cause:treatment}, se v~důsledku ztráty hlasivek část informace z produkované řeči ztrácí.
V poslední době bylo prezentováno několik přístupů jak ztracenou informaci obnovit.
Souhrn těch nejperspektivnějších je v~\cite{Denby2010}.
Ve valné většině případů se využívá obohacení akustického modelu o artikulační data, nebo dokonce využití jen těchto artikulačních dat \cite{Hofe2013} .
% Problém ale spočívá v~tom, že ne všechny akustické nuance mezi podobnými fonémy jsou artikulací ovlivněny.
% Pořízení záznamu artikulačních dat často vyžaduje používaní dalšího zařízení (kamery, ultrazvuku, atp. \cite{Hueber2010, Fagan2008, Jorgensen2010, Hirahara2010},
% nebo dokonce nutnost podstoupení dalšího operačního zákroku (magnety \cite{Hofe2011}).
Samozřejmě je nutno říct, že většina těchto vyvíjených systémů si klade za cíl kompletně nahradit současné metody rehabilitace hlasu.
Na druhou stranu faktem je, že ani po dlouholetém vývoji se většina těchto systémů nedostala z~rané vývojové fáze.
Nepochybně hraje roli i skutečnost, že tato problematika není v~ohnisku zájmu řečařské komunity.

Pokud tedy není úplně reálné získat ztracenou informaci pomocí kompletní změny paradigmatu fungování systémů rozpoznávání řeči, tak zbývá jen pracovat s~informací, která je  k~dispozici, a adaptovat současný model.
Určitou možností je nahrazení ztracené informace konkrétní cílenou změnou produkované řeči,
která je zohledněna modelem.
Samozřejmě takovýto přístup nezbaví řečníka elektrolarynxu, ale může mu pomoci v~situacích, které jsou pro něj stresující a~v konečném důsledku mu velmi komplikují život.

% Jako nejjednodušší možnost augmentace se jeví protažení určitých fonémů.
% Člověk je naprosto bez problémů schopen měnit tempo promluvy.
% Dokonce se tato změna velmi často děje mimoděk, protože tempo řeči velmi významně závisí na emočním a fyzickém stavu jedince.
% Pokud by se řečník naučil automaticky protahovat určité fonémy, teoreticky by to mohlo pomoci při rozpoznávání.
% U HMM modelů se délka fonému modeluje pomocí přechodu ze stavu $s_x$ do stejného stavu $s_x$, viz kapitola \ref{chap:asr:acoustic:HMM}.
% Z výsledků poskytnutých \uv{bigrams} experimentem (část \ref{chap:realisation:listening:bigrams}) se dá usuzovat, že modely fonémových párů lišících se znělostí jsou si velmi podobné.
% Protažení jednoho fonému z inkriminovaného páru může vést  k~lepšímu odlišení těchto modelů, a tím pádem i k~vyšší přesnosti rozpoznávání.

% K ověření jsou potřeba data.
Jako nejjednodušší možnost augmentace se jeví protažení určitých fonémů.
% Člověk je naprosto bez problémů schopen měnit tempo promluvy.
Bohužel získání reálných dat je časově náročný proces (viz \ref{chap:construction:corpus} a \ref{chap:realisation:corpus}).
Navíc není zřejmé, jestli se vůbec vyplatí taková data pořizovat, protože se jedná o hypotézu.
Mnohem snadněji se jeví pro účely testování možnost uměle data protáhnout v~místech výskytu zájmových fonémů.
Toto protažení je teoreticky možné realizovat dvěma způsoby:

\begin{enumerate}
  \item protažením na příznacích;
  \item protažením na zvuku.
\end{enumerate}

\noindent V~obou případech je nezbytné získat co možná nejpřesnější fonetické zarovnání. Pokud bude obsahovat chyby, tak mohou být protahovány úplně jiné úseky řeči.
% K~natrénování HMM-DNN modelu na protažených datech je zapotřebí zarovnání získané HMM-GMM nebo HMM-DNN modelem, viz \ref{chap:asr:acoustic:DNN}.
U obou variant protažení je postup stejný:

\begin{enumerate}
  \item natrénování akustického modelu na originálních datech;
  \item získání zarovnání;
  \item protažení zájmových fonémů podle zarovnání;
  \item natrénování nového akustického modelu na protažených datech.
\end{enumerate}

\noindent Nový model může být otestován a získané výsledky mohou být porovnány s~těmi dosavadními.
Tyto experimenty navíc pomohou určit vhodné hodnoty parametrů pro případné skutečné protažení dat.

\subsection{Protažení na příznacích}
\label{chap:realisation:augmentation:features}

Protažení na příznacích je založeno na skutečnosti, že při protažení (např. fonému) a následné parametrizaci, budou v~inkriminovaných mikrosegmentech po sobě následovat velmi podobné příznakové vektory.
% Teoreticky by v~krajním případě mohlo dojít i  k~situaci, kdy část těchto příznakových vektorů bude identických.
% Pokud bude cílem zjistit, zda protažení může pomoci při rozpoznávání EL řeči, tak je možné (teoreticky) docílit protažení zkopírováním určitých příznakových vektorů.
% Tato myšlenka je přiblížena na obr. \ref{fig:realisation:augmentation:features}.
V prvním kroku je nahrávka standardně parametrizována.
% Barevně jsou vyznačeny případné vektory odpovídající zájmovému úseku (tedy fonému).
% Konkrétní hranice jsou získány ze zarovnání.
Následně jsou inkrimivané vektory zduplikovány. Tím je dosaženo dvojnásobné protažení.
V reálné situaci by řečník mluvil jako doposud a k~protažení by docházelo až při zpracování, což je velmi netriviální úkol.
Jedná se spíše o teoretickou možnost protažení, která však může snadno ověřit hypotézu protažení.
% Teoreticky by musel algoritmus parametrizace být doplněn o mechanismus, který by určité příznakové vektory definovaným způsobem duplikoval.
% Problém je, že by ke kopírování docházelo v~momentě předzpracování, kdy jsou o daných příznakových vektorech k~dispozici minimální informace.
% Prosté zkopírování navíc narušuje dynamický charakter řeči.
% V rámci jednoho zpracovávaného mikrosegmentu jsou parametry považovány za statické, ale jak se okénko v~rámci zpracování posouvá, tak už nelze hovořit o stacionaritě parametrů.
% Tento problém by se musel řešit např. pomocí interpolace mezi dvěma konsekutivními vektory.
% Mechanismus by zároveň vyřešil omezení, kdy je kopírováním možné získat pouze protažení odpovídající celočíselnému násobku původní délky.
% Proč tedy vůbec zkoušet tento typ protažení?
% Odpověď je jednoduchá, nehraje zde takovou roli přesnost zarovnání.
% V průběhu zpracování je využíváno posuvného okénka a překryvu.
% Díky tomu dojde  k~určitému \uv{rozmazání} hranic.
% Pro prvotní experimenty je to pak relativně vhodné zjednodušení úlohy.

% \begin{figure}[hbpt]
%   \centering
%   \includegraphics[width=0.9\textwidth]{./ch5-construction/img/augmentation_features.pdf}
%   \caption{Princip protažení na příznacích.}
%   \label{fig:realisation:augmentation:features}
% \end{figure}

\subsubsection{Dosažené výsledky}

% Prvním bodem výše zmíněného algoritmu bylo získat standardní model, který by bylo vhodné použít  k~zarovnání dat.
% Jako vhodné se jevilo použít již natrénovaný model z~experimentů popsaných v~části \ref{chap:realisation:corpus:elimination}.
% Tento model dosáhl s~fonémovým zerogramovým LM přesnosti rozpoznávání $84,66~\%$.
% S jeho pomocí bylo získáno zarovnání trénovací i testovací sady.

Pro prvotní ověřovací experiment bylo zvoleno $2\mathrm{x}$ protažení fonému $/s/$, tedy všechny vektory odpovídající $/s/$ byly zduplikovány.
% Následně byl standardním způsobem natrénován HMM-DNN model.
Otestování bylo jako v~předchozích případech realizováno na testovací sadě s~fonémovým zerogramovým jazykovým modelem, aby byl minimalizován vliv LM.
Tento nový model dosáhl přesnosti rozpoznávání $85,11~\%$.
% , což lze označit jako malé zlepšení oproti baseline HMM-DNN modelu.

Další experiment byl realizován na protažených fonémech $/k/$, $/p/$, $/s/$, $/t/$ a $/v/$, které reprezentují většinu neznělých zájmových fonémů.
Zarovnání bylo identické jako u předchozího experimentu, stejně tak bylo uvažováno jejich $2\mathrm{x}$ protažení, tzn. že všechny vektory inkriminovaných fonémů byly zduplikovány.
Znovu byl natrénován HMM-DNN model se stejnými parametry a násedně byl otestován s~fonémovým zerogramovým jazykovým modelem.
Přesnost rozpoznávání na testovací sadě dosáhla hodnoty $87,50~\%$, což lze považovat za významné zlepšení.

V další fázi bylo potřeba ověřit, zda jiné hodnoty násobku protažení nemohou poskytnout lepší výsledek.
Experiment byl zopakován pro $3\mathrm{x}$, $4\mathrm{x}$ a $5\mathrm{x}$ jejich původní délky.
Protaženy byly fonémy $/k/$, $/p/$, $/s/$, $/t/$ a $/v/$.
Proces natrénování a otestování modelu korespondoval s~předchozími experimenty.
Dosažené výsledky byly zaznamenány do tab. \ref{tab:realisation:augmentation:influence}.
Pro úplnost byla tabulka doplněna o baseline model neobsahuhjící protažení a model s~$2\mathrm{x}$ protažením.
Z uvedených výsledků je patrný jasný trend, větší než $2\mathrm{x}$ protažení vede ke zhoršení přesnosti rozpoznávání.
Optimální hodnota protažení tak teoreticky leží někde mezi jednonásobkem a trojnásobkem původní délky.
Bohužel uvedeným postupem nelze přesně určit hodnotu míry protažení.

\begin{table}[htpb]
  \centering
  \def\arraystretch{1.5}
  \pgfplotstabletypeset[
    col sep=semicolon,
    string type,
    columns/extension/.style={
      column name={},
      column type={l},
      string replace={accp}{$Acc_{p}\ [\%]$}
    },
    columns/1x/.style={column name=$1\mathrm{x}$, column type={r}},
    columns/2x/.style={column name=$2\mathrm{x}$, column type={r}},
    columns/3x/.style={column name=$3\mathrm{x}$, column type={r}},
    columns/4x/.style={column name=$4\mathrm{x}$, column type={r}},
    columns/5x/.style={column name=$5\mathrm{x}$, column type={r}},
    every head row/.style={
      after row={
        % \cmidrule{2-10}
        \midrule
      },
      before row={\toprule & \multicolumn{5}{c}{Míra protažení} \\}
    },
    every last row/.style={after row={\bottomrule}},
  ]{./parts/ch6-realisation/tabs/0301-features.csv}
  \caption{Vliv míry protažení na přesnost modelu.}
  \label{tab:realisation:augmentation:influence}
\end{table}

% Zhoršení přesnosti u vyšších hodnot míry protažení dozajista souvisí s~faktem, že výsledná augmentovaná data neodpovídají realitě.
% Čím vícekrát je vektor zkopírován, tím více je vnášena chyba způsobená ignorováním dynamické povahy signálu.
% Nicméně jako proof-of-concept myšlenky posloužil tento experiment velmi dobře.

\subsection{Protažení na zvuku}
\label{chap:realisation:augmentation:audio}

Protažení na příznacích vedlo sice  k~významnému zlepšení přesnosti rozpoznávání, ale tento přístup není bohužel reálně použitelný. Vhodnou alternativou může být model pracující s~fonémy protaženými přímo v~audio signálu.
Taková data budou teoreticky více odpovídat reálným datům získaným od řečníka.
Stejně jako v~předchozím případě je  k~protažení potřeba zarovnání.
To s~určitou mírou přesnosti určuje počáteční a koncové hranice jednotlivých fonémů.
S ohledem na stanovené hranice je možné určitý úsek protáhnout například pomocí

\begin{itemize}
  \item převzorkování signálu,
  \item TD-PSOLA algoritmu,
  \item fázového vokodéru.
\end{itemize}

\noindent Asi nejjednodušší metodou je převzorkování dat, pro jehož realizaci stačí načíst všechny vzorky odpovídající vybranému fonému a změnit jejich vzorkovací frekvenci.
Pokud je cílem tento úsek protáhnout, je nová vzorkovací frekvence menší než originální.
Hlavním problémem této metody je tonální posun\footnote{Mění se fundamentální frekvence $F_0$. Pokud dojde ke zrychlení, frekvence se zvýší. Při zpomalení naopak sníží.}.
% Přestože protahovaný úsek tvoří jen malou část z celkové délky nahrávky, cílem je vygenerovat takové protaženín nahrávky, které co nejvíce odpovídá realitě.
% Proto není protažení pomocí převzorkování nejvhodnější metodou.

Zbylé dva uvažované přístupy využívají sofistikovanější úpravy signálu v~časové oblasti.
Obě metody využívají \textit{analýzu} signálu, pro jeho následné \textit{zpracování}, které je zakončené \textit{syntézou}.
% Metody z rodiny PSOLA pracují s~hlasivkovými pulsy, které jsou nejprve v~analytické části nalezeny\footnote{Výsledkem analýzy jsou periodicky se opakující značky, angl. pitch marks. Úpravou jejich parametrů dochází ke změnám parametrů řeči.}, aby pak v~části zpracování došlo  k~jejich transformaci na základě požadavků na výslednou řeč.
% V posledním kroku dochází k~syntéze signálu na základě upravených analytických krátkodobých signálů (hlasivkových pulsů), podrobněji v~\cite{Psutka2006}.

% Fázový vokodér pracuje na podobném principu, s~tím rozdílem, že v~analytické části dochází  k~převodu signálu do frekvenční oblasti.
% Ve fázi zpracování je signál upraven tak, aby mohl být ve fázi syntézy opět převeden do časové oblasti.

Pomocí těchto dvou zmíněných metod je možné upravit nejen délku, ale i  fundamentální frekvenci $F_0$ signálu.
% Negativně se může úprava projevit např. vznikem artefaktů, které u TD-PSOLA vznikají v~důsledku nespojitosti mezi sousedními úseky řeči; u fázového vokodéru mohou vznikat artefakty např. vlivem fázového posunu.
Obě metody poskytují velmi dobré výsledky protažení na jednotlivých fonémech.
Dosažené protažení je téměř identické.
Pro umělé protažení fonémů byla použita metoda TD-PSOLA.
% Přestože interně vyvinutý nástroj umožňuje k~ovlivnění délky řeči (a priori používaný pro účely řešení úlohy syntézy řeči) využít obě výše zmíněné metody, tak pro účely této práce byli protažení modelováno metodou TD-PSOLA.
% Ukázka původního a protaženého slova \uv{kosa} je na obr. \ref{fig:realisation:augmentation:compare}.
% Protahován byl foném $/s/$, který je v~signálu reprezentovaný šumem mezi dvěma výraznými částmi signálu.
% Inkriminovaný foném byl protažen na dvojnásobek.
% Na obr. \ref{fig:realisation:augmentation:compare:augmented} je pak zřetelně vidět protažení úseku odpovídající $/s/$.
% V signálu a ve spektru není vidět žádný významný artefakt.

% \begin{figure}[htpb]
%   \centering
%   \begin{subfigure}[b]{0.42\textwidth}
%     \includegraphics[width=\textwidth]{./ch6-realisation/img/energy_spec_word-kosa.png}
%     \caption{Originální}
%     \label{fig:realisation:augmentation:compare:original}
%   \end{subfigure}
%   %
%   \begin{subfigure}[b]{0.42\textwidth}
%     \includegraphics[width=\textwidth]{./ch6-realisation/img/energy_spec_word-kosa_real.png}
%     \caption{Protažené}
%     \label{fig:realisation:augmentation:compare:augmented}
%   \end{subfigure}
%   \caption{Amplituda a spektrogram původního (protaženého) slova \uv{kosa}.}
%   \label{fig:realisation:augmentation:compare}
% \end{figure}

\subsubsection{Dosažené výsledky s~DNN}

K ověření schopností modelu pracovat s~uměle protaženými daty byl použit stejný HMM-DNN model jako v~předchozích případech.
V datech jsou protaženy všechny výskyty fonémů $/k/$, $/p/$, $/s/$, $/t/$ a $/v/$. Uvažováno je protažení $1,25\mathrm{x}$, $1,50\mathrm{x}$, $1,75\mathrm{x}$ a $2,00\mathrm{x}$. Jazykový model je stejně jako v~případě protažení na příznacích fonémový zerogramový.
Dosažené výsledky jsou shrnuty v~tab. \ref{tab:realisation:augmentation:influence:dnn}.
Nejlepšího výsledku dosáhl \textit{baseline} model s~hodnotou $84,66~\%$.
S libovolným protažením dochází  k~poklesu přesnosti.

\begin{table}[htpb]
  \centering
  \def\arraystretch{1.5}
  \pgfplotstabletypeset[
    col sep=semicolon,
    string type,
    columns/extension/.style={
      column name={},
      column type={l},
      string replace={accp}{$Acc_{p}\ [\%]$}
    },
    columns/100/.style={column name={$1,00\mathrm{x}$}, column type={r}},
    columns/125/.style={column name={$1,25\mathrm{x}$}, column type={r}},
    columns/150/.style={column name={$1,50\mathrm{x}$}, column type={r}},
    columns/175/.style={column name={$1,75\mathrm{x}$}, column type={r}},
    columns/200/.style={column name={$2,00\mathrm{x}$}, column type={r}},
    every head row/.style={
      after row={
        % \cmidrule{2-6}
        \midrule
      },
      before row={\toprule & \multicolumn{5}{c}{Míra protažení} \\}
    },
    every last row/.style={after row={\bottomrule}},
  ]{./parts/ch6-realisation/tabs/0302-audio_1.csv}
  \caption{Vliv míry protažení fonému na přesnost DNN modelu.}
  \label{tab:realisation:augmentation:influence:dnn}
\end{table}

\subsubsection{Upravené zarovnání a time delay neural network}

Při analýze výsledků se ukázalo, že zarovnání v~mnoha případech není zrovna nejpřesnější, a to zvláště u inkriminovaných neznělých fonémů.
% Na obr. \ref{fig:realisation:augmentation:alignemnt:wrong} je ukázáno získané zarovnání slova \uv{kosa} společně s~vyznačenými hranicemi v~audiu signálu a spektru.
% Z obr. \ref{fig:realisation:augmentation:alignemnt:wrong:audio} je zřejmé, že počáteční hranice fonému $/s/$ zasahuje do předchozího fonému $/o/$.
% Tím pádem dochází  k~protažení nevhodné části signálu a model se tak učí na špatných datech.
% Pokud by všechny fonémy $/s/$ následovaly po fonému $/o/$, tak by se nejednalo o závažný problém, ale toto samozřejmě neplatí.
% V~době, kdy byly prováděny experimenty s~protažením konkrétních fonémů, se začaly stále více prosazovat time-delay neural networks (TDNN).
% Přestože patří do rodiny feed-forward sítí jako DNN, tak se oproti nim snaží vzít v~potaz i dynamickou složku řeči, podrobněji v~části \ref{chap:asr:acoustic:DNN}.

% \begin{figure}[htpb]
%   \centering
%   \begin{subfigure}[b]{0.26\textwidth}
%     \includegraphics[width=\textwidth]{./ch6-realisation/img/alignment_text.pdf}
%     \caption{Zarovnání}
%     \label{fig:realisation:augmentation:alignemnt:wrong:text}
%   \end{subfigure}
%   %
%   \begin{subfigure}[b]{0.65\textwidth}
%     \includegraphics[width=\textwidth]{./ch6-realisation/img/energy_spec_word-segment.png}
%     \caption{V signálu}
%     \label{fig:realisation:augmentation:alignemnt:wrong:audio}
%   \end{subfigure}
%   \caption{Špatně zarovnaný foném $/s/$ ve slově \uv{kosa}.}
%   \label{fig:realisation:augmentation:alignemnt:wrong}
% \end{figure}

% Stejně jako u DNN modelu je na počátku trénování nutné mít  k~dispozici zarovnání.
% To ale nemusí být naprosto přesné, protože je vstupní vektor zpracováván jiným způsobem než u DNN.
% Díky FIR filtraci a více množinám vah je brán v~potaz i dynamický charakter řeči \cite{Peddinti2015}.
% Model založený na TDNN by tak měl generovat přesnější zarovnání, tedy zlepšit výsledky modelu pracujícího s~uměle protaženými daty.

% Jako startovní bod trénování je použito DNN zarovnání z předchozího experimentu. Topologie sítě vychází z hodnot prezentovaných v~\cite{Peddinti2015}, tedy síť má $4$ skryté vrstvy. Každá vrstva má 650 neuronů. První vrstva pracuje s~kontextem $t-2$ a $t+2$, druhá vrstva s~$t-1$ a $t+2$, třetí vrstva s~$t-3$ a $t+4$ a čtvrtá s~$t-7$ a $t+2$. Výpočet výstupu pak bere v~potaz kontext $t-13$ a $t+9$ mikrosegmentů, viz \cite{Peddinti2015}.

% Na obr. \ref{fig:realisation:augmentation:alignemnt:correct} je zobrazeno získané zarovnání slova \uv{kosa} TDNN modelem.
% Z vyznačených hranic fonému $/s/$ (obr. \ref{fig:realisation:augmentation:alignemnt:correct:audio}) je patrné v~podstatě přesné zarovnání.
% Přesnost TDNN modelu s~fonémovým zerogramovým jazykovým modelem dosáhla hodnoty $Acc_{p}= 85,41~\%$.

% \begin{figure}[htpb]
%   \centering
%   \begin{subfigure}[b]{0.26\textwidth}
%     \includegraphics[width=\textwidth]{./ch6-realisation/img/alignment_text-2.pdf}
%     \caption{Zarovnání}
%     \label{fig:realisation:augmentation:alignemnt:correct:text}
%   \end{subfigure}
%   %
%   \begin{subfigure}[b]{0.65\textwidth}
%     \includegraphics[width=\textwidth]{./ch6-realisation/img/energy_spec_word-segment-2.png}
%     \caption{V signálu}
%     \label{fig:realisation:augmentation:alignemnt:correct:audio}
%   \end{subfigure}
%   \caption{Správně zarovnaný foném $/s/$ ve slově \uv{kosa}.}
%   \label{fig:realisation:augmentation:alignemnt:correct}
% \end{figure}

K získání přesnějšího zarovnání byl použit model založený na TDNN.
S přesnějším zarovnáním je možné přistoupit k~protažení fonémů $/k/$, $/p/$, $/s/$, $/t/$, $/v/$ a vytvoření nového modelu pracujícího s~těmito daty.
Jako další model je tedy použita TDNN síť.
K otestování modelu je využit standardní fonémový zerogramový jazykový model.
Uvažováno je protažení od $1,25\mathrm{x}$ do $3,00\mathrm{x}$ s~krokem $0,25$. Výsledky experimentu jsou uvedeny v~tab. \ref{tab:realisation:augmentation:influence:tdnn}.
Oproti hodnotám rozpoznávání přesnosti v~tab. \ref{tab:realisation:augmentation:influence:dnn} je vidět výrazné zlepšení přesnosti oproti baselinu modelu ($Acc_{p} = 85,41~\%$).
Nejvyšší přesnosti $87,90~\%$ dosáhl model pracující s~$2,5\mathrm{x}$ protaženými daty, navíc modely pracující s~protažením od $1,75\mathrm{x}$ do $2,75\mathrm{x}$ dosahují velmi blízkých hodnot přesnosti rozpoznávání.
To poskytuje relativně široký pracovní interval pro případné skutečné protažení dat řečníkem.

\begin{table}[htpb]
  \centering
  \def\arraystretch{1.5}
  \pgfplotstabletypeset[
    col sep=semicolon,
    string type,
    columns/extension/.style={
      column name={},
      column type={l},
      string replace={accp}{$Acc_{p}\ [\%]$}
    },
    columns/100/.style={column name={$1,00\mathrm{x}$}, column type={r}},
    columns/150/.style={column name={$1,50\mathrm{x}$}, column type={r}},
    columns/125/.style={column name={$1,25\mathrm{x}$}, column type={r}},
    columns/175/.style={column name={$1,75\mathrm{x}$}, column type={r}},
    columns/200/.style={column name={$2,00\mathrm{x}$}, column type={r}},
    columns/225/.style={column name={$2,25\mathrm{x}$}, column type={r}},
    columns/250/.style={column name={$2,50\mathrm{x}$}, column type={r}},
    columns/275/.style={column name={$2,75\mathrm{x}$}, column type={r}},
    columns/300/.style={column name={$3,00\mathrm{x}$}, column type={r}},
    every head row/.style={
      after row={
        % \cmidrule{2-10}
        \midrule
      },
      before row={\toprule & \multicolumn{9}{c}{Míra protažení} \\}
    },
    every last row/.style={after row={\bottomrule}},
  ]{./parts/ch6-realisation/tabs/0303-audio_2.csv}
  \caption{Vliv míry protažení fonému na přesnost TDNN modelu.}
  \label{tab:realisation:augmentation:influence:tdnn}
\end{table}

Podstatnou otázkou je robustnost modelu.
S ohledem na hodnoty přesnosti rozpoznávání uvedené v~tab. \ref{tab:realisation:augmentation:influence:tdnn:robust} se dá říci, že model je v~rámci možností robustní v~poměrně širokém rozsahu protažení.
Očekávaným výsledkem je nejnižší hodnota přesnosti rozpoznávání pro neprotažená data (konkrétně $78,61~\%$).
Většina chyb v~rozpoznávání nastala v~důsledku neznalosti inkriminovaných neprotažených fonémů.
Tento výsledek je předpokladem pro funkci trenažéru prezentovaného v~části \ref{chap:realisation:trainer}.

\begin{table}[htpb]
  \centering
  \def\arraystretch{1.5}
  \pgfplotstabletypeset[
    col sep=semicolon,
    string type,
    columns/extension/.style={
      column name={},
      column type={l},
      string replace={accp}{$Acc_{p}\ [\%]$}
    },
    columns/100/.style={column name={$1,00\mathrm{x}$}, column type={r}},
    columns/150/.style={column name={$1,50\mathrm{x}$}, column type={r}},
    columns/125/.style={column name={$1,25\mathrm{x}$}, column type={r}},
    columns/175/.style={column name={$1,75\mathrm{x}$}, column type={r}},
    columns/200/.style={column name={$2,00\mathrm{x}$}, column type={r}},
    columns/225/.style={column name={$2,25\mathrm{x}$}, column type={r}},
    columns/250/.style={column name={$2,50\mathrm{x}$}, column type={r}},
    columns/275/.style={column name={$2,75\mathrm{x}$}, column type={r}},
    columns/300/.style={column name={$3,00\mathrm{x}$}, column type={r}},
    every head row/.style={
      after row={
        % \cmidrule{2-10}
        \midrule
      },
      before row={\toprule & \multicolumn{9}{c}{Míra protažení} \\}
    },
    every last row/.style={after row={\bottomrule}},
  ]{./parts/ch6-realisation/tabs/0305-robust.csv}
  \caption{Robustnost nejlepšího TDNN modelu ($2,5\mathrm{x}$) na míru protažení.}
  \label{tab:realisation:augmentation:influence:tdnn:robust}
\end{table}

Experimenty s~uměle protaženými daty potvrdily správnost uvažované hypotézy.
Protažením jednoho z párových fonémů dojde  k~dostatečnému odlišení velmi podobných zvukových reprezentací.
Tím dojde  k~natrénování odlišných modelů fonémů v~HMM.
Model pracující s~fonémy protaženými přímo ve zvuku nakonec dosáhl lepších výsledků než model s~uměle protaženými daty na příznacích.
% Svůj podíl na tom má i použití TDNN modelu.
% Model pracující s~duplikovanými příznaky naznačoval, že optimální hodnota protažení bude v~intervalu jedno až trojnásobku, což druhý typ modelu potvrdil.

\subsection{Aktualizace výsledků porovnání}
\label{chap:realisation:augmentation:comparison}

V části \ref{chap:realisation:listening} byla prezentováno srovnání schopností člověka a stroje.
Posloužily k~tomu dva poslechové testy a celkem $3$ ASR experimenty \uv{one-mil}, \uv{reduced} a \uv{bigrams}.
S využitím nového modelu je možné aktualizovat hypotetické výsledky stroje.
Hypotetické z toho důvodu, že použitá data jsou uměle protažena.
% Nicméně to nebrání provedení tohoto experimentu.
% Získané hodnoty mohou být brány jako jakási teoretická maxima ASR systému.
% Uměle upravená data budou relativně přesně a konzistentně protažena, u reálných dat toto nelze a priori očekávat.
K aktualizaci výsledků byl použit
%nejlepší
model z části \ref{chap:realisation:augmentation:audio}, tedy ten s~$2,5\mathrm{x}$ protaženými daty.
Dosažené výsledky byly zaznamenány do tab. \ref{tab:realisation:augmentation:comparison}.
Ve všech třech případech došlo  k~významnému zlepšení.
Výsledky člověka se nezměnily, protože realizace poslechového testu je zdlouhavý proces a je obtížné získat dostatečný počet respondentů.
Nicméně se dá očekávat, že i člověk by dosáhl zlepšení.
Pokud by měl informaci o fonémech, které jsou protaženy.
% Parametry experimentů byly totožné s~těmi v~části \ref{chap:realisation:listening}.
% V případě experimentu \uv{one-mil} byl použit zerogramový jazykový model s~1 milionem slov, experiment \uv{reduced} pak pouze zerogramový LM se slovy obsaženými v~poslechovém testu ($N = 320$).
% Speciální LM, obsahující 4 kombinace slov, byl vygenerován pro každou položku \uv{bigrams} experimentu.

% Dosažené výsledky byly zaznamenány do tab. \ref{tab:realisation:augmentation:comparison}.
% Ve všech třech případech došlo  k~významnému zlepšení.
% U experimentu \uv{one-mil} to je o~$23~\%$ absolutně, u~\uv{reduced} experimentu pak o~$24~\%$.
% K nejmarkantnějšímu zlepšení došlo u experimentu \uv{bigrams}, dokonce o~$44~\%$.
% Výsledky člověka se nezměnily, protože realizace poslechového testu je zdlouhavý proces a je obtížné získat dostatečný počet respondentů.
% Nicméně se dá očekávat, že i člověk by dosáhl zlepšení.
% Pokud by měl informaci o fonémech, které jsou protaženy.
% V opačném případě by ke zlepšení nutně nemuselo dojít, protože kromě protažení nebyl zvuk nijak pozměněn.

% Velmi zajímavé je porovnání zvýšení přesnosti rozpoznávání TDNN modelu s~fonémovým zerogramovým LM ($2,49~\%$ absolutně mezi baseline a $2,5\mathrm{x}$ modelem, viz tab. \ref{tab:realisation:augmentation:influence:tdnn}) a dosaženými výsledky prezentovanými v~tab. \ref{tab:realisation:augmentation:comparison}.
% Oproti nim je zlepšení o~$2,49~\%$ absolutně poměrně zanedbatelné, přesto velmi významné zlepšení.
% Jasně to ukazuje ideu experimentů s~fonémovým zerogramovým jazykovým modelem.
% I drobné zlepšení u akustického modelu může vést  k~rapidnímu zlepšení sofistikovanějšího systému.

\begin{table}[htpb]
  \centering
  \def\arraystretch{1.5}
  \pgfplotstabletypeset[
    col sep=semicolon,
    string type,
    columns/type/.style={column name=, column type={l}},
    columns/onemil/.style={column name={one-mil}, column type={r}},
    columns/reduced/.style={column name=reduced, column type={r}},
    columns/bigrams/.style={column name=bigrams, column type={r}},
    every head row/.style={
      after row={
        % \cmidrule(lr){1-1}
        % \cmidrule(lr){2-4}
        \midrule
      },
      before row={\toprule & \multicolumn{3}{c}{$Acc_{p}\ [\%]$} \\}
    },
    every last row/.style={
      after row={\bottomrule},
      before row={\midrule}
    },
  ]{./parts/ch6-realisation/tabs/0304-comparison.csv}
  \caption{Aktualizované porovnání dosažených výsledků člověka a stroje.}
  \label{tab:realisation:augmentation:comparison}
\end{table}

\subsection{Reálně protažená data}
\label{chap:realisation:augmentation:real}

Získané výsledky s~uměle protaženými daty potvrdily hypotézu, že model pracující s~těmito daty může dosahovat lepších výsledků.
Dalším krokem bylo získání reálně protažených dat.
% Nahrávání je relativně zdlouhavý proces,
% proto je nezbytné dobře vybrat správné promluvy.
% Problémem je navíc samotné protažení.
% Pokud by bylo cílem získat protažené celé slovo, lze řečníka instruovat, aby mluvil pomaleji.
% To ale cílem není.
% Výsledné promluvy mají mít protažené pouze určité fonémy, a to přibližně na dvojnásobek.
% Jako nejjednodušší, a svým způsobem i elegantní, se ukázal zápis se zdvojenými písmeny, která mají představovat protažený foném, např. \uv{kossa}.
% Řečník je obeznámen s~tím, že pokud slovo v~promluvě obsahuje tento dvojitý zápis, měl by se pokusit toto slovo patřičným způsobem protáhnout.
% Tento zápis navíc řečníka podvědomě \uv{nutí} vyslovit slovo jinak než v~případě normálního zápisu.
Nahrávání se zhostil stejný řečník jako v~1. a 2. etapě.
Tedy žena v~důchodovém věku, která používá EL v~běžném životě již více než 15 let.
Nahrávání se uskutečnilo v~průběhu 5 měsíců od července 2018 do listopadu 2018.
Texty určené k~nahrávání obsahovaly většinu izolovaných slov z poslechového testu a věty, které doposud neobsahuje řečový korpus složený z 1. a 2. etapy nahrávání.
Řečník byl instruován, aby slova, která obsahují zdvojená písmena (např. \uv{kossa}), adekvátně prodloužil.
% Nahrávání bylo realizováno za stejných podmínek jako v~2. etapě nahrávání (viz \ref{chap:realisation:corpus}).
% Nahrávání byl vždy přítomen operátor, který kontroloval, zda se řečník nevědomky nedopustil chyby či přeřeknutí.
% Stejně jako v~předchozí etapě obsahuje každá nahrávka minimálně $0,5\ s$ pauzu na začátku a konci.
% Celkem se takto v~rámci 3. etapy nahrávání podařilo získat dohromady $998$ promluv obsahující věty a slova (protažená i neprotažená), kterých je $267$.
Celkový čas promluv v~3. etapě dosáhl $2$ hodin a $28$ minut.

% Na obr. \ref{fig:realisation:real:compare} je zobrazena amplituda a spektrogram slova \uv{kose} protaženého řečníkem (\ref{fig:realisation:real:compare:real}) a 2x uměle (\ref{fig:realisation:real:compare:augmented}).
% Hlavním viditelným rozdílem je slabší zastoupení frekvencí kolem $2\ kHz$ ve spektrogramu mezi časem $3,2\ s$ a $3,4\ s$.
% Ač to tak na první pohled nevypadá, tak obě protažení mají téměř identickou délku $0,258\ s$ (reálné) vs. $0,261\ s$ (umělé).
% Vizuální rozdíl je způsoben vyšší celkovou délkou reálně protaženého slova.
% Foném $/s/$ je zástupcem neznělých fonémů, zdrojem zvuku je tedy a priori šum a nikoli periodický signál produkovaný hlasivkami, tím pádem je velikost amplitudy taktéž velmi podobná.
% Další viditelný rozdíl v~místě protažení fonému $/s/$ je způsoben vyšší maximální amplitudou u uměle protažené nahrávky (\ref{fig:realisation:real:compare:augmented}), která byla pořízena v~průběhu 2. etapy nahrávání.

% \begin{figure}[htpb]
%   \centering
%   \begin{subfigure}[b]{0.42\textwidth}
%     \includegraphics[width=\textwidth]{./ch6-realisation/img/energy_spec_word-kose_real.png}
%     \caption{Protažené řečníkem}
%     \label{fig:realisation:real:compare:real}
%   \end{subfigure}
%   %
%   \begin{subfigure}[b]{0.42\textwidth}
%     \includegraphics[width=\textwidth]{./ch6-realisation/img/energy_spec_word-kose_augmented.png}
%     \caption{Uměle protažené}
%     \label{fig:realisation:real:compare:augmented}
%   \end{subfigure}
%   \caption{Amplituda a spektrogram slova \uv{kose} protažené řečníkem/uměle.}
%   \label{fig:realisation:real:compare}
% \end{figure}

Analýza pořízených slov ve 3. etapě nahrávání ukázala, že proces umělého protažení produkuje svými charakteristikami velmi podobné nahrávky těm reálným.
Bohužel pro vytvoření modelu pouze z reálně protažených dat se však nepodařilo získat dostatečné množství dat.
Pokud jsou reálně protažená data opravdu velmi podobná uměle protaženým datům, tak by mělo být možné dosáhnout dobrých výsledků s~modelem, který je natrénovaný na uměle protažených datech, ale otestovaný těmi reálně protaženými.

% Porovnáním neprotažených nahrávek z 2. etapy a protažných nahrávek z 3. etapy se jako ideální jeví použití $2\mathrm{x}$ protaženého modelu z části \ref{chap:realisation:augmentation:audio} (viz tab. \ref{tab:realisation:augmentation:influence:tdnn}).
% Jedná se tedy o TDNN model, který je natrénován na datech obsahujících $2\mathrm{x}$ protažené fonémy $/k/$, $/p/$, $/s/$, $/t/$ a $/v/$.
% Na testovací sadě dosáhl přesnosti $Acc_{p} = 87,71~\%$.
% K otestování výkonu na reálně protažených datech jsou použity všechny věty a slova obsahující protažení zmíněných fonémů\footnote{Stejně jako u 1. a 2. etapy bylo i zde aplikováno CMN napočítané přes všechny nahrávky v~rámci etapy.}.
% S touto testovací sadou dosáhl zmíněný model s~fonémovým zerogramovým jazykovým modelem $Acc_{p} = 84,51~\%$.

% Dosažená přesnost je sice horší něž původní přesnost na uměle protažených datech, ale nedošlo  k~dramatickému propadu jako v~případě křížového testu mezi 1. a 2. etapou (viz tab. \ref{tab:realisation:verification:cmn:full}\footnote{V tabulce jsou prezentovány hodnoty odpovídající GMM modelu. Srovnání na první pohled není úplně korektní, nicméně validní, protože GMM model trénovaný na uměle protažených datech a otestovaný na reálně protažených datech dosáhl $Acc_{p} = 81,29~\%$, což je srovnatelná hodnota s~ostatními GMM modely.}).
% Dosažený výsledek tak potvrzuje podobnost uměle a reálně protažených dat.

V případě, že se k~trénovací sadě s~$2\mathrm{x}$ promluvami přidaly věty z 3. etapy, které obsahují uměle protažené fonémy, výsledná přesnost TDNN modelu s~fonémovým zerogramovým jazykovým modelem dosáhla hodnoty $Acc_{p} = 85,88~\%$.
To je lepší hodnota než v~případě baseline TDNN model ($Acc_{p} = 85,41~\%$, viz tab. \ref{tab:realisation:augmentation:influence:tdnn}).
Dosažená přesnost je sice horší něž původní přesnost na $2\mathrm{x}$ uměle protažených datech ($Acc_{p} = 87,71~\%$), ale rozdíl není tak markantní vezme-li se v potaz rozšíření testovací sady.
Tyto experimenty podporují ideu trenažéru prezentovaného v~části \ref{chap:realisation:trainer}.
