\ifdefined\CELE
\else
\documentclass[a4paper,12pt]{scrreprt}

\usepackage[utf8]{inputenc}
\usepackage[czech]{babel}
\usepackage[T1]{fontenc}
\usepackage{amsmath}
\usepackage{amsfonts}
\usepackage{amssymb}
\usepackage{graphicx}
\usepackage{longtable}
\author{Jan Bartošek}
\usepackage{cite}
\usepackage{url}
\usepackage{subfig}
\usepackage{multibib}
\usepackage[hidelinks,unicode]{hyperref}

%\usepackage{epsfig}
%\usepackage{epstopdf}

\linespread{1.1}
\usepackage{a4wide}

%\usepackage[raggedright]{titlesec} % zamezeni deleni nadpisu

%\newcites{all}{Seznam literatury}
\newcites{my}{Soupis publikací}

\newcommand*{\tabh}[1]{\multicolumn{1}{|c|}{#1}}

\captionsetup{justification=centering}

\begin{document}

\fi

\chapter*{Závěr}
\addcontentsline{toc}{chapter}{Závěr}
\label{chap:conclusion}

Předložená disertační práce se zabývá problematikou rozpoznávání řeči pacientů po totální laryngektomii, kteří komunikují pomocí elektrolarynxu. Motivací pro zpracování tohoto tématu bylo obohatit stávacíjí postupy využívané pro rehabilitaci hlasu o možnosti, které přináší využití moderních technologií, konkrétně možností automatického rozpoznávání řeči. V~kapitole \ref{chap:mot} byly vytyčeny cíle práce, jejichž naplnění bylo v~následujících odstavcích zhodnoceno.

V kapitole \ref{chap:cause} jsou přiblíženy nejčastější příčiny ztráty hlasu a metody užívané k~jeho rehabilitaci.
Pomocí klasických rehabilitačních technik lze pacientům navrátit možnost mluvit, ale kvalita produkované řeči nutně nemusí splňovat očekávání pacientů  a požadavky kladené na mluvčího okolím. Například použití
elektrolarynxu sice neklade na uživatele vysoké nároky co se týče učení, ale
kvalita řeči není vůbec přirozená. Oproti tomu pomocí jícnového hlasu je
produkován relativně kvalitní hlas, ale k~edukaci je potřeba vynaložit opravdu
nemalé úsilí. Jako ideální se může jevit použití tracheoezofageální píštěle,
která umožňuje proudění vzduchu z plic do dutiny ústní. Produkovaný hlas se v
tomto případě vyznačuje vysokou kvalitou, dobrou srozumitelností,
individuálním zabarvením a relativně dlouhou fonační dobou. Za nedostatek se
dá považovat nutnost pravidelně čistit a měnit píštěle. Existují i další
metody,
% popsané v~části \ref{chap:cause:treatment},
ale ty jsou zatím používány
spíše autorskými týmy a o masovém použití se rozhodně nedá hovořit. Bohužel
žádná z~technik nepředstavuje univerzální řešení, a proto se je lékaři
stále snaží zdokonalovat, a tím zkvalitňovat život pacientů. U všech aktuálně používaných metod rehabilitace hlasu
je patrný významný negativní dopad na psychiku pacienta, který se musí vyrovnat nejen se ztrátou vlastního hlasu, ale i s~ostychem, který provází opětovné snahy mluvit.

Pomoc s~rehabilitací hlasu mohou poskytnout řečové technologie zpracovávající přirozenou řeč. V~současnosti využívané obecné systémy automatického rozpoznávání řeči (ASR systémy) ale poskytují spolehlivé výsledky v~případě rozpoznávání promluv zdravého řečníka. V~případě, že se charakteristiky rozpoznávané řeči příliš líší ( např. řeč obsahuje vyšší množství šumu), může se u běžně uživaných ASR systémů projevit jejich nedostačná robustnost. Proto bylo pro potřeby návrhu ASR systému, který bude sloužit pro rozpoznávání řeči pacientů po totální laryngektomii, nutno pořídit řecová data odpovídající kvality. V~takovýchto promluvách se ukazuje jako problematická zejména přílišná podobnost produkovaných znělých a neznělých fonémů. Proto byla navázana spolupráce s~mluvčí, která prodělala TL a komunikuje pomocí elektrolarynxu. V~průběhu pěti let byly pořízeny 3 sady nahrávek, což odpovídá necelým 15 hodinám řečových dat. První sada je složená z vět, které jsou součástí interně využívaného řečového korpusu, 2. sada rozšiřuje řečový korpus o další sadu vět a problematických izolovaných slov. Ve 3. sadě jsou obsaženy další věty a slova respektující protažení problematických fonémů.
S~ohledem na toto lze i vytyčený cíl č. 2 považovat za naplněný.

Pro otestování robustnosti obecného ASR systému byla využita data poskytnutá řečníkem po totální laryngektomii. Testovaný systém vykázal pro trigramový jazykový model obsahující 1 milion unikátních slov úspěšnost rozpoznávání slov pouze $\boldsymbol{18,49~\%}$. Takto nízká přesnost rozpoznávání indikovala nutnost navrhnout individuální akustický model, který bude reflektovat specifika řeči produkované s~využitím EL. Při využití dat z EL korpusu dosáhl systém přesnosti rozpoznávání slov $\boldsymbol{83,33~\%}$.

Následně byl minimalizován vliv jazykového modelu,
% viz \ref{chap:construction:results},
a byly hledány optimální parametry akustického modelu. Byl ověřen vliv maximálního počtu unikátních stavů HMM modelu a vzorkovací frekvence na přesnost rozpoznávání. Nejlepších výsledků bylo dosaženo pro model pracující s~maximálně 4096 unikátními stavy a vzorkovací frekvencí $16\ kHz$. Pro HMM-GMM bylo dosaženo přesnosti rozpoznávání $\boldsymbol{81,20~\%}$, pro HMM-DNN pak $\boldsymbol{85,23~\%}$. Po provedení analýzy získaných výsledků se jako problematické ukázalo rozpoznávání neznělých fonémů, proto bylo přistoupeno  k~redukci fonetické sady prostřednictvím různých kombinací náhrady neznělých fonémů za znělé, což ve většině případů nemělo pozitivní dopad. Proto byly navrženy další úpravy ASR systému, konkrétně protaženy neznělé fonémy a následně byl ASR systém rozšířen o tzv. duration model akcentující právě délku fonémů. Na základě provedených experimentů se jako vhodné ukázalo prodloužit neznělé fonémy na dvojnásobek jejich původní délky. Úspěšnost rozšířeného modelu dosahovala $\boldsymbol{88,54~\%}$.

S ohledem na získané výsledky vyvstala potřeba porovnat schopnosti rozpoznávání člověka a stroje. Za tímto účelem byl navržen tzv. poslechový test, jehož princip byl přiblížen v~části \ref{chap:realisation:listening}. Pro model s~výše navrženými optimálními parametry ( max. 4096 unikatní stavů a vzorkovací frekvencí $16\ kHz$) stroj dosál přesnosti rozpoznávání  $\boldsymbol{69,91~\%}$ pro případ izolovaných slov a $\boldsymbol{54,82~\%}$ pro případ bigramů. U člověka bylo dosaženo přesnosti $\boldsymbol{74,47~\%}$, resp. $\boldsymbol{66,24~\%}$. Při zohlednění umělého protažení akustických dat dosáhl stroj přesnosti $\boldsymbol{94,36~\%}$ pro izolovaná slova, resp. $\boldsymbol{98,80~\%}$ pro bigramy. Po následném rozšíření systému rozpoznávání o duration model dosáhl stroj úspěšnosti $\boldsymbol{94,42~\%}$, resp. $\boldsymbol{98,81~\%}$. S ohledem na výsledky poskytnuté ASR systémem byla nahrána další sada řečových dat respektující protažení neznělých fonémů. Na takto rozšířené testovací sadě bylo dosaženo přesnosti rozpoznávání $\boldsymbol{87,02~\%}$ na úrovni fonémů. Tím bylo ověřeno, že model natrénovaný na uměle protažených datech je schopen rozpoznávat i data reálná, a lze ho tedy využít jako základ pro vývoj trenažéru, který bude výukovým nástrojem pro osvojení schopnosti řečníka protahovat neznělé fonémy.

S ohledem na výše uvedené výsledky a z~nich vyvozené závěry lze říci, že vytyčené cíle disertační práce byly naplněny a s~ohledem na aktuálnost řešené problematiky lze získané poznatky využít jako základ další práce.

\ifdefined\CELE
\else
%\bibliographystyle{plain}
%\bibliography{literatura}

%\bibliographystyleall{plain}
%\bibliographyall{literatura.bib}
%\bibliographymy{literatura}
\end{document}

\fi
