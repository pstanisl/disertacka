%!TEX root = ../autoreferat.tex
\ifdefined\CELE
\else
\documentclass[a4paper,12pt]{scrreprt}

\usepackage[utf8]{inputenc}
\usepackage[czech]{babel}
\usepackage[T1]{fontenc}
\usepackage{amsmath}
\usepackage{amsfonts}
\usepackage{amssymb}
\usepackage{graphicx}
\usepackage{longtable}
\author{Jan Bartošek}
\usepackage{cite}
\usepackage{url}
\usepackage{subfig}
\usepackage{multibib}
\usepackage[hidelinks,unicode]{hyperref}

%\usepackage{epsfig}
%\usepackage{epstopdf}

\linespread{1.1}
\usepackage{a4wide}

%\usepackage[raggedright]{titlesec} % zamezeni deleni nadpisu

%\newcites{all}{Seznam literatury}
\newcites{my}{Soupis publikací}

\newcommand*{\tabh}[1]{\multicolumn{1}{|c|}{#1}}

\captionsetup{justification=centering}

\begin{document}

\fi

\chapter*{Anotace}

Disertační práce se zabývá problematikou rozpoznávání řeči pacientů, kteří podstoupili totální laryngektomii, a  k~produkci hlasu využívají elektrolarynx. V první části práce jsou přiblíženy důvody ztráty hlasu a metody, které jsou v současnosti využívány pro jeho rehabilitaci spolu s~jejich principy. Významnou pomoc s~rehabilitací hlasu mohou poskytnout řečové technologie zpracovávající přirozenou řeč. Z tohoto důvodu jsou v práci popsány metody, které jsou využívány pro  konstrukci automatických systémů rozpoznávání řeči (ASR). S~ohledem na specifika řeči generované za pomoci elektrolarynxu je v práci prezentován postup pro sestavení speciálního řečového korpusu složeného z nahrávek hlasu pacienta po totální laryngektomii. Specifická řečová data slouží následně pro otestování robustnosti obecného systému rozpoznávání řeči. Získané výsledky však indikují potřebu navrhnout speciální ASR systém s~individuálními požadavky vzhledem ke specifikům rozpoznávané řeči. Následně je navrženo několik postupů úpravy akustických dat za účelem zvýšení přesnosti rozpoznávání. Jako nejúčinnější se ukázalo protažení neznělých fonémů, proto byl vyvíjený ASR systém rozšířen o modul zohledňující právě toto protažení. V~práci je popsáno nemalé množství experimetů, které byly provedeny za účelem ověření dílčích hypotéz.

\ifdefined\CELE
\else
%\bibliographystyle{plain}
%\bibliography{literatura}

%\bibliographystyleall{plain}
%\bibliographyall{literatura.bib}
%\bibliographymy{literatura}
\end{document}

\fi

% \clearpage

% \section*{Abstract}

% This dissertation thesis deals with speech recognition of patients who have undergone total laryngectomy and who use electrolarynx for voice production. In the first part of the thesis, the reasons of voice loss and methods that are currently used for its rehabilitation together with their principles are described. Significant help with voice rehabilitation can be provided by speech technologies processing natural speech. For this reason, the methods that are used for the construction of automatic speech recognition systems (ASR) are described. Regarding the specifics of speech generated with the help of electrolarynx, the paper presents a procedure for assembling a special speech corpus consisting of recordings of the patient's voice after total laryngectomy. The specific speech data is then used to test the robustness of the general speech recognition system. However, the obtained results indicate the need to design a special ASR system with individual requirements due to the specifics of the speech. Subsequently, several methods for adjusting acoustic data are proposed to increase recognition accuracy. The elongation of voiceless phonemes proved to be the most effective, so the developed ASR system was extended by a module taking into account this elongation. The work describes a considerable number of experiments, which were conducted to verify partial hypotheses.

% This is a \LaTeX{} template and document class for Ph.D. dissertations at Princeton University. It was created in 2010 by Jeffrey Dwoskin, and adapted from a template provided by the math department. Their original version is available at: \url{http://www.math.princeton.edu/graduate/tex/puthesis.html}

% This is \textbf{NOT} an official document. Please verify the current Mudd Library dissertation requirements~\cite{mudd2009} and any department-specific requirements before using this template or document class.


% Your abstract can be any length, but should be a maximum of 350 words for a Dissertation for ProQuest's print indicies (or 150 words for a Master's Thesis); otherwise it will be truncated for those uses~\cite{proquest2006}.


% Dwoskin Ph.D. Dissertation Template --- version 1.0, 5/19/2010
