% !TEX root = ../thesis.tex
\chapter{Motivace a cíle disertační práce}
%\chapter{Příčiny ztráty hlasu a možnosti jeho rehabilitace}
\label{chap:mot}

\begin{enumerate}
  \item Seznamte se s~přístupy, které umožňují alespoň částečnou obnovu schopnosti řečové komunikace u pacientů po totální laryngektomii (TL).
  \item Pro účely konstrukce systému automatického rozpoznávání řeči u lidí po totální laryngektomii využívajících pro komunikaci elektrolarynx navrhněte a pořiďte vhodný korpus řečových nahrávek.
  \item Natrénujte základní systém rozpoznávání řeči pro jednoho řečníka - pacienta po totální laryngektomii mluvícího pomocí elektrolarynxu - a porovnejte funkcionalitu systému (zejména jeho přesnost) se systémem rozpoznávajícím řeč zdravých lidí. Ke konstrukci systému využijte state-of-the-art metody.
  \item Analyzujte základní příčiny případné zvýšené chybovosti realizovaného systému rozpoznávání řeči a pokuste se navrhnout vhodné úpravy v~jeho konstrukci, které chybovost sníží. Diskutujte vhodnost navrženého řešení.
\end{enumerate}
