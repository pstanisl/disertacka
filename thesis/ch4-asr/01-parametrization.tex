% !TEX root = ../thesis.tex
\section{Parametrizace řečového signálu}
\label{chap:asr:parametrization}

Jako v mnoha jiných odvětvích, tak i při rozpoznávání řeči je v mnoha případech inspirací člověk. Pro získání sekvence pozorování (příznaků) vycházíme z \textbf{modelování produkce řeči} a \textbf{modelování procesu slyšení}, které se inspirují právě člověkem.

\subsection{Modelování produkce řeči}
\label{chap:asr:parametrization:production}

Cílem modelování produkce řeči je nalezení matematických vztahů, které poslouží k reprezentaci fyzikálních dějů spojených s produkcí řeči. Základem je parametrizační technika \textbf{lineárního prediktivního kódování}, známá pod anglickou zkratkou LPC\footnote{Linear Predictive Coding} \cite{Benesty2007}. Vychází z představy, že hlasové ústrojí člověka je schopno vytvářet tři různé typy řečových zvuků:

\begin{itemize}
  \item \textit{samohlásky} - ty se řadí mezi znělé typy zvuků produkované periodickým buzením vznikajícím pulsy vzduchu, které jsou produkovány hlasivkami;
  \item \textit{frikativy} (např. $/f/$\footnote{Zápis $/f/$ symbolizuje foném, což je akustická reprezentace písmene, \textit{f}. Konkrétní zápisy se mohou lišit podle použité fonetické abecedy. V Čechách se nejčastěji používá abeceda $SAMPA$ či $Z\check{C}FA$.}) - někdy nazývané jako třené souhlásky, protože vznikají třením výdechovaného proudu vzduchu o překážku v některém místě hlasového ústrojí. Těmito překážkami může být jazyk, zuby ap.;
  \item \textit{explozivy} (např. $/b/$, $/p/$ ap.) - také nazývané jako souhlásky výbuchové, se tvoří úplnýn uzavřením vydechovaného proudu vzduchu pomocí artikulačních orgánů. To se následně projeví jako krázká pauza (tzv. okluze), po které následuje náhlé jednorázové uvolnění a únik nahromaděného vzduchu, tzv. exploze \cite{Psutka2006};
\end{itemize}

Snahou modelování je navržení modelu hlasového traktu, který bude dobře popisovat výše popsané zvuky. Nesmí se však zapomenout na možnou přílišnou složitost~a nedostatečnou přesnost modelu. Jako ideální se může jevit lineárně časově invariantní model. Bohužel lidská řeč je zástupcem kontinuálního časově variantní a~některých situacích dokonce nelineárního proces, takže je téměř nemožné jej přesně namodelovat. Pokud se však, za určitého předpokladu, provedou určitá zjednodušení, tak je možné navrhnout lineární časově invariantní model řeči, který je platný pro krátké časové úseky. Jinými slovy, tímto předpokladem je, že v konkrétním krátkém časovém úseku zůstává buzení a parametry hlasivkového traktu přibližně konstantní. Tuto podmínka lze považovat za platnou pro intervaly od $10$ do $30\ ms$. Odtud také vychází uvažovaná perioda segmentů řeči, viz úvod této kapitoly. Pro tyto segmenty je pak možné proces vytváření řeči modelovat pomocí tzv. \textbf{krátkodobého modelu}, který má v krátkých časových intervalech pevné parametry. \cite{Holmes2001}

\begin{figure}[hbpt]
  \centering
  \includegraphics[width=0.9\textwidth]{./ch4-asr/img/speech_model.pdf}
  \caption{Blokové schéma modelu produkce řeči}
  \label{fig:asr:model:speech}
\end{figure}

Pro odvození obecného diskrétního modelu hlasivkového traktu se vychází ze zjednodušeného modelu produkce řeči (obr. \ref{fig:asr:model:speech}). Ten je tvořen modelem hlasivek, modelem hlasivkového traktu a modelem vyzařovaného zvuku, které jsou seriově řazeny. K odvození a popisu vlastností modelu se využívá výhod z-transformace \cite{Psutka2006}. Po zjednodušení je krátkodobý model produkce řeči aproximovat celopólovým modelem (filtrem) $H(z)$ ve tvaru

\begin{equation}
  H(z) = \frac{G}{1 + \sum_{i = 1}^{Q} a_{i} z^{-i}} = \frac{S(z)}{U(z)},
  \label{eq:asr:lpc:generic}
\end{equation}

\noindent kde $G$ představuje celkové zesílení, $Q$ je řád modelu odpovídající $2K + 1$ počtu formantů, které má model postihovat, $a_i$ jsou parametry modelu. Vstupem modelu je buzení $u(k)$ (viz obr. \ref{fig:asr:model:speech}), tedy pro znělé zvuky sled pulsů s periodou $T_0$\footnote{Prioda základního hlasivkového tónu.} a pro neznělé zvuky náhodný šum s plochým spektrem. V časové oblasti je pak diskrétní výstupní odezva při fixovaných parametrech hlasového traktu ($10 - 30\ ms$) dána konvolucí buzení a impulzní odezvy krátkodobého modelu. Na základě toho je možné model upravit na tvar podle obr. \ref{fig:asr:model:speech:excitation}, kde je $u(k)$ buzení a $s(k)$ je výstupní signál s parametry hlasového ústrojí odpovídající $a_i$ celopólového modelu.

\begin{figure}[hbpt]
  \centering
  \includegraphics[width=0.9\textwidth]{./ch4-asr/img/speech_process.pdf}
  \caption{Blokové schéma upraveného modelu produkce řeči}
  \label{fig:asr:model:speech:excitation}
\end{figure}

K odhadu parametrů $a_i$ slouží \textbf{lineární prediktivní analýza}. Odhad probíhá přímo z krátkodobého řečového signálu. Přenosové vlastnosti krátkodobého modelu je možné popsat rovnicí (\ref{eq:asr:lpc:generic}). Myšlenka metody LPC staví na předpokladu, že vzorek $k$ řečového signálu je možné popsat lineární kombinací $Q$ předchozích vzorků a buzení $u(k)$, což lze zapsat úpravou vztahu (\ref{eq:asr:lpc:generic}) do tvaru

\begin{equation}
  s(k) = - \sum_{i = 1}^{Q} a_i s(k-1) + Gu(k).
  \label{eq:asr:lpc:generic:edited}
\end{equation}

\noindent Z něj je patrné, že se LPC snaží parametry modelu $a_i$ a zesílení $G$ odhadnout pomocí známe reálně naměřené posloupnosti $s(k)$. K vyřešení se používá principu minimalizace kvadratické chyby krátkodobé energie signálu. Ta je v časové oblasti popsána vztahem

\begin{equation}
  E = \sum_{k} e^2(k) = \sum_{k} \left[ s(k) - s'(k)\right]^2 = \sum_{k} \left( s(k) + \sum_{i = 1}^{Q} a_i s(k-1) + Gu(k) \right),
\end{equation}

\noindent kde $s(k)$ jsou vzorky reálného řečového signálu a $s'(k)$ jsou ty predikováné LPC filtrem. K získání řešení krátkodobé chyby predikce $E$, pro konkrétní analyzovaný segment, je použita metoda nejmenších čtverců. K výpočtu konkrétních koeficientů modelu $a_i$ je možné použít rekurzivního Durbinova algoritmu. \cite{Holmes2001}

Další možností jak popsat hlasový trakt je pomocí \textbf{kepstrálních koeficientů lineární predikce}. Kepstrum je definováno jako inverzní diskrétní Fourierova transformace (IDFT) logaritmu velikosti transformovaného vstupního signálu pomocí diskretní Fourierovy transformace (DFT), matematicky popsáno vztahem

\begin{equation}
  c(k) = \mathcal{F}^{-1}\left\{\log\left| \mathcal{F}\left\{s(k)\right\} \right|\right\},
  \label{eq:asr:lpc:cepstrum:generic}
\end{equation}

\noindent a graficky znázorněno na obr. \ref{fig:asr:model:speech:cepstrum}.

\begin{figure}[hbpt]
  \centering
  \includegraphics[width=0.9\textwidth]{./ch4-asr/img/cepstrum.pdf}
  \caption{Blokové schéma principu výpočtu kepstra}
  \label{fig:asr:model:speech:cepstrum}
\end{figure}

Pro získání kepstrálních koeficientů linearní predikce logaritmujeme rovnici (\ref{eq:asr:lpc:generic}) čímž vznikne vztah

\begin{equation}
  \log H(z) = \log \left( \frac{G}{A(z)} \right).
  \label{eq:asr:lpc:cepstrum}
\end{equation}

\noindent Člen $A(z)$ je polynomem proměnné $z^{-1}$ řádu $Q$, a pokud všechny jeho kořeny leží uvnitř jednotkové kružnice, tak lze aplikovat Taylorův rozvoj na vztah (\ref{eq:asr:lpc:cepstrum}) ve tvaru

\begin{equation}
  \log \left( \frac{G}{A(z)} \right) = c(0) + c(1)z^{-1} + \dots = \sum_{k=0}^{\infty} c(k)z^{-k},
  \label{eq:asr:lpc:cepstrum:taylor}
\end{equation}

\noindent kde $c(k)$ jsou tzv. kepstrální koeficienty LPC. K odstranění logaritmu je potřeba obě strany rovnice derivovat. Po úpravě je výsledný vztah

\begin{equation}
  - \sum_{i=1}^{Q} ia_iz^{-i} = \left( \sum_{k=0}^{\infty} kc(k)z^{-k} \right)\left( \sum_{i=0}^{Q} a_iz^{-i}\right).
  \label{eq:asr:lpc:cepstrum:deriv}
\end{equation}

\noindent Pokud se $a_i = 1$, pak je možné roznásobení rovnice (\ref{eq:asr:lpc:cepstrum:deriv}) a porovnáním členů u stejných mocnin $z$ zapsat vztahy pro výpočet kepstrálních koeficientů LPC

\begin{align}
  \begin{split}
    c(1) &= -a_1, \\
    c(k) &=
    \begin{cases}
      - a_k - \sum_{i=1}^{k-1} \left(\frac{i}{k}\right) c(i) a_{k-1},  & \quad \text{pro } 2 \leq k \leq Q, \\
      - \sum_{i=1}^{Q} \left(\frac{k - i}{k}\right) c(k-i) a_i,  & \quad \text{pro } k = Q + 1, Q + 2, \dots \quad ,
    \end{cases}
  \end{split}
  \label{eq:asr:lpc:cepstrum:coef}
\end{align}

\noindent kde $k = 1, 2, \dots , Q^{*}$ a $Q^{*}$ je počet kepstrálních koeficientů a $Q^{*} \geq Q$.

Kepstrální koeficienty LPC jsou vztaženy ke spektrální obálce mikrosegmentu řeči odvozené LPC analýzou, tu je možné získat dosazením $e^{j\omega}$ za $z$ v rovnici (\ref{eq:asr:lpc:generic}). Pro uspokojivou reprezentaci se tradičně volí $Q = 7\ \text{až}\ 15$ v závislosti na spektrální šířce přenášeného pásma, požadované přesnosti aproximace apod. Z toho plyne, že pro popis mikrosegmentu řeči by mohl stačit příznakový vektor o $15$ koeficientech.

\subsection{Modelování procesu slyšení}
\label{chap:asr:parametrization:hearing}

Zvuk představuje mechanické vlnění hmotných částic, které se šíří v plyném, kapaném nebo tuhém prostředí. Z fyziologického pohledu je však zvukem považováno pouze slyšitelné vlnění. To je takové, které je schopno vnímat sluchové ústrojí člověka. Zpravidla se jedná o frekvence $16\ Hz - 20\ kHz$. Pro každého člověka je, ale toto rozmezí individuální a mění se s věkem. S přibývajícím věkem a sluchovou zátěží klesá hlavně horní mezní kmitočet. \cite{Psutka2006}

To, zda je člověk schopen daný zvuk slyšet však není závislé pouze na frekvenci zvuku. Velmi podstatná je i intenzita zvuku, která se rovná energii zvukového vlnění, která projde za jednotku času jednotkovou plochou kolmou ke směru šíření vln. Zároveň je úměrná akustickému tlaku zvukové vlny, tj. tlaku, kterým zvukové vlny působí na nějakou překážku. V případě člověka tedy ušní bubínek. Závislost mezi intenzitou zvuku $I\ \left[Wm^{-2}\right]$ a akustickým tlakem $p\ \left[Pa\right]$ je vyjádřen vztahem

\begin{equation}
  I = \frac{p^{2}}{z},
  \label{eq:asr:mfcc:intesity}
\end{equation}

\noindent kde $z$ je měrná akustická impedance prostředí, kterým se zvuk šíří. Lidské ucho je schopno vnímat akustický tlak v rozsahu od $2\cdot10^{-5}$ až $2\cdot10^{2}\ Pa$, tj. v rozsahu sedmi řádů. Z praktického důvodu se tedy používá logaritmické stupnice. K vyjadřování pak slouží logaritmus poměru uvažované veličiny a mezinárodně normované referenční hodnoty téže veličiny. \cite{Psutka2006} Hladina intenzity je pak definována vztahem

\begin{equation}
  L_{I} = 10\log_{10}\frac{I}{I_{0}},
  \label{eq:asr:mfcc:intesity:level}
\end{equation}

\noindent kde $I$ představuje intenzitu zvuku a $I_{0} = 10^{-12}\ Wm^{-2}$ referenční hodnotu intenzity. Pro hladinu akustického tlaku platí

\begin{equation}
  L_{p} = 20\log_{10}\frac{p}{p_{0}},
  \label{eq:asr:mfcc:pressure:level}
\end{equation}

\noindent kde $p$ je akustický tlak a $p_{0} = 2\cdot10^{-5}\ Pa$ je refereční hodnota akustického tlaku. Jak $L_{I}$, tak $L_{p}$ je udáváno v decibelech $\left[dB\right]$.

Důležitým pojmem je pak \textbf{práh slyšitelnosti}, který představuje minimální intenzitu zvuku potřebnou k tomu, aby jej šlověk mohl slyšet, viz obr. \ref{fig:asr:mfcc:acoustic:characteristic}. Tento práh je zcela subjektivní a je závislý na frekvenci. Obecně je lidský sluch nejcitlivější na frekcence $3 - 4\ kHz$. Směrem k nižším a vyšším kmitočtům citlivost sluchi klesá. \textbf{Práh bolesti} představuje horní mez intenzity sluchového pole (viz obr. \ref{fig:asr:mfcc:acoustic:characteristic}), při níž již posluchač pociťuje bolest. Překročení této meze může vést k poškození sluchu. \cite{Holmes2001}

\begin{figure}[hbpt]
  \centering
  \includegraphics[width=0.75\textwidth]{./ch4-asr/img/listening_perception.pdf}
  \caption{Oblasti vnímání akustického signálu lidským sluchem.}
  \label{fig:asr:mfcc:acoustic:characteristic}
\end{figure}

Hlasitost zvuku je závislost intenzity na frekvenci a je zcela subjektivní pocit, kterým člověk posuzuje intenzitu daného zvuku. Na obr. \ref{fig:asr:mfcc:acoustic:levels} jsou vyznačeny hladiny hlasitosti, které vznikly spojením bodů ve sluchovém poli (obr. \ref{fig:asr:mfcc:acoustic:characteristic}) odpovídající tónům, které člověk vnímá stejně hlasitě. Z křivek je patrné, že subjektivní hlasitost se mění s frekvencí zvuku. Zvuky s nižší frekvencí vnímáme méně hlasitěji než zvuky s vyšší frekvencí, zejména pak zvuky v rozmezí $3 - 4\ kHz$. \cite{Psutka2006}

\begin{figure}[hbpt]
  \centering
  \includegraphics[width=0.75\textwidth]{./ch4-asr/img/listening_levels.pdf}
  \caption{Oblasti vnímání akustického signálu lidským sluchem.}
  \label{fig:asr:mfcc:acoustic:levels}
\end{figure}

Principem medelování procesu slyšení je pak právě kompenzace nelineárního vnímání frekvencí lidským sluchem (viz obr. \ref{fig:asr:mfcc:acoustic:levels}). Dále pak i respektování maskování zvuků včetně tzv. kritických pásem slyšení, což je přirozená vlastnost lidského sluchu. Maskováním se rozumí jev, kdy vnímání jednoho zvuku je ovlivněno přítomností jiného zvuku. Jinými slovy lze říci, že přítomnost jednoho zvuku zvyšuje práh slyšitelnosti pro jiný zvuk. Ten buď zní současně nebo s drobným časovým odstupem od toho prvního. Tento jev je jakýsi \uv{psychologický filtr}, který ignoruje věškerý šum ležící mimo určité kriticé pásmo. Šířka takového kritického pásma je přitom závislá na frekvenci poslouchaného tónu.

Typickým příkladem metod modelující proces slyšení jsou \textbf{melovská kepstrální filtrace} a \textbf{perceptivní lineární prediktivní analýza}.

\subsubsection{Melovské kepstrální koeficienty}

Metoda melovských frekvenčních kepstrálních koeficientů (MFCC) se snaží respektovat výše zmíněné vlastnosti lidského sluchu. Zejména se snaží dodržet kritická pásma slyšení a vliv subjektivního vnímání výšky tónů.

Základem MFCC je využití banky filtrů a lineárním rozložením frekvencí v tzv. \textbf{melovské frekvenční škále} definované vztahem

\begin{equation}
  f_m = 2595 \log \left(1 + \frac{f}{700}\right),
  \label{eq:asr:mfcc:melscale}
\end{equation}

\noindent kde $f \left[Hz\right]$ je frekvence v lineární škále a $f_m \left[mel\right]$ je odpovídající frekvence v melovské stupnice. Melovský filtr má trojúhelníkový tvar. Banka obsahuje filtry rozmístěné lineárně v melovských frekvenčních souřanicích, a to tak, že dva sousední filtry se navzájem o polovinu překrývají. Pro střední frekvence jednotlivých filtrů $b_{m,i}$ platí v melovské škále vztah

\begin{equation}
  b_{m,i} = b_{m,i-1} + \Delta_{m},
  \label{eq:asr:mfcc:freq}
\end{equation}

\noindent kde $b_{m, 0} = 0\ mel$, $i = 1, 2,\ \dots\ , M^{*}$, a $\Delta_m = B_{m,w} / (M^{*} + 1)$. Ukázka banky filtrů je na obr. \ref{fig:asr:mfcc:bank:mel}. Pro výpočet odezvy filtrů je však nezbytné přepočítat všechny koeficienty FFT do melovské frekvenční škály. Vhodnější je vyjádření trjúhelníkových filtrů ve frekvenční škále s měřítkem v herzích.

\begin{figure}[hbpt]
  \centering
  \includegraphics[width=0.9\textwidth]{./ch4-asr/img/filter_bank-mel.pdf}
  \caption{Rozložení banky trojúhelníkových filtrů v melovské frekvenční škále}
  \label{fig:asr:mfcc:bank:mel}
\end{figure}

\noindent K přepočtu středních frekvencí $b_{m,i}$ se využívá inverzního vztahu k (\ref{eq:asr:mfcc:melscale}) tedy

\begin{equation}
  f = 700 \left[ \exp\left( 0,887.10^{-3} f_m \right) - 1 \right].
  \label{eq:asr:mfcc:melscale:inverse}
\end{equation}

\noindent Střední frekvence $b_i,\ i=1,2,\ \dots\ , M^{*}+1$ jsou vyjádřené v jednotce $[Hz]$. Filtry jsou rozmístěny nelineárně viz obr. \ref{fig:asr:mfcc:bank:hz}.

\begin{figure}[hbpt]
  \centering
  \includegraphics[width=0.9\textwidth]{./ch4-asr/img/filter_bank-hz.pdf}
  \caption{Rozložení banky trojúhelníkových filtrů ve frekvenční škále}
  \label{fig:asr:mfcc:bank:hz}
\end{figure}

Při výpočtu melovských kepstrálních koeficientů jsou na vstup systému přivedeny mikrosegmenty ($10$ až $30\ ms$) řečového signálu $s(k)$. Na vzorcích $s(k)$ byla ještě předtím provedena preemfáze\footnote{Preemfáze znamená zdůraznění amplitud spektrálních složek řečového signálu s jejich vzrůstající frekvencí. \cite{Psutka2006}}. Pro jednotlivé mikrosegmenty je pomocí FFT vypočteno amplitudové spektrum $\left| S(f) \right|$ a následuje klíčová část celého procesu, melovský filtrace. Odezvy filtrů ve frekvenční oblasti lze vyjádřit vztahem

\begin{equation}
  y_m(i) = \sum_{f=b_{i-1}}^{b_{i+1}} \left| S(f) \right| u\left(f, i\right),  \quad i = 1, 2,\ \dots\ ,M^{*},
  \label{eq:asr:mfcc:freq:responce}
\end{equation}

\noindent kde frekvence $f$ jsou vybírány ze souboru frekvencí využívaných při FFT výpočtu a $u(f, i)$ je vyjádření konkrétního troúhelníkového filtru $i$. Průchod filtrem tedy znamená, že každý koeficient FFT je násoben odpovídajícím ziskem filtru a výsledky jsou pro příslušné filtry akumulovány. Logaritomováním akumulovaných koeficientů $y_{m}(i)$ se provede převod do kepstrální oblasti. Tento krok příznivě omezí dynamiku signálu \cite{Benesty2007}.

Posledním krokem při výpočtu melovských kepstrálních koeficientů $\left\{c_m\left(j\right)\right\}_{j=1}^{M}$ je provedení IDFT (viz (\ref{eq:asr:lpc:cepstrum:generic})). V případě MFCC se, ale používá diskrétní kosinova transformace (DCT), protože spektrum je reálné a symetrické. K výpočtu slouží vztah

\begin{equation}
  c_{m}(j) = \sum_{i=1}^{M^{*}} \log y_m(i) \cos\left( \frac{\pi j}{M^{*}}\left(i - 0,5\right) \right),  \quad \text{pro}\ j = 0, 1,\ \dots\ ,M,
  \label{eq:asr:mfcc:coef}
\end{equation}

\noindent kde $M^{*}$ je počet pásem melovkého pásmového filtru a $M$ je počet melovských kepstrálních koeficientů. Počet těchto koeficientů $M$ se volí podstatně menší, než je počet pásem melovského pásmového filtru $M^{*}$, obvykle se uvažuje prvních $M = 10\ \text{až}\ 13$ koeficientů. Velmi často se také používá $1.$ a $2.$ těchto koeficientů, protože svým způsobem zohledňují dynamickou složku řeči.

\subsubsection{Perceptivní lineární prediktivní analýza}

Stejně jako MFCC, tak také i \textbf{perceptivní lineární prediktivní analýza (PLP)} vychází z lidského vnímání a slyšení zvuků. Snaha je postihnout z psychofyziky slyšení zejména kritická pásmá spektrální citlivosti, vztah mezi intenzitou a vnímáním hlasitosti a také křivky stejné hlasitosti. \cite{Psutka2006} PLP (podobně jako LPC) pak aproximuje získané sluchové spektrum koeficienty autoregresního celopólového modelu.

Prvním krokem PLP analýzy je \textbf{výpočet výkonového spektra řečového signálu}. Pro konkrétní mikrosegment řečového signálu $s(k)$ aplikujeme\footnote{Ještě před výpočtem je, stejně jako u MFCC aplikována preemfáze.} DFT. Krátkodoné spektrum je pak definováno vztahem

\begin{equation}
  P\left(\omega\right) = \left| S\left(\omega\right) \right|^{2} = \left[Re\ S\left(\omega\right)\right]^2 + \left[Im\ S\left( \omega \right) \right]^2.
  \label{eq:asr:plp:spectr}
\end{equation}

\noindent Poté následuje kompenzace nelineárního vnímání změn ve výšce zvuku. Vnímání je logaritmické, proto je nutné provést nelineární transformaci frekvenční osy pomocí vzorce

\begin{equation}
  \Omega\left(\omega\right) = 6 \ln \left( \frac{\omega}{1200\pi} + \sqrt{\left(\frac{\omega}{1200\pi}\right)^2 + 1} \right),
  \label{eq:asr:plp:transform}
\end{equation}

\noindent kde $\omega = 2\pi f\ \left[rad/s\right]$ a $\Omega\left(\omega\right)\ \left[bark\right]$.

Zahrnutí kritických pásem slyšení (tzv. maskování zvuku) je realizováno navržením vhodného filtru typu pásmová propust šířky jednoho kritického pásma. Stejně jako v případě MFCC se jedná o banku filtrů, kde na sebe jednotlivé filtry ve frekvenční oblasti navazují. Na Barkově frekvenční ose (viz (\ref{eq:asr:plp:transform})) mají všechny filtry šířku $1$ a jsou lineárně rozmístěny. Na obr. \ref{fig:asr:plp:filter} je zobrazen průběh jednoho takového filtru. Filtr má strmost $+20\ dB/Bark$ směrem k nížsím frekvencím a $-50\ dB/Bark$ sněren k vyšším frekvencím.

\begin{figure}[hbpt]
  \centering
  \includegraphics[width=0.5\textwidth]{./ch4-asr/img/plp_filter.pdf}
  \caption{Ukázka filtru umístného na Barkově frekvenční ose}
  \label{fig:asr:plp:filter}
\end{figure}

Rozmístění filtrů na Barkově frekvenční ose je pak znázorněno na obr. \ref{fig:asr:plp:bank}.

\begin{figure}[hbpt]
  \centering
  \includegraphics[width=0.9\textwidth]{./ch4-asr/img/plp-bank.pdf}
  \caption{Rozmístění filtrů na Barkově frekvenční ose}
  \label{fig:asr:plp:bank}
\end{figure}

Jelikož člověk vnímá intenzitu zvuku v závislosti na frekvenci, tak je potřeba aplikovat \textbf{přizpůsobení křivkám stejné hlasitosti}. Na začátku je důležité definovat referenční hlasitost, tj. hlasitost, na kterou budeme normalizovat. Obvykle se volí $40\ Ph$ \cite{Psutka2006}, což přibližně odpovídá hlasitosti běžné řeči. K normalizaci je použit inverzní filter popsaný vztahem

\begin{equation}
  E\left(\omega\right) = K \frac{\omega^4\left(\omega^2 + 56,9 \cdot 10^6\right)}{\left(\omega^2 + 6,3 \cdot 10^6\right)^2\left(\omega^2 + 379,4 \cdot 10^6\right)\left(\omega^6 + 9,6 \cdot 10^{26}\right)},
  \label{eq:asr:plp:filter}
\end{equation}

\noindent kde $\omega = 2\pi f$ a $K$ je konstanta nastavená podle požadovaného zesílení. Přizpůsobení křivce stejné hlasitosti je pak možné například přenásobením celého výkonového spektra mikrosegmentů podle vztahu

\begin{equation}
  P'\left(\omega\right) = E\left(\omega\right)P\left(\omega\right),
  \label{eq:asr:plp:filter:application1}
\end{equation}

\noindent kde $P'\left(\omega\right)$ je spektrum transformované na stejnou hlasitost. Případně lze upravit tvar jednotlivých filtrů pomocí vztahu

\begin{equation}
  \Phi\left(\omega, i\right) = E\left(\omega\right)\Psi\left(\omega - \omega_i, i\right),
  \label{eq:asr:plp:filter:application2}
\end{equation}

\noindent kde $\Phi\left(\omega, i\right)$ je nový tvar filtru $i$ v závislosti na frekvenci $\omega$, $\Psi\left(\omega - \omega_i, i\right)$ je odezva filtru $i$ se středovou frekvencí $\omega_i$.

Po přizpůsobení následuje \textbf{výpočet energie jednotlivých filtrů}, to je obdobné jako u MFCC. Výpočet se provádí pro jednotlivé filtry a výsledky se pak sčítají. Matematicky to je zapsáno vztahem

\begin{equation}
  \zeta_m = \sum_{\Omega = \Omega_m - 2,5}^{\Omega_m + 1,3} P\left(\Omega\right)\Phi\left(\Omega, m\right), \quad\ m=1, 2,\ \dots\ M - 2,
  \label{eq:asr:plp:energy}
\end{equation}

\noindent kde $M$ je počet použitých filtrů (kritických pásem).

Dalším krokem výpočtu je uplatnění \textbf{\uv{zákona slyšení}}. Ten popisuje závislost mezi intenzitou a vnímanou hlasitostí. Na energie $\zeta_m$ je aplikována nelineární transformace vyjádřena vztahem

\begin{equation}
  \xi_m = \left(\zeta_m\right)^{0,3}, \quad\ m = 1, 2,\ \dots\ M-2,
  \label{eq:asr:plp:energy:transform}
\end{equation}

\noindent kde $M$ je opět počet filtrů. Díky této operaci dojde také k redukci proměnlivosti \uv{výstupů} kritických pásemových filtrů a výsledný hledaný celopólový model může být relativně nízkého řádu.

Finálním krokem je \textbf{aproximace celopólového modelu}. Ta vychází z výpočtu koeficientů celopólového modelu metody LPC, kde je model popsán vztahem (\ref{eq:asr:lpc:generic:edited}). Pro chybu predikce pak platí

\begin{equation}
  e\left(k\right) = \sum_{k} \left(s\left(k\right) + \sum_{i=1}^{Q} a_i s\left(k - i\right)\right).
  \label{eq:asr:plp:error}
\end{equation}

\noindent Aplikací z-transformace a uvážením rovnice (\ref{eq:asr:lpc:generic}), je možné (\ref{eq:asr:plp:error}) upravit do tvaru

\begin{equation}
  E\left(z\right) = \left[1 + \sum_{i=1}^{Q} a_i z^{-i}\right] S\left(z\right) = A\left(z\right)S\left(z\right),
  \label{eq:asr:plp:error:transform}
\end{equation}

\noindent kde $A\left(z\right)$ je inverzní filtr a $E\left(z\right)$ a $S\left(z\right)$ jsou z-transformace $e\left(k\right)$ a $s\left(k\right)$. Celkovou chybu predikce je pak možné vyjádřit vztahem

\begin{equation}
  E\left(z\right) = \frac{1}{2\pi} \int_{-\pi}^{\pi} P\left(\omega\right) A\left(e^{j\omega}\right) A\left(e^{-j\omega}\right)d\omega,
  \label{eq:asr:plp:error:final}
\end{equation}

\noindent kde $P\left(\omega\right)$ je vypočtené výkonové spektrum. Podobně jako u LPC je řešením nalezení minima celkové chyby autokorelační funkce $R\left(i\right)$. Pro konečný počet známých frekvencí je tato funkce definována vztahem

\begin{equation}
  R\left(i\right) = \frac{1}{N} \sum_{n=0}^{N-1} P\left(\omega_n\right) \cos\left(i\omega_n\right),
  \label{eq:asr:plp:error:solution}
\end{equation}

\noindent kde $i = 0,\ \dots\ Q$ a $Q$ je řád autoregresního modelu, a $N$ je počet bodů spektrální charakteristiky. Frekvence $\omega_n$ jsou ty, pro které jsou známé spektrální hodnoty. Pro dobrou aproximaci se volí $Q = 5$.

\textbf{Výpočet kepstrálních koeficientů PLP} lze pak pro známé hodnoty $R\left(i\right)$, podobně jako u LPC, určit  Durbinovým algoritmem. Nalezené koeficienty lze již využít jako příznaky při návrhu parametrizátoru řeči, ale častěji se používají kepstrální koeficienty PLP. \cite{Holmes2001}

K vytvoření parametrizátoru je možné použít libovolnou metodu představenou v \ref{chap:asr:parametrization:production} a \ref{chap:asr:parametrization:hearing}. V současnosti, ale převládají metody postavené na principu fungování lidského sluchu, protože amplifikují podstatnou informaci zakódovanou v řeči.
