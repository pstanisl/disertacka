% !TEX root = ../thesis.tex
\chapter{Automatické rozpoznávání řeči}
\label{chap:asr}

TBD

% Jedním z hlavních důsledků TL (popsané v \todo{TBD}{[xx]}) je ztráta hlasivek a tím i hlasu. Problematikou komunikace pomocí mluvené řeči i v situacích, kdy akustický řečový signál není k dispozici, se zabývají  systémy zpracovávající \uv{tichou} řeč (angl. Silent speech interface, zkr. SSI). Ve většině případů se snaží získat informaci, která je normálně zakódována v akustickém signálu získat jinou cestou.

% Produkce mluvené řeči je komplexní proces, který začíná v možku a končí produkcí slyšitelného zvuku. Pokud odstraníme komponentu starající se o vznik zvuku, ještě to neznamená, že i ostatní komponenty také ztrácejí svou funkci. Tento fakt je základní premisou pro funkci všech v současnosti vyvíjených SSI systémů.

% Vývoj komplexního SSI je velmi náročný problém, který se zatím (i přes nemalé usílí) doposud nepodařilo uspokojivě vyřešit.

% \begin{itemize}
%   \item lehce popsat technické přístupy
%   \begin{itemize}
%     \item NAM
%     \item magnety
%     \item brain interface
%   \end{itemize}
% \end{itemize}
