% !TEX root = ../thesis.tex
\section{Analýza dat a první modely}
\label{chap:experiments:analysis}

Rozpoznávání řeči se věnuje nemalé usílí již od 50. let 2O. století a v současné době nikoho nepřekvapí téměř bezchybně fungující obecný rozpoznávač v mobilních zařízeních. Pro obecné systémy dokonce existují korpusy s desítkami či stovkami a více hodin promluv, které je možné využít při vytváření těchto systémů.

Tyto korpusy však obsahují ve většině případů pouze \uv{standardní}\footnote{Slovením spojením \uv{standardní řeč} je myšlena řeč neobsahující vyrazné řečové vady, případně jiné formy produkce a často v nepřílíš akusticky náročném prostředí.} řeč. Pokud je snaha vytvořit nebo ověřit funkčnost systému za specifických podmínek (ať už se jedná o rušné prostředí či speciální typy promluv), tak je nezbytné získat potřebná data.

\subsection{Vytvoření korpusu EL promluv}

Na začátku byla idea o pomoci skupině lidí mající problémy s přirozenou řečí. Vůbec prvním předpokladem, na cestě k úspěšnému dosažení vůbec nějakého cíle, jsou data. Jelikož se jedná o velmi spefická data, tak je potřeba zajistit co možná největší množství kvalitních\footnote{Kvalitou je myšlena věrnost dat dané doméně, dále se mluví o přesnosti ve smyslu bezchybnosti přepisů.} a přesných dat.

V části \ref{sec:cause:desease} bylo zmíněno, že ročně se objevý více než 100 nových případů trvalé ztrázy hlasu ročně. Zaroveň bylo řečeno \cite{Skvrnakova2010}, že více rizikovými osobami jsou starší lidé, kteří intenzivně kouří a konzumují alkohol. Přesto je patrný trend snižujícího se věku pacientů a s tím souvisejícím nárůstem případů ztráty hlasu. Přičteme-li již zmíněný psychologický aspekt jeho ztráty, tak je zřejmé, jak komplikované je získat ke spolupráci i jen jednoho řečníka ochotného podstoupit naročné\footnote{I pro zdravého člověka je někdy někalikahodinové nahrávání vysilující. Pro jedince po TL to je z mnoha důvodů ještě řádově náročnější.} nahrávání.

Při libovolné práci s pacienty po TL, dřív nebo později dojde k určité formě spolupráce s oddělením ORL, které má nastarosti péči o tyto pacienty. V našem připadě nejprve s ORL klinikou při Fakultní nemocnici v Plzni a poté i s ORL klinikou Fakultní nemocnice v Motole. S jejich pomocí jsme získali ke spolupráci jednoho řečníka. Konkrétně se jedná o dámu v duchodovém věku, která podstoupila TL před více než 15 lety. Po překonání ostychu\footnote{Podle jejích vlastních slov nebyla schopna několik let po operaci ani zvednout nečekaný telefonní hovor, natož mluvit na veřejnosti.} se byla schopna naplno vrátit do běžného života a dokonce v určité formě opět přednášet o stomatologii na Lekařské fakultě v Plzni Univerzity Karlovy.

S její pomocí jsem, v 1. etapě nahrávání, byli schopni pořídit přes 10 hodin promluv, viz tab. \ref{tab:experiments:analysis:recording}. Nahrávání probíhala v relativně spartánských podmínkách za plného běžného provozu katedry. Přesto získaná data neosahují žádný nežadoucí ruch, kromě toho produkovaného samotným EL.

Nahravací aparatůra sestávala z miniaturního profesionálního mikrofonu (DPA d:screet 4061-FM), zesilovačem (DPA MMA6000), externí zvukovou kartou a běžného notebooku. Mikrofon byl pomocí bezpolštářkové náplasti přilepen poblíž pravého koutku úst, abychom zaznamenaná řeč měla co možná nejvyšší kvalitu.

Celé nahrávání bylo v 1. etapě rozděleno do 14 samostatných sezení a probíhalo od prosince roku 2010 do května roku 2011. Každé sezení trvalo přibližně dvě hodiny během kterých se podařilo získat necelou hodinu akustických dat. Samotné nahrávání se sestávalo z 10 - 20 minutového úseku pořizování nahrávky a přibližně 10 minut dlouhého odpočinku. Ten byl nezbytný hlavně z důvodu únavy řečníka.

Ještě před samotným nahrávánám byly pečlivě vybrány a vytvořeny 2 sady vět:

\begin{enumerate}
  \item sada obsahující všechny možné české fonémy - \textit{40 vět}.
  \item sada obsahující věty s reálnou četností fonémů - \textit{5000 vět}.
\end{enumerate}

\noindent Pořízené nahrávky vždy odpovídají 10 - 20 minutovému úseku nepřerušovaného nahrávání a soubory tak vždy obsahují několik vět. Ty jsou od sebe odděleny minimálně 5 sekundovým úsekem ticha. Nahrávky dále mouhou obsahovat opakování chybně vyslovené věty, přeřeknutí, kýchnutí a další neřečové události. Z tohoto důvodu bylo nezbytné pořízené nahrávky anotovat, přestože byly pořízené na základě připravené sady vět.

Ještě před samotným anotováním byly nahrávky, podle úseků s tichem, rozsekány na menší části. V tomto případě úplně dobře nefungovali\footnote{Problémem byl zvuk EL, se kterým nebylo při návrhu VAD počítáno.} standardně používané sofistikovanější metody pro voice activity detection (angl. zkratka VAD), a proto bylo využito principu energie. Pro každou nahrávku obsahující více vět se pomocí vzorce

\begin{equation}
  \label{eq:experiments:analysis:energy}
  E_{RMS}(n) = \sqrt{\frac{1}{N} \sum_{n=1}^{N} \left| x(n) \right|^2},
\end{equation}

\noindent kde $N$ představuje počet vzorků v nahrávce a $x(n)$ představuje pravoúhlé okénko vzorku $n$. Pro tento případ se ukázalo jako vhodnější volit root-mean-square energy ($E_{RMS}$) a empericky se ukázalo, že vhodná délka okénka je v rozmezí $10 - 100$ ms. Na obr. \ref{fig:experiments:analysis:el_speech} je zobrazena podoba audio signálu a spektrogram promluvy \textit{\uv{Akcie Komerční banky}}. Zároveň je zde vypočtené hodnoty energie a celková průměrná energie. Tyto hodntoty slouží pro určení míst kde začíná a končí věta. Na začátku a konci každé věty je dobré mít minimálně $0.5$ s ticha. Tím pádem, pokud energie nějakého úseku $x$ je $E_{RMS}(x) < avg(E_{RMS})$ a zároveň délka tohoto úseku $dur(x) \geq 1\ [s]$, tak nahrávku můžeme v tomto úseku rozdělit.

\begin{figure}[hbpt]
  \centering
  \includegraphics[width=0.9\textwidth]{./ch4-experiments/img/energy_spec_el.png}
  \caption{Průběh a spektrogram promluvy a vyznačenou energií EL promluvy.}
  \label{fig:experiments:analysis:el_speech}
\end{figure}

Samozřejmě pokud řečník v průbehu věty z libovolného důvodu udělal větší pauzu než $1$ s, tak tato věta byla rozdělena. Jelikož jsou výsledné kratší useky promluv následně anotovány, tak to nepředstavuje problém. Pro budoucí zpracování není podstatné zda promluva je opravdu celá věta, ale to jestli je tento úsek správně přepsán. Fakt, že některé věty jsou rozděleny je důvodem proč v tab. \ref{tab:experiments:analysis:recording} více souborů než vět.

K anotaci posloužil interní nástroj určený k tomuto účelu a podíleli se na něm celkem 3 anotátoři z řad studentů, kteří si vzájemně kontrolovali své anotace. Ačkoli bylo potřeba anotovat relativně malé množství dat (cca 10 hodin audio záznamu), tak anotace zabrala přibližně 2 měsice. Hlavním důvodem byla relativně dlouhá doba, po kterou se anotátoři adaptovali na specificka EL řeči. Problémem bylo to, že nejprve nebyli vůbec schopni poruzumnět obsahu promluvy a tím pádem jej správně přepsat.

Pokud je pro produkci řeči použit elektrolarynx, tak vedlejším produktem je nezanedbatelný ruch způsobený samotným zařízením \ref{sub:cause:treatment:foniatric}. Přeci jen jeho jedinout funkcí je vybudit vzduch v dutině ústní a tím umožnit produkci slyšitelné řeči. Z tohoto důvodu byly v průběhu anotace ignorovány v podstatě všechny skupiny neřečových událostí, protože vetšinu obsahu nahrávek by bylo nezbytné anotovat jako, že obsahují šum.

Výsledný korpus tedy představuje $5040$ unikátních vět rozdělených do $6385$ souborů (viz tab. \ref{tab:experiments:analysis:recording}), které v průměru obsahují $7$ slov o průmerné délce $5$ znaků. Tento korpus slouží jako základ pro všechny budoucí experimenty.

\begin{table}[htpb]
  \centering
  \def\arraystretch{1.5}
  \pgfplotstabletypeset[
    col sep=comma,
    string type,
    columns/phase/.style={column name=Nahrávání, column type={|l}},
    columns/length/.style={column name=Délka \textit{[HH:MM:SS]}, column type={|r}},
    columns/sentences/.style={column name=Počet vět, column type={|r}},
    columns/files/.style={column name=Počet souborů, column type={|r|}},
    every head row/.style={after row=\hline, before row=\hline},
    every last row/.style={after row=\hline},
  ]{./ch4-experiments/tabs/02-recording1-stats.csv}
  \caption{Infoemace o korpusu nahrávek z 1. etapy nahravání.}
  \label{tab:experiments:analysis:recording}
\end{table}

\subsection{Analýza získaných dat}

Po dokončení anotace obsahuje korpus přes 10 hodin akustických záznamů promluv a více či méně přesných přepisů\footnote{I přes nemalou snahu a několikastupňovou kontrolu, je téměř jisté, že by nebylo obtížné najít přepis, který obsahuje chybu například ve formě překlepu.}. Když jsou k dispozici data je možné se podívat na specifika EL řeči a případně porovnat se zdravým řečníkem.

Pro potřeby porovnání byl použit začátek promluvy \textit{\uv{Akcie Komerční banky...}}. Tuta promluva je součástí standardní množiny vět používaných při vytváření řečových korpusů na KKY při ZČU. Tím pádem je k dispozici v relativně velkém množství příkladů pro zdravé řečníky a také je součástí korpusu EL řeči.

Na obr. \ref{fig:experiments:analysis:el_speech} a \ref{fig:experiments:analysis:normal_speech} je zobrazen průběh signálu a spektrogram vybrané promluvy. Už na první pohled je možné zaznamenat určité rozdíly. Prvním takovým je délka promluvy, v případě zdravého řečníka je o celou 1 vteřinu kratší než v případě EL řeči. Tempo řeči je samozřejmě velmi individuální, ale z principu je EL řeč pomalejší. Z průbehu signálu na obr. \ref{fig:experiments:analysis:el_speech} je patrné, že řečník dělá výraznější pauzy mezi jednotlivými slovy promluvy. To může být způsobené například potřebou naplnit jícen vzduchem. Po TL je dýchání realizováno přes tracheu a pokud nebyl voperován shunt (více v \ref{sub:cause:treatment:tracheo}), tak je trvale oddělen hrtan a hltan. Přesto, pro produkci některých neznělých fonémů je potřeba exhalovat vzduch z dutiny ústní. Zkušený EL řečník to dělá naprosto automaticky, nicméně \uv{polykání} vzduchu zabere nějaký čas. Nevyhnutelným důsledkem je pak velmi častý výskyt samovolného říhání v průběhu promluvy\footnote{Fakt, že je říhání jako neřečová událost běžnou součástí téměř každé promluvy, vedl k ignorování těchto událostí během anotace.}.

\begin{figure}[hbpt]
  \centering
  \includegraphics[width=0.9\textwidth]{./ch4-experiments/img/energy_spec_normal.png}
  \caption{Průběh a spektrogram promluvy a vyznačenou energií promluvy.}
  \label{fig:experiments:analysis:normal_speech}
\end{figure}

Dalším důvodem může být je nutnost správné artikulace. Při používání EL je to nezbytné, aby bylo produkované řeči alespoň trochu dobře rozumnět. A pokud se dobře artikuluje, tak není snadné mluvit rychle. Při nahrávání bylo také velmmi běžné, že v průběhu promluvy řečník udělal pauzu, aby mohl lépe umístit EL, protože jeho umístění má velký vliv na kvalitu produkované řeči. Nicméně je třeba říci, že tempo není a priory pro ASR systémy problém, protože různá délka fonémů je v relativně snadno modelována, např. v HMM přechodem ze stavu do stejného stavu.

Dalším způsobem jak ukázat rozdíly mezi promluvou zdravého řečníka a řečníka s EL je srovnání ve frekvenční oblasti. Pro větší názornost jsou na obr. \ref{fig:experiments:analysis:spectrogram} vedle sebe zobrazeno spektrum ukázková promluva zdravého řečníka (\ref{fig:experiments:analysis:spectrogram:normal}) a toho s EL (\ref{fig:experiments:analysis:spectrogram:el}). Obsah obou promluv je identický a přesto jsou obě spektra odlišná.

Prvním markantním rozdílem je mnohem větší zastoupení šumu v úsecích \uv{ticha} na obr. \ref{fig:experiments:analysis:spectrogram:el}. To je nepochyně způsobnemo samotným EL, který řečník nevypíná mezi jednotlivými slovy. Na obr. \ref{fig:experiments:analysis:el_speech} je to také zřetelně patrný, zejména na průběhu energie, šum zejména před prvním a druhým slovem promluvy. Zajímavá je přítomnost šumu v célém frekvenčním spektru, přestože EL produkuje konstantní buzení. Toto buzení je ve spektru, na obr. \ref{fig:experiments:analysis:spectrogram:el}, viditelná jako výrazná souvislá linie v nízkých frekvencích. Přitomnost šumu ve vyšších frekvencích je způsobena umístěním mikrofonu, který je nalepen na pokožku a tím pádem snímá namodulované vibrace, přenášené měkkou tkání. Tento fakt se potvrdil v dalších etapách nahrávání (viz \todo{přidat zmínku o tom, že v dalších etapách to je krapet jinak}{porovnání}), kde byl použit studiový mikrofon vzdálený od úst minimálně 15 cm. Nicméně z pohledu použitelnosti nějakého budoucího systému je nezbytné počítat i se situací, kdy mikrofon bude zaznamenávat i vibrace přenášené tkání.

Dalším markantním rozdílem je absence vyšších frekvencí u většiny produkovaných fonémů. Vyjímku tvoří afrikáty $/c/$ a $/\check{c}/$, u kterých jsou hlasivky (u zdravého jedince) v klidu a vznikají uvolněním nahromaděného vzduchu v dutině ústní\footnote{Nahromadění vzduchu je realizováno přitisknutím jazyka k přední/zadní části horního patra.} \cite{Psutka2006}. V tomto případě není, u řečníka po TL, principiálně tento mechanizmus produkce těchto fonému ovlivněn. Problémem teoreticky může být zdroj vzduchu, jelikož jej z plic není možné dostat do dutiny ústní, ale jak už bylo zmíněno (a spektrogram to potvrzuje) zkušený uživatel EL se dokáže adaptovat.

Absence vyšších frekcencí se dá vysvětlit použitím EL, kde samotný EL má vždy konstantní frekvenci buzení a dále tím, že nedochází k modulaci v ve všech dutinách vokálního traktu. Nicméně nejdůležitější složky, zajišťující srozumitelnost, se vyskytují ve frekvenčním pásmu od 1 kHz do 3 kHz. Vyšší frekvence se a priory podílejí na zabarvení hlasu.

\begin{figure}[htpb]
  \centering
  \begin{subfigure}[b]{0.4\textwidth}
    \includegraphics[width=\textwidth]{./ch4-experiments/img/spectrogram_normal.png}
    \caption{Zdravý řečník}
    \label{fig:experiments:analysis:spectrogram:normal}
  \end{subfigure}
  %
  \begin{subfigure}[b]{0.4\textwidth}
    \includegraphics[width=\textwidth]{./ch4-experiments/img/spectrogram_el.png}
    \caption{EL řečník}
    \label{fig:experiments:analysis:spectrogram:el}
  \end{subfigure}
  \caption{Spektrogram promluvy \uv{Akcie Komerční banky} dvou řečníků.}
  \label{fig:experiments:analysis:spectrogram}
\end{figure}

Dalším způsobem jak porovnat řeč zdravého řečníka a tím s EL je pomocí analýzy jednotlivých fonémů. Na obr. \ref{fig:experiments:analysis:phonemes} jsou zobrazeny průběhy amplitudy v čase\footnote{Hodnoty času, na obr. \ref{fig:experiments:analysis:phonemes}, odpovídají časům výskytu v původní promluvě.} pro fonémy $/k/$, $/m/$ a $/\check{c}/$. V případě $/k/$ a $/m/$ (1. a 2. průběh) se jedná o okluzivy, kde v prvním případě se jedná o neznělou plozivu a druhém o znělou plozivu. Tyto fonémy obecně vznikají uzavřením vydechovaného proudu vzduchu, pomocí artikulačních orgánů, což se projeví jako krátká pauza (tzv. okluze). Po té následuje náhlé jednorázové překážky a únik nahromaděného vzduchu, tzv. exploze \cite{Psutka2006}. Takto popsáno to samozřejmě funguje u zdravého jedince, ale u EL řečníka jde principiálně o stejný mechanizmus. S tím rozdílem, že vzduch nepochází z plic, ale z hltanu. Dalším rozdílem je samozřejmě absence hlasivek.

\begin{figure}[htpb]
  \centering
  \begin{subfigure}[b]{0.42\textwidth}
    \includegraphics[width=\textwidth]{./ch4-experiments/img/phonemes_normal.png}
    \caption{Zdravý řečník}
    \label{fig:experiments:analysis:phonemes:normal}
  \end{subfigure}
  %
  \begin{subfigure}[b]{0.42\textwidth}
    \includegraphics[width=\textwidth]{./ch4-experiments/img/phonemes_el.png}
    \caption{EL řečník}
    \label{fig:experiments:analysis:phonemes:el}
  \end{subfigure}
  \caption{Ukázky průběhů amplitudy pro fonémy $/k/$, $/m/$ a $/\check{c}/$.}
  \label{fig:experiments:analysis:phonemes}
\end{figure}

Foném $/k/$ představuje zástupce neznělých fonémů, ty se vyznačují tím, že do jejich produkce nezasahují hlasivky, které jsou v klidu. Zdrojem buzení je tedy šum. Pokud se podíváme na průběh amplitudy v čase u zdravého řečníka (obr. \ref{fig:experiments:analysis:phonemes:normal}), tak zde není vidět žádný periodický signál. Hlasivky jsou tedy opravdu v klidu. Oproti tomu u EL řečníka (obr. \ref{fig:experiments:analysis:phonemes:el}) je jasně patrné, že je zde přítomno aktivní buzení vytvořené EL. Na obr. \ref{fig:experiments:analysis:freq:k} je pak zobrazeno tzv. amplitudové spektrum, které znázorňuje vývoj amplitudy signálu ve frekcenci pro oba řečníky. V případě zdravého řečníka odpovídá vývoj očekávání tedy, že zde není žádná výrazná frekvence a také, že nedochází k výraznému útlumu. Přestože se v obou případech jedná o stejný foném, tak z časového i frekvešního průběhu amplitudy je zřejmé, že parametry signálu se u obou řečníku diametrálně liší.

\begin{figure}[hbpt]
  \centering
  \includegraphics[width=0.9\textwidth]{./ch4-experiments/img/freq_analysis_(k).png}
  \caption{Vývoj amplitudy fonému $/k/$ ve frekvenci zdravého (horní) a EL (dolní) řečníka.}
  \label{fig:experiments:analysis:freq:k}
\end{figure}

Jako druhý ukázkový foném slouží $/m/$. Opět se jedná o plozivu, ale v tomto případě o znělou. U těchto fonémů hrají velký vliv hlasivky, protože jsou zdrojem buzení. Z obr. \ref{fig:experiments:analysis:phonemes:normal} je krásně zřetelné buzení ve formě perodického průběhu amplitudy. Narozdíl tomu, u EL řečníka (obr. \ref{fig:experiments:analysis:phonemes:el}) je také vidět periodický signál, ale úplně jiného průběhu. Svým způsoběm dost podobný tomu, který je zřetelný u fonému $/k/$. Rozdíl je zřetelný i ve frekvenční oblasti (obr. \ref{fig:experiments:analysis:freq:m}), kdy u EL řečnía nedochází útlumu ve střední oblasti frekvenčního spektra.

\begin{figure}[hbpt]
  \centering
  \includegraphics[width=0.9\textwidth]{./ch4-experiments/img/freq_analysis_(m).png}
  \caption{Vývoj amplitudy fonému $/m/$ ve frekvenci zdravého (horní) a EL (dolní) řečníka.}
  \label{fig:experiments:analysis:freq:m}
\end{figure}

Posledním úkázkovým fonémem je již zmiňované $/\check{c}/$. Jedná se o neznělý foném, který vzniká přiložením jazyku k zadní části horního patra. Tím je zadržen vzduch v dutině ústní a vzniká krátká pauza. Uvolněním pak dochází k explozi a vytvoření zvuku. Do produkce se nezapojují hlasivky a produkovaný zvuk by měl být dostatečně intenzivní, aby jej (v případě EL řečníka) tolik neovlivňoval EL. Tím pádem by měl být průběh signálu, u obou řečníků podobný, a to jak v časové, tak i ve frekvenční oblasti. Na obr. \ref{fig:experiments:analysis:phonemes} a \ref{fig:experiments:analysis:freq:c} je pak jasně vidět, že se jedná o platný předpoklad.

\begin{figure}[hbpt]
  \centering
  \includegraphics[width=0.9\textwidth]{./ch4-experiments/img/freq_analysis_(c).png}
  \caption{Vývoj amplitudy fonému $/\check{c}/$ ve frekvenci zdravého (horní) a EL (dolní) řečníka.}
  \label{fig:experiments:analysis:freq:c}
\end{figure}

% \csvautotabular{./ch4-experiments/test.csv}

\begin{table}[htpb]
  \centering
  \def\arraystretch{1.5}
  \pgfplotstabletypeset[
    col sep=comma,
    string type,
    columns/model/.style={column name=Model, column type={|c}},
    columns/8k/.style={column name=8 kHz $[\%]$, column type={|r}},
    columns/16k/.style={column name=16 kHz $[\%]$, column type={|r|}},
    every head row/.style={before row={
      \hline
      & \multicolumn{2}{c|}{WER} \\
    },after row=\hline},
    every last row/.style={after row=\hline},
  ]{./ch4-experiments/tabs/01-frequency.csv}
  \caption{Vliv frekvence na kvalitu modelu.}
\end{table}

\begin{itemize}
  \item popis jak byla data získána
  \item informace o datech (frakvenční rozsah, atp.)
  \item prvotní experimenty k určení parametrů modelu
  \item popsat výsledky, výsledky převážně na HTK (HMM + GMM)
\end{itemize}
