% !TEX root = ../thesis.tex
\section{Analýza dat a první modely}
\label{chap:experiments:analysis}

Rozpoznávání řeči se věnuje nemalé usílí již od 50. let 2O. století a v současné době nikoho nepřekvapí téměř bezchybně fungující obecný rozpoznávač v mobilních zařízeních. Pro obecné systémy dokonce existují korpusy s desítkami ne-li stovkami a více hodin promluv, které je možné využít při vytváření těchto systémů.

Tyto korpusy však obsahují vě většině případů pouze \uv{standardní}\footnote{Slovením spojením \uv{standardní řeč} je myšlena řeč neobsahující vyrazné řečové vady, případně jiné formy produkce a často v nepřílíš akusticky náročném prostředí.} řeč. Pokud je snaha vytvořit nebo ověřit funkčnost systému za specifických podmínek (ať už se jedná o rušné prostředí či speciální typy promluv), tak je nezbytné získat potřebná data.

\subsection{Vytvoření korpusu EL promluv}

Na začátku všeho je idea, v tomto případě o snaze pomoci skupině lidí mající problémy s přirozenou řečí. Vůbec prvním předpokladem, na cestě k úspěšnému dosažení nějakého cíle, jsou data. Jelikož se jedná o velmi spefická data, tak je potřeba zajistit co možná největší množství kvalitních\footnote{Kvalitou se myslí věrnost dat dané doméně. Mohlo by se také mluvit o přesnosti ve smyslu přepisů.} dat.

V části \ref{sec:cause:desease} bylo zmíněno, že ročně se objevý více než 100 nových případů trvalé ztrázy hlasu ročně. Zaroveň bylo zmíněno, že více rizikovými osobami jsou starší lidé, kteří intenzivně kouří a konzumují alkohol, i když je velmi patrný trend nárůstu mladších pacientů. Přičteme-li již zmíněný psychologický aspekt ztráty hlasu, tak je zřejmé jak komplikované je získat ke spolupráci i jen jednoho řečníka ochotného podstoupit naročné\footnote{I zdravého člověka je někdy někalikahodinové nahrávání vysilující. Pro jedince po TL to je z mnoha důvodů ještě řádově náročnější.} nahrávání.

Při libovolné práci s pacienty po TL, dřív nebo později dojde k určité formě spolupráce s oddělením ORL, které má nastarosti péči o tyto pacienty. V našem připadě nejprve s ORL klinikou při Fakultní nemoci v Plzni a poté i s ORL klinikou Fakultní nemocnice v Motole. S jejich pomocí jsme získali ke spolupráci jednoho řečníka. Konkrétně se jedná o dámu v duchodovém věku, která podstoupila TL před více než 15 lety. Po překonání ostychu\footnote{Podle jejích vlastních slov nebyla schopna několik let po operaci ani zvednout nečekaný telefonní hovor natož mluvit na veřejnosti.} se byla schopna naplno vrátit do běžného života a dokonce v určité formě opět přednášet o stomatologii na Lekařské fakultě v Plzni Univerzity Karlovy.

S její pomocí jsem v první etapě nahrávání byly schopni pořídit přes 10 hodin promluv, viz tabulka \todo{Tabulka}{TBD}. Nahrávání probíhala v relativně spartánských podmínkách v běžných prostorách katedry. K nahravání byl použit profesionální mikrofon (\todo{typ mikrofonu}{TBD}), ale mezi ostatními komponentami byla již běžná externí zvuková karta a soukromý notebook. Celé nahrávání bylo rozděleno do 14 samostatných sezení a trvalo přibližně 6 měsíců. Každé sezení trvalo přibližně dvě hodiny během kterých jsme byli schopni získat necelou hodinu akustických dat. Získaná data bylo pak potřeba anotovat, to zabrolo přibližně další 2 měsíce.


\begin{itemize}
  \item popis jak byla data získána
  \item informace o datech (frakvenční rozsah, atp.)
  \item prvotní experimenty k určení parametrů modelu
  \item popsat výsledky, výsledky převážně na HTK (HMM + GMM)
\end{itemize}

% \csvautotabular{./ch4-experiments/test.csv}

\begin{table}[htpb]
  \centering
  \def\arraystretch{1.5}
  \pgfplotstabletypeset[
    col sep=comma,
    string type,
    columns/model/.style={column name=Model, column type={|c}},
    columns/8k/.style={column name=8 kHz $[\%]$, column type={|r}},
    columns/16k/.style={column name=16 kHz $[\%]$, column type={|r|}},
    every head row/.style={before row={
      \hline
      & \multicolumn{2}{c|}{WER} \\
    },after row=\hline},
    every last row/.style={after row=\hline},
  ]{./ch4-experiments/tabs/01-frequency.csv}
  \caption{Vliv frekvence na kvalitu modelu.}
\end{table}
