% !TEX root = ../thesis.tex
\section{Analýza dat a první modely}
\label{chap:experiments:analysis}

\begin{itemize}
  \item popis jak byla data získána
  \item informace o datech (frakvenční rozsah, atp.)
  \item prvotní experimenty k určení parametrů modelu
  \item popsat výsledky, výsledky převážně na HTK (HMM + GMM)
\end{itemize}

% \csvautotabular{./ch4-experiments/test.csv}

\begin{table}[htpb]
  \centering
  \def\arraystretch{1.5}
  \pgfplotstabletypeset[
    col sep=comma,
    string type,
    columns/model/.style={column name=Model, column type={|c}},
    columns/8k/.style={column name=8 kHz $[\%]$, column type={|r}},
    columns/16k/.style={column name=16 kHz $[\%]$, column type={|r|}},
    every head row/.style={before row={
      \hline
      & \multicolumn{2}{c|}{WER} \\
    },after row=\hline},
    every last row/.style={after row=\hline},
  ]{./ch4-experiments/tabs/01-frequency.csv}
  \caption{Vliv frekvence na kvalitu modelu.}
\end{table}
