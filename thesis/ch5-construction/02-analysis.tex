% !TEX root = ../thesis.tex
\section{Analýza akustického signálu a jeho parametrizace}
\label{chap:construction:analysis}

Rozpoznávání řeči se věnuje nemalé úsilí již od 50. let 20. století a v současné době nikoho nepřekvapí téměř bezchybně fungující obecný rozpoznávač souvislé řeči v mobilních zařízeních. Pro obecné systémy dokonce existují korpusy s desítkami,stovkami i více hodin promluv, které je možné využít při vytváření těchto systémů.

Tyto korpusy však obsahují, ve většině případů, pouze \uv{standardní}\footnote{Slovním spojením \uv{standardní řeč} je myšlena řeč neobsahující výrazné řečové vady, případně jiné formy produkce a často v nepříliš akusticky náročném prostředí.} řeč. Pokud je snaha vytvořit nebo ověřit funkčnost systému za specifických podmínek (ať už se jedná o rušné prostředí či speciální typy promluv), tak je nezbytné získat potřebná data, viz \ref{chap:construction:corpus}.

\subsection{Analýza získaných dat}
\label{chap:construction:analysis:data}

Získaný korpus obsahuje přes 10 hodin akustických záznamů promluv a více či méně přesných přepisů\footnote{I přes nemalou snahu a několikastupňovou kontrolu, je téměř jisté, že by nebylo obtížné najít přepis, který obsahuje chybu například ve formě překlepu.}. V momentě, kdy jsou k dispozici data, je možné se podívat na specifika EL řeči a případně porovnat se zdravým řečníkem.

Pro potřeby porovnání byl použit začátek promluvy \textit{\uv{Akcie Komerční banky...}}. Tato promluva je součástí standardní množiny vět používaných při vytváření řečových korpusů na KKY při ZČU. Tím pádem je k dispozici v relativně velkém množství příkladů pro zdravé řečníky. Tato věta je součástí také korpusu EL řeči.

Na obr. \ref{fig:construction:analysis:comparison} je zobrazen průběh amplitudy a spektrogram vybrané promluvy pro zdravého (obr. \ref{fig:construction:analysis:comparison:normal}) a EL (obr. \ref{fig:construction:analysis:comparison:el}) řečníka. Už na první pohled je možné zaznamenat určité rozdíly i přesto, že obsah obou promluv je identický. Prvním takovým je délka promluvy. V případě zdravého řečníka je v průměru\footnote{Vypočteno na základě 10 náhodně vybraných promluv ze standardně používaného korpusu na katedře kybernetiky ZČU.} o celou $1$ vteřinu kratší než v případě EL řeči. Tempo řeči je samozřejmě velmi individuální, ale z principu je EL řeč pomalejší. Z průběhu signálu na obr. \ref{fig:construction:analysis:comparison:el} je patrné, že řečník dělá výraznější pauzy mezi jednotlivými slovy promluvy. To je často způsobené potřebou naplnit jícen vzduchem. Po TL je dýchání realizováno přes tracheu a pokud nebyl voperován shunt (více v \ref{chap:cause:treatment:tracheo}), tak je trvale oddělen hrtan a hltan. Přesto, pro produkci některých neznělých fonémů je potřeba exhalovat vzduch z dutiny ústní. Zkušený EL řečník to dělá naprosto automaticky, nicméně \uv{polykání} vzduchu zabere nějaký čas. Nevyhnutelným důsledkem je pak velmi častý výskyt samovolného říhání v průběhu promluvy\footnote{Fakt, že je říhání jako neřečová událost běžnou součástí téměř každé promluvy, vedl k ignorování těchto událostí během anotace.}.

\begin{figure}[htpb]
  \centering
  \begin{subfigure}[b]{0.45\textwidth}
    \includegraphics[width=\textwidth]{./ch5-construction/img/energy_spec_normal.png}
    \caption{Zdravý řečník}
    \label{fig:construction:analysis:comparison:normal}
  \end{subfigure}
  %
  \begin{subfigure}[b]{0.45\textwidth}
    \includegraphics[width=\textwidth]{./ch5-construction/img/energy_spec_el.png}
    \caption{EL řečník}
    \label{fig:construction:analysis:comparison:el}
  \end{subfigure}
  \caption{Průběh a spektrogram promluvy a vyznačenou energií promluvy zdravého a EL řečníka.}
  \label{fig:construction:analysis:comparison}
\end{figure}

% \begin{figure}[hbpt]
%   \centering
%   \includegraphics[width=0.9\textwidth]{./ch5-construction/img/energy_spec_normal.png}
%   \caption{Průběh a spektrogram promluvy a vyznačenou energií promluvy.}
%   \label{fig:construction:normal_speech}
% \end{figure}

Svou roli může hrát i snaha správně artikulovat. Při používání EL je nezbytné, aby bylo produkované řeči alespoň trochu rozumět. A pokud se dobře artikuluje, není snadné mluvit rychle. Při nahrávání bylo velmi běžné, že v průběhu promluvy řečník udělal pauzu, aby mohl lépe umístit EL, protože jeho umístění má velký vliv na kvalitu produkované řeči. Nicméně je třeba říci, že tempo není a priory pro ASR systémy problém, protože různá délka fonémů je v relativně snadno modelována, viz část \ref{chap:asr:acoustic}.

Dalším způsobem jak ukázat rozdíly mezi promluvou zdravého a řečníka s EL, je srovnání ve frekvenční oblasti. Pro větší názornost jsou na obr. \ref{fig:construction:spectrogram} zobrazena společně spektra ukázkové promluvy zdravého řečníka (\ref{fig:construction:spectrogram:normal}) a toho s EL (\ref{fig:construction:spectrogram:el}). Obsah obou promluv je identický a přesto jsou obě spektra odlišná.

Prvním markantním rozdílem je mnohem větší zastoupení šumu v úsecích \uv{ticha}, viz obr. \ref{fig:construction:spectrogram:el}. To je nepochybně způsobnemo samotným EL, který řečník nevypíná mezi jednotlivými slovy. Na obr. \ref{fig:construction:el_speech} je zřetelně patrný, zejména na průběhu energie, šum před prvním a druhým slovem promluvy. Zajímavá je přítomnost šumu v celém frekvenčním spektru, přestože EL produkuje konstantní buzení. To je ve spektru (obr. \ref{fig:construction:spectrogram:el}) viditelné jako výrazná souvislá linie v nízkých frekvencích. Přítomnost šumu ve vyšších frekvencích je způsobena umístěním mikrofonu, který je nalepen přímo na pokožku, a tím pádem snímá namodulované vibrace, přenášené měkkou tkání. Tento fakt se potvrdil v dalších etapách nahrávání (viz část \ref{chap:realisation:corpus}), kde je použit studiový mikrofon vzdálený od úst minimálně 15 cm a tyto vibrace již nezaznamenává. Nicméně z pohledu použitelnosti nějakého budoucího systému je nezbytné počítat i se situací, kdy mikrofon zaznamenává i vibrace přenášené tkání.

Dalším významným rozdílem je absence vyšších frekvencí u většiny produkovaných fonémů. Výjimku tvoří afrikáty $/c/$ a $/\check{c}/$, u kterých jsou hlasivky (u zdravého jedince) v klidu, a vznikají uvolněním nahromaděného vzduchu v dutině ústní\footnote{Nahromadění vzduchu je realizováno přitisknutím jazyka k přední/zadní části horního patra.} \cite{Psutka2006}. U těchto fonémů není, u  řečníka po TL, mechanizmus produkce těchto fonémů ovlivněn. Problémem teoreticky může být zdroj vzduchu, jelikož jej z plic není možné dostat do dutiny ústní, ale jak už bylo zmíněno (a spektrogram to potvrzuje), zkušený uživatel EL se dokáže adaptovat.

Absence vyšších frekvencí se dá vysvětlit použitím EL, kde samotný EL má vždy konstantní frekvenci buzení a dále tím, že nedochází k modulaci ve všech dutinách vokálního traktu. Nicméně nejdůležitější složky, zajišťující srozumitelnost, se vyskytují ve frekvenčním pásmu od 1 kHz do 3 kHz. Vyšší frekvence se a priory podílejí na zabarvení hlasu.

\begin{figure}[htpb]
  \centering
  \begin{subfigure}[b]{0.4\textwidth}
    \includegraphics[width=\textwidth]{./ch5-construction/img/spectrogram_normal.png}
    \caption{Zdravý řečník}
    \label{fig:construction:spectrogram:normal}
  \end{subfigure}
  %
  \begin{subfigure}[b]{0.4\textwidth}
    \includegraphics[width=\textwidth]{./ch5-construction/img/spectrogram_el.png}
    \caption{EL řečník}
    \label{fig:construction:spectrogram:el}
  \end{subfigure}
  \caption{Spektrogram promluvy \uv{Akcie Komerční banky} dvou řečníků.}
  \label{fig:construction:spectrogram}
\end{figure}

Dalším způsobem jak porovnat řeč zdravého a EL řečníka je pomocí analýzy jednotlivých fonémů. Na obr. \ref{fig:construction:phonemes:k}, \ref{fig:construction:phonemes:m} a \ref{fig:construction:phonemes:c} jsou zobrazeny průběhy amplitudy v čase\footnote{Hodnoty času odpovídají časům výskytu v původní promluvě.} pro fonémy $/k/$, $/m/$ a $/\check{c}/$. V případě $/k/$ a $/m/$ (obr. \ref{fig:construction:phonemes:k} a \ref{fig:construction:phonemes:m}) se jedná o okluzivy, kde v prvním případě jde o neznělou plozivu a v druhém o znělou plozivu. Tyto fonémy obecně vznikají uzavřením vydechovaného proudu vzduchu pomocí artikulačních orgánů, což se projeví jako krátká pauza (tzv. okluze). Po té následuje náhlé jednorázové uvolnění překážky a únik nahromaděného vzduchu, tzv. exploze \cite{Psutka2006}. Takto popsáno to samozřejmě funguje u zdravého jedince, ale u EL řečníka jde sice o stejný mechanizmus, ale s tím rozdílem, že vzduch nepochází z plic, ale z hltanu. Dalším rozdílem je samozřejmě absence hlasivek.

Foném $/k/$ je tedy zástupcem neznělých fonémů, ty se vyznačují tím, že do jejich produkce nevstupují hlasivky, které jsou v klidu. Zdrojem buzení je tedy šum, viz část \ref{chap:asr}. Pokud se podíváme na průběh amplitudy v čase u zdravého řečníka (obr. \ref{fig:construction:phonemes:k:normal}), tak zde není vidět žádný periodický signál. Hlasivky jsou tedy opravdu v klidu. Oproti tomu u EL řečníka (obr. \ref{fig:construction:phonemes:k:el}) je jasně patrné, že je zde přítomno aktivní buzení vytvořené EL. Ve frekvenční oblasti je zobrazeno tzv. amplitudové spektrum, které znázorňuje vývoj amplitudy signálu ve frekvenci. V případě zdravého řečníka odpovídá vývoj předpokladům, není zde žádná výrazná frekvence a také nedochází k výraznému útlumu. Přestože se v obou případech jedná o stejný foném, tak z časového i frekvenčního průběhu je zřejmé, že parametry signálu se u obou řečníku diametrálně liší.

\begin{figure}[htpb]
  \centering
  \begin{subfigure}[b]{0.45\textwidth}
    \includegraphics[width=\textwidth]{./ch5-construction/img/signal-normal_k.png}
    \caption{Zdravý řečník}
    \label{fig:construction:phonemes:k:normal}
  \end{subfigure}
  %
  \begin{subfigure}[b]{0.45\textwidth}
    \includegraphics[width=\textwidth]{./ch5-construction/img/signal-el_k.png}
    \caption{EL řečník}
    \label{fig:construction:phonemes:k:el}
  \end{subfigure}
  \caption{Průběh amplitudy $/k/$ v časové a frekvenční oblasti fonému u normálního a EL řečníka}
  \label{fig:construction:phonemes:k}
\end{figure}

Jako druhý ukázkový foném slouží $/m/$. Jedná se o plozivu, ale v tomto případě znělou. U těchto fonémů hrají velký vliv hlasivky, protože jsou zdrojem buzení. Z obr. \ref{fig:construction:phonemes:m:normal} je toto buzení zřetelné ve formě periodického průběhu amplitudy. U EL řečníka (obr. \ref{fig:construction:phonemes:m:el}) je také vidět periodický signál, ale úplně jiného charakteru. Svým způsobem dost podobný tomu, který je zřetelný u fonému $/k/$. Rozdíl je zřetelný i ve frekvenční oblasti, kdy u EL řečníka nedochází k útlumu ve střední oblasti frekvenčního spektra.

\begin{figure}[htpb]
  \centering
  \begin{subfigure}[b]{0.45\textwidth}
    \includegraphics[width=\textwidth]{./ch5-construction/img/signal-normal_m.png}
    \caption{Zdravý řečník}
    \label{fig:construction:phonemes:m:normal}
  \end{subfigure}
  %
  \begin{subfigure}[b]{0.45\textwidth}
    \includegraphics[width=\textwidth]{./ch5-construction/img/signal-el_m.png}
    \caption{EL řečník}
    \label{fig:construction:phonemes:m:el}
  \end{subfigure}
  \caption{Průběh amplitudy fonému $/m/$ v časové a frekvenční oblasti fonému u normálního a EL řečníka}
  \label{fig:construction:phonemes:m}
\end{figure}

Posledním ukázkovým fonémem je již zmiňované $/\check{c}/$. Jedná se o neznělý foném, který vzniká přiložením jazyku k zadní části horního patra. Tím je zadržen vzduch v dutině ústní a vzniká krátká pauza. Uvolněním pak dochází k explozi a vytvoření zvuku. \cite{Psutka2006} Do produkce se nezapojují hlasivky a produkovaný zvuk by měl být dostatečně intenzivní, aby jej (v případě EL řečníka) tolik neovlivňoval EL. Tím pádem by měl být průběh signálu, u obou řečníků podobný, a to jak v časové, tak i ve frekvenční oblasti, viz obr. \ref{fig:construction:phonemes:c}.

\begin{figure}[htpb]
  \centering
  \begin{subfigure}[b]{0.45\textwidth}
    \includegraphics[width=\textwidth]{./ch5-construction/img/signal-normal_c.png}
    \caption{Zdravý řečník}
    \label{fig:construction:phonemes:c:normal}
  \end{subfigure}
  %
  \begin{subfigure}[b]{0.45\textwidth}
    \includegraphics[width=\textwidth]{./ch5-construction/img/signal-el_c.png}
    \caption{EL řečník}
    \label{fig:construction:phonemes:c:el}
  \end{subfigure}
  \caption{Průběh amplitudy fonému $/\check{c}/$ v časové a frekvenční oblasti fonému u normálního a EL řečníka}
  \label{fig:construction:phonemes:c}
\end{figure}

Z doposud provedené analýzy plyne, že EL řeč je v mnoha charakteristikách odlišná od té produkované zdravým řečníkem. Zejména u porovnání ve frekvenční oblasti (obr. \ref{fig:construction:phonemes:k} a \ref{fig:construction:phonemes:m}) je to nejvíce patrné.
% Tento fakt nepochybně přispívá k tomu, že standardní obecné modely rozpoznávání řeči nedosahují takové přesnosti jako v případě běžné promluvy, viz dále.

% \csvautotabular{./ch5-construction/test.csv}
