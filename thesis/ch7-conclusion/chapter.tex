% !TEX root = ../thesis.tex
\chapter{Závěr}
\label{chap:conclusion}

% Přestože nádorovitá onemocnění hrtanu nepatří mezi nejčastější onemocnění, je
% jim věnována velká pozornost, protože případné následky mohou výrazně zhoršit
% kvalitu života pacienta. Klasické rehabilitační techniky (zmíněné v části
% \ref{sec:cause:treatment}) dokážou navrátit schopnost mluvit, ale kvalita
% produkované řeči nemusí být rozhodně optimální. Například použití
% elektrolarynxu sice neklade na uživatele vysoké nároky co se týče učení, ale
% kvalita řeči není vůbec přirozená. Oproti tomu pomocí jícnového hlasu je
% produkován relativně kvalitní hlas, ale k edukaci je potřeba vynaložit opravdu
% nemalé úsilí. Jako ideální se může jevit použití tracheoezofageální píštěle,
% která umožňuje proudění vzduchu z plic do dutiny ústní. Produkovaný hlas se v
% tomto případě vyznačuje vysokou kvalitou, dobrou srozumitelností,
% individuálním zabarvením a relativně dlouhou fonační dobou. Za nedostatek se
% dá považovat nutnost pravidelně čistit a měnit píštěle. Existují i další
% metody, popsané v části \ref{sec:cause:treatment}, ale ty jsou zatím používány
% spíše autorskými týmy a o masovém použití se rozhodně nedá hovořit. Bohužel
% žádná z~technik není univerzální řešení, a proto se je lékaři
% stále snaží zdokonalovat a tím zkvalitňovat život pacientů.

% Pomoc může poskytnout rozvoj technologií zpracovávajících přirozenou řeč
% umožňující jejich využití ve stále rozmanitějších oblastech lidské činnosti.
% Jednou z takovýchto oblastí představuje zpracování tzv. \uv{tiché} řeči (SSI),
% při které není produkována slyšitelná promluva. K jejímu záznamu je tak
% potřeba zaznamenat jiný druh dat. Jelikož produkce mluveného slova představuje
% komplexní proces, kterého se běžně účastní mozek, nervový systém, hlasivky,
% artikulační orgány a svaly, je v~případě absence akustického signálu nutné
% snímat aktivitu jednotlivých částí zapojených do tohoto procesu. V současnosti
% se vyvíjí systémy (jejich přehled je v části \ref{ch:ssi}) snažících se
% rozpoznat obsah promluvy z aktivity artikulačních orgánů, mozkové aktivity či
% zaznamenané tělem šířené řeči. Takovéto systémy umožňují použití řečových
% technologií například v prostředích s nadměrným hlukem (kde běžné systémy
% nedosahují požadovaných výkonů) nebo dokonce lidmi trpící ztrátou hlasu.

% Schopnost pracovat bez akustických dat předurčuje SSI jako další možnost při
% rehabilitaci hlasu. Zachycením a zpracováním promluvy dovolují vygenerovat
% její obsah pomocí TTS a dosáhnout přirozeného a kvalitního hlasu (v ideálním
% případě dokonce hlasem mluvčího, který měl před ztrátou hlasu).

% Nejblíže reálnému nasazení je technologie postavená na snímání tělem šířené
% řeči pomocí speciálního NAM mikrofonu (popsáno v části \ref{ssec:ssi:nam}).
% Tento přístup staví na faktu, že část energie promluvy je přenášena i skrze
% tkáně lidského těla. Speciálně vyvinutý mikrofon tuto řeč dokáže zachytit a
% zpracovat. Ke vzniku tělem šířené řeči je, ale potřeba hlasivek nebo jiného
% zdroje buzení (podobný princip jako elektrolarynx) což nepatrně omezuje
% možnosti využití v rámci rehabilitace hlasu.

% Nejoblíbenějším přístupem je jednoznačně zaznamenání artikulace mluvčího při
% promluvě. K tomuto účelu se vyvíjejí systémy využívající permanentních magnetů
% (část \ref{ssec:ssi:pma}), spojení ultrazvuku a kamery (část
% \ref{ssec:ssi:us}) či elektromyografie (část \ref{ssec:ssi:others}).
% Jednotlivé přístupy se sice nacházejí v různých stádiích vývoje, ale dohromady
% dávají tušit, že záznamem artikulace je možné získat dostatečně kvalitní
% informaci k~dekódování promluvy.

% Relativně futuristicky se na první pohled může jevit snaha BCI zachytit
% informace o promluvě již na samém počátku, tedy v mozku (zmíněno v části
% \ref{ssec:ssi:others}). V případě funkčnosti se však ještě rozšiřují
% schopnosti SSI. Bohužel tyto snahy jsou teprve na samém počátku a k prvním
% výsledků je ještě potřeba usilovného vývoje.

% Předkládaný přehled (část \ref{ch:ssi}) vyvíjených SSI demonstruje snahu
% tvůrčích týmů vytvořit funkční systém umožňující pracovat v podmínkách, ve
% kterých klasické přístupy selhávají. Naneštěstí většina prezentovaných
% technologií je v raném stádiu vývoje a k~reálnému nasazení je ještě potřeba
% vyřešit nespočet problémů. Velmi zajímavě se jeví možnost vzájemné kombinace
% prezentovaných přístupů. Z pohledu katedry kybernetiky Fakulty aplikovaných
% věd pak zejména spojení NAM a záznamu artikulace rtů pomocí kamery, z důvodu
% dostupnosti a zkušeností s těmito technologiemi.

% V současnosti není snadné určit, který z prezentovaných přístupů má největší
% naději stát masově používaným plnohodnotným SSI systémem, ale rozmanitost
% přístupů přináší optimismus v otázce zda je možné takovýto systém vůbec
% sestrojit.

