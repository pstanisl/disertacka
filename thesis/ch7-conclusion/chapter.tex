%% !TEX root = ../thesis.tex
\chapter*{Závěr}
\addcontentsline{toc}{chapter}{Závěr}
\label{chap:conclusion}

Předložená disertační práce se zabývá problematikou rozpoznávání řeči pacientů po totální laryngektomii, kteří komunikují pomocí elektrolarynxu. Motivací pro zpracování tohoto tématu bylo obohatit stávacíjí postupy využívané pro rehabilitaci hlasu o možnosti, které přináší využití moderních technologií, konkrétně možností automatického rozpoznávání řeči. V~kapitole \ref{chap:mot} byly vytyčeny cíle práce, jejichž naplnění bylo v~následujících odstavcích zhodnoceno.

V kapitole \ref{chap:cause} jsou přiblíženy nejčastější příčiny ztráty hlasu a metody užívané k~jeho rehabilitaci.
Pomocí klasických rehabilitačních technik lze pacientům navrátit možnost mluvit, ale kvalita produkované řeči nutně nemusí splňovat očekávání pacientů  a požadavky kladené na mluvčího okolím. Například použití
elektrolarynxu sice neklade na uživatele vysoké nároky co se týče učení, ale
kvalita řeči není vůbec přirozená. Oproti tomu pomocí jícnového hlasu je
produkován relativně kvalitní hlas, ale k~edukaci je potřeba vynaložit opravdu
nemalé úsilí. Jako ideální se může jevit použití tracheoezofageální píštěle,
která umožňuje proudění vzduchu z plic do dutiny ústní. Produkovaný hlas se v
tomto případě vyznačuje vysokou kvalitou, dobrou srozumitelností,
individuálním zabarvením a relativně dlouhou fonační dobou. Za nedostatek se
dá považovat nutnost pravidelně čistit a měnit píštěle. Existují i další
metody, popsané v~části \ref{chap:cause:treatment}, ale ty jsou zatím používány
spíše autorskými týmy a o masovém použití se rozhodně nedá hovořit. Bohužel
žádná z~technik nepředstavuje univerzální řešení, a proto se je lékaři
stále snaží zdokonalovat, a tím zkvalitňovat život pacientů. U všech aktuálně používaných metod rehabilitace hlasu
je patrný významný negativní dopad na psychiku pacienta, který se musí vyrovnat nejen se ztrátou vlastního hlasu, ale i s~ostychem, který provází opětovné snahy mluvit.

Pomoc s~rehabilitací hlasu mohou poskytnout řečové technologie zpracovávající přirozenou řeč. V~současnosti využívané obecné systémy automatického rozpoznávání řeči (ASR systémy) ale poskytují spolehlivé výsledky v~případě rozpoznávání promluv zdravého řečníka. V~případě, že se charakteristiky rozpoznávané řeči příliš líší ( např. řeč obsahuje vyšší množství šumu), může se u běžně uživaných ASR systémů projevit jejich nedostačná robustnost. Proto bylo pro potřeby návrhu ASR systému, který bude sloužit pro rozpoznávání řeči pacientů po totální laryngektomii, nutno pořídit řecová data odpovídající kvality. V~takovýchto promluvách se ukazuje jako problematická zejména přílišná podobnost produkovaných znělých a neznělých fonémů. Proto byla navázana spolupráce s~mluvčí, která prodělala TL a komunikuje pomocí elektrolarynxu. V~průběhu pěti let byly pořízeny 3 sady nahrávek, což odpovídá necelým 15 hodinám řečových dat. První sada je složená z vět, které jsou součástí interně využívaného řečového korpusu, 2. sada rozšiřuje řečový korpus o další sadu vět a problematických izolovaných slov. Ve 3. sadě jsou obsaženy další věty a slova respektující protažení problematických fonémů.
S~ohledem na toto lze i vytyčený cíl č. 2 považovat za naplněný.

Pro otestování robustnosti obecného ASR systému byla využita data poskytnutá řečníkem po totální laryngektomii. Testovaný systém vykázal pro trigramový jazykový model obsahující 1 milion unikátních slov úspěšnost rozpoznávání slov pouze $\boldsymbol{18,49~\%}$. Takto nízká přesnost rozpoznávání indikovala nutnost navrhnout individuální akustický model, který bude reflektovat specifika řeči produkované s~využitím EL. Při využití dat z EL korpusu dosáhl systém přesnosti rozpoznávání slov $\boldsymbol{83,33~\%}$.

Následně byl minimalizován vliv jazykového modelu, viz \ref{chap:construction:results}, a byly hledány optimální parametry akustického modelu. Byl ověřen vliv maximálního počtu unikátních stavů HMM modelu a vzorkovací frekvence na přesnost rozpoznávání. Nejlepších výsledků bylo dosaženo pro model pracující s~maximálně 4096 unikátními stavy a vzorkovací frekvencí 16kHz. Pro HMM-GMM bylo dosaženo přesnosti rozpoznávání $\boldsymbol{81,20~\%}$, pro HMM-DNN pak $\boldsymbol{85,23~\%}$. Po provedení analýzy získaných výsledků se jako problematické ukázalo rozpoznávání neznělých fonémů, proto bylo přistoupeno  k~redukci fonetické sady prostřednictvím různých kombinací náhrady neznělých fonémů za znělé, což ve většině případů nemělo pozitivní dopad. Proto byly navrženy další úpravy ASR systému, konkrétně protaženy neznělé fonémy a následně byl ASR systém rozšířen o tzv. duration model akcentující právě délku fonémů. Na základě provedených experimentů se jako vhodné ukázalo prodloužit neznělé fonémy na dvojnásobek jejich původní délky. Úspěšnost rozšířeného modelu dosahovala $\boldsymbol{88,54~\%}$.

S ohledem na získané výsledky vyvstala potřeba porovnat schopnosti rozpoznávání člověka a stroje. Za tímto účelem byl navržen tzv. poslechový test, jehož princip byl přiblížen v~části \ref{chap:realisation:listening}. Pro model s~výše navrženými optimálními parametry ( max. 4096 unikatní stavů a vzorkovací frekvencí 16 kHz) stroj dosál přesnosti rozpoznávání  $\boldsymbol{69,91~\%}$ pro případ izolovaných slov a $\boldsymbol{54,82~\%}$ pro případ bigramů. U člověka bylo dosaženo přesnosti $\boldsymbol{74,47~\%}$, resp. $\boldsymbol{66,24~\%}$. Při zohlednění umělého protažení akustických dat dosáhl stroj přesnosti $\boldsymbol{94,36~\%}$ pro izolovaná slova, resp. $\boldsymbol{98,80~\%}$ pro bigramy. Po následném rozšíření systému rozpoznávání o duration model dosáhl stroj úspěšnosti $\boldsymbol{94,42~\%}$, resp. $\boldsymbol{98,81~\%}$. S ohledem na výsledky poskytnuté ASR systémem byla nahrána další sada řečových dat respektující protažení neznělých fonémů. Na takto rozšířené testovací sadě bylo dosaženo přesnosti rozpoznávání $\boldsymbol{87,02~\%}$ na úrovni fonémů. Tím bylo ověřeno, že model natrénovaný na uměle protažených datech je schopen rozpoznávat i data reálná, a lze ho tedy využít jako základ pro vývoj trenažéru, který bude výukovým nástrojem pro osvojení schopnosti řečníka protahovat neznělé fonémy.

S ohledem na výše uvedené výsledky a z~nich vyvozené závěry lze říci, že vytyčené cíle disertační práce byly naplněny a s~ohledem na aktuálnost řešené problematiky lze získané poznatky využít jako základ další práce.










%
%Doplnění chybějící informace je možné v~případě využití různých druhů multimodálních systémů, které se snaží různými způsoby, zejména na základě snímání artikulace. Tyto systémy jsou však ještě musí urazit dlouhou cestu  k~dosažení produkčního nasazení. Hlavním problémem jsou zatím výkony některých systému a také jimi kladené nároky na mluvčího.
%
%Tato práce se zaměřila na možnost doplnění informace pomocí drobné cílené změny produkované řeči a úpravy ASR systému tak, aby tato změna byla co možná nejvíce akcentována. V~části \ref{chap:realisation:augmentation} je představena a následně otestována možnost protahování vybraných fonémů  k~minimalizaci problémů ASR systémů s~krátkými promluvami. K ověření funkčnosti konceptu byla data nejprve uměle protažena. Dosažené výsledky (viz tab. \ref{tab:realisation:augmentation:influence} a tab. \ref{tab:realisation:augmentation:influence:tdnn}) potvrzují nezanedbatelné zlepšení pouze drobnou úpravou pronášených promluv. Pokud se navíc upraví ASR systém tak, aby toto prodloužení příslušně využil, dojde ještě  k~dalšímu zlepšení (viz tab. \ref{tab:realisation:duration:layers}, \ref{tab:realisation:duration:context:symetric} a \ref{tab:realisation:duration:context:asymetric}).
%
%K reálně použitelnému systému jsou samozřejmě zapotřebí reálně protažená data, ale díky těm umělým bylo možné určit vhodnou míru protažení. V~části \ref{chap:realisation:augmentation:real} je pak představen jednoduchý a pro řečníka intuitivní způsob získání reálně protažených dat. Tento způsob společně s~uměle protaženými daty jsou pak stavebními kameny představeného trenažéru (viz část \ref{chap:realisation:trainer}), který slouží ke snadnému trénování řečníka v~protahování. Sekundární funkcí je pak získání reálných dat, které pak mohou posloužit  k~vytvoření co možná nejlepšího ASR systému.
%
%Hlavním nedostatkem představených ASR systému postavených na duration modelech (viz část \ref{chap:realisation:durationmodels}) je v~současné chvíli jejich offline funkcionalita. V~současné podobě je není možné využít jejich použití v~systémech zpracovávající promluvy v~reálném čase. Principiálně tomu však nic nebrání.
%
%V~ideálním případě pak může ASR systém s~duration modelem ve spojení s~TTS systémem posloužit v~určitých situacích ke snadnější komunikaci TL řečníků s~ostatními lidmi. Při diskuzích s~pacienty po TL laryngektomii se jako velmi lukrativní jeví využití takovéhoto systému například při telefonování, protože to drasticky redukuje míru stresu a strachu z reakce druhé strany. Ač se to na první pohled nemusí zdát, může to vést  k~významnému zlepšení kvality života a snížení psychologické zátěže TL řečníků. Zejména pak v~kritické době následující po operaci.
%
%% TODO: psychologický aspekt
%% TODO: popis zhodnocení výsledků
%% TODO: využití zejména v~telefonii
%% TODO: problém s~offline zpracováním dává možnost budoucího vývoje
%% TODO: zmínit, že navržený postup je nezávislý na zvolené standardní metodě rehabilitace, principiálně je to u všech stejné
%
%% Pomoc může poskytnout rozvoj technologií zpracovávajících přirozenou řeč
%% umožňující jejich využití ve stále rozmanitějších oblastech lidské činnosti.
%% Jednou z takovýchto oblastí představuje zpracování tzv. \uv{tiché} řeči (SSI),
%% při které není produkována slyšitelná promluva. K jejímu záznamu je tak
%% potřeba zaznamenat jiný druh dat. Jelikož produkce mluveného slova představuje
%% komplexní proces, kterého se běžně účastní mozek, nervový systém, hlasivky,
%% artikulační orgány a svaly, je v~případě absence akustického signálu nutné
%% snímat aktivitu jednotlivých částí zapojených do tohoto procesu. V~současnosti
%% se vyvíjí systémy (jejich přehled je v~části \ref{ch:ssi}) snažících se
%% rozpoznat obsah promluvy z aktivity artikulačních orgánů, mozkové aktivity či
%% zaznamenané tělem šířené řeči. Takovéto systémy umožňují použití řečových
%% technologií například v~prostředích s~nadměrným hlukem (kde běžné systémy
%% nedosahují požadovaných výkonů) nebo dokonce lidmi trpící ztrátou hlasu.
%
%% Schopnost pracovat bez akustických dat předurčuje SSI jako další možnost při
%% rehabilitaci hlasu. Zachycením a zpracováním promluvy dovolují vygenerovat
%% její obsah pomocí TTS a dosáhnout přirozeného a kvalitního hlasu (v~ideálním
%% případě dokonce hlasem mluvčího, který měl před ztrátou hlasu).
%
%% Nejblíže reálnému nasazení je technologie postavená na snímání tělem šířené
%% řeči pomocí speciálního NAM mikrofonu (popsáno v~části \ref{ssec:ssi:nam}).
%% Tento přístup staví na faktu, že část energie promluvy je přenášena i skrze
%% tkáně lidského těla. Speciálně vyvinutý mikrofon tuto řeč dokáže zachytit a
%% zpracovat. Ke vzniku tělem šířené řeči je, ale potřeba hlasivek nebo jiného
%% zdroje buzení (podobný princip jako elektrolarynx) což nepatrně omezuje
%% možnosti využití v~rámci rehabilitace hlasu.
%
%% Nejoblíbenějším přístupem je jednoznačně zaznamenání artikulace mluvčího při
%% promluvě. K tomuto účelu se vyvíjejí systémy využívající permanentních magnetů
%% (část \ref{ssec:ssi:pma}), spojení ultrazvuku a kamery (část
%% \ref{ssec:ssi:us}) či elektromyografie (část \ref{ssec:ssi:others}).
%% Jednotlivé přístupy se sice nacházejí v~různých stádiích vývoje, ale dohromady
%% dávají tušit, že záznamem artikulace je možné získat dostatečně kvalitní
%% informaci k~dekódování promluvy.
%
%% Relativně futuristicky se na první pohled může jevit snaha BCI zachytit
%% informace o promluvě již na samém počátku, tedy v~mozku (zmíněno v~části
%% \ref{ssec:ssi:others}). V~případě funkčnosti se však ještě rozšiřují
%% schopnosti SSI. Bohužel tyto snahy jsou teprve na samém počátku a  k~prvním
%% výsledků je ještě potřeba usilovného vývoje.
%
%% Předkládaný přehled (část \ref{ch:ssi}) vyvíjených SSI demonstruje snahu
%% tvůrčích týmů vytvořit funkční systém umožňující pracovat v~podmínkách, ve
%% kterých klasické přístupy selhávají. Naneštěstí většina prezentovaných
%% technologií je v~raném stádiu vývoje a k~reálnému nasazení je ještě potřeba
%% vyřešit nespočet problémů. Velmi zajímavě se jeví možnost vzájemné kombinace
%% prezentovaných přístupů. Z pohledu katedry kybernetiky Fakulty aplikovaných
%% věd pak zejména spojení NAM a záznamu artikulace rtů pomocí kamery, z důvodu
%% dostupnosti a zkušeností s~těmito technologiemi.
%
%% v~současnosti není snadné určit, který z prezentovaných přístupů má největší
%% naději stát masově používaným plnohodnotným SSI systémem, ale rozmanitost
%% přístupů přináší optimismus v~otázce zda je možné takovýto systém vůbec
%% sestrojit.
%
