% !TEX root = ../thesis.tex
\section{Poslechový test a porovnání výsledků člověka a stroje}
\label{chap:realisation:listening}

V předchozím textu byly prezentovány různé dosažené výsledky, ale ty zatím nedokázali odpovědět na zásadní otázku: \uv{Dokáže se stroj\footnote{Stroj je zde reprezentován systémem automatického rozpoznávání řeči.} vyrovnat človéku?}. Přestože je EL řeč na první poslech obtížně srozumitelná, tak již po krátké době je člověk schopen obstojně poruzumnět. S přibývajícím časem se do určité míry porozumnéní ještě zlepšuje. Jak je na tom tedy stroj v porovnání s člověkem?

Ještě než je vůbec možné na tuto otázku odpovědět, tak je dobré si odpovědět na otázku: \uv{Jakým způsobem porovnat schopnosti člověka a stroje?}. K tomu může posloužit poslechový test, ve kterém mají posluchačí za úkol určit, z předem definovaných možností, co je obsahem promluv. Otestování schopností stroje pak probíhá pomocí experimentu, kde na vstupu systému jsou stejné promluvy, která jsou součástí poslechového testu, a výstupem je přepis. Metrika je pak počítána na základě správně/špatně určeného obsahu promluv. Prostým porovnáním správně určeného obsahu člověka a stroje je pak možné odpovědět na první \uv{položenou} otázku.

Při přípravě experimentu vykrystalizovaly tyto varianty poslechového testu:

\begin{itemize}
  \item test na izolovaných slovech,
  \item test na slovních bigramech.
\end{itemize}

\noindent Tím, že promluvy obsahují pouze izolovaná slovo (v druhém případě dvojici slov) je do značné míry eliminován vliv kontextu. Ten v mnoha případech pomáhá se správným určením významu i přesto, že nebylo dobře rozumnět. Pokud se bude experiment skládat z množiny promluv, která obsahují pouze slova popsaná v části \ref{chap:realisation:corpus}, tak bude možné určit, jak \uv{dobře} dokáže člověk (stroj) určit význam těchto slov a případně je od sebe odlišit.

\subsection{Izolovaná slova}
\label{chap:realisation:listening:isolated}

Určení významu slova, které bylo vysloveno v klidném prostředí se jeví jako velice jednoduchý úkol, ale pokud jej vyslovil řečník používající EL, tak už to tak snadné být nemusí. Zvlášť pokud se jedná o slova v popsaná v \ref{chap:realisation:corpus}. Účastníci poslechového testu na izolovaných slovech mají za úkol postupně vyslechnout $320$ nahrávek izolovaných slov a vybrat jednu z předem definovaných odpovědí:

\begin{enumerate}[label=\alph*)]
  \item slovo A \textit{(např. kosa)},
  \item slovo B \textit{(např. koza)},
  \item nemohu rozhodnout.
\end{enumerate}

\noindent Ve výčtu možností je vždy skutečně pronesené slovo a k němu pak varianta lišící se pouze znělostí jednoho fonému. První dvě možnosti jsou vždy v abecedním pořadí. Nahrávky použité v rámci poslechového testu pocházejí z 2. etapy nahrávání. Poslechového testu se účastnilo $19$ subjektů z řad kolegů.

Výstupem poslechového testu je tabulka s procentuálním zastoupením jednotlivých odpovědí pro každou nahrávku. V tab. \ref{tab:realisation:listening:isolated} je zobrazen výňatek získaných výsledků. Správné odpovědi jsou zvýrazněny tučně. První příklad reprezentuje situaci kdy účastník nebyl jednoznačně schopen určit význam slova. Druhý příklad ukazuje situaci, kdy všichni účastníci vybrali z komplementárních slov vždy pouze jediné, a to nehledě na to, které jim bylo ve skutečnosti puštěno. V tomto konkrétním případě tedy posluchači vždy \uv{slyšeli} slovo \uv{koza}. Dalo by se tedy usuzovat, že slova \uv{kosa} je akusticky identické se slovem \uv{koza}. Poslední případ reprezentuje situaci, kdy většina účastníků byla schopna určit správný význam slova. Celková přesnost dosáhla hodnoty $Acc_w^{human} = 70,47\ \%$ a byla vypočtena vzorcem

\begin{equation}
  Acc_w^{human} = \frac{1}{n} \sum_{i=1}^{N} f_i * 100,
  \label{eq:realisation:accuracy:human}
\end{equation}

\noindent kde $n=320$ a $f_i$ se rovná relativní četnosti správných odpovědí na otázku $i$ v poslechovém testu s izolovanými slovy.

\begin{table}[htpb]
  \centering
  \def\arraystretch{1.5}
  \pgfplotstabletypeset[
    col sep=semicolon,
    string type,
    columns/word/.style={column name=Slovo, column type={|l}},
    columns/option1/.style={column name=Možnost \textit{\textbf{a)} [\%]}, column type={|r}},
    columns/option2/.style={column name=Možnost \textit{\textbf{b)} [\%]}, column type={|r}},
    columns/option3/.style={column name=Možnost \textit{\textbf{c)} [\%]}, column type={|r|}},
    every head row/.style={after row=\hline, before row=\hline},
    every row no 1/.style={after row=\hline},
    every row no 3/.style={after row=\hline},
    every last row/.style={after row=\hline},
  ]{./ch5-construction/tabs/0206-isolated.csv}
  \caption{Ukázka výsledku poslechového testu na izolovaných slovech.}
  \label{tab:realisation:listening:isolated}
\end{table}

\subsection{Slovní bigramy}
\label{chap:realisation:listening:bigrams}

V druhém poslechovém testu mají posluchači za úkol vyslechnout $333$ nahrávek slovních bigramů\footnote{Nahrávky obsahují dvě po sobě vyslovená slova.} a vybrat jednu z předem definovaných odpovědí. Ty mají vždy tento formát

\begin{enumerate}[label=\alph*)]
  \item slovo A + slovo A \textit{(např. kosa + kosa)},
  \item slovo A + slovo B \textit{(např. kosa + koza)},
  \item slovo B + slovo A \textit{(např. koza + kosa)},
  \item slovo B + slovo B \textit{(např. koza + koza)}.
\end{enumerate}

\noindent Je zřejmé, že to představuje všechny kombinace, které lze z dvojice slov vytvořit. Rozšířený řečový korpus, tak jak je popsaný v části \ref{chap:realisation:corpus}, ale neobsahuje tento typ nahrávek. Tím pádem je potřeba je vytvořit \uv{uměle}. Což není velký problém, každá nahrávka izolovaného slova obsahuje minimálně $0,5\ s$ ticha na svém začátku a konci. Pokud jsou tyto nahrávky spojeny\footnote{Ke spojení je možné použít nástroj \textit{ffmpeg} nebo \textit{sox}.}, vznikne jediná nahrávka obsahující dvě zájmová slova oddělena krátkou pauzou. Z každé dvojice slov vznikly vždy dvě nahrávky lišící se pořadím slov.

Vyšší počet položek v testu je zapříčiněn faktem, že pro určitá slova existuje více než jedna kombinace s jiným slovem\footnote{Ve valné většině se jedná o slova obsahující písmena \textit{i/y}, která jsou v akustické formě identická. Příkladem může být dvojice \textit{nebyli + nepili} a \textit{nebili + nepili}.}. Ve snaze zkrátit, už tak docela náročený poslechový test, byly vygenerovány bigramy odpovídající pouze možnostem \textit{b)} a \textit{c)}. Účastníci poslechového testu o tom však nebyli informováni. Přesto tento poslechový test dokončilo pouze $12$ účastníků.

Stejně jako u testu s izolovanými slovy je výstupem testu tabulka obsahující procentuální zastoupení jednotlivých odpovědí na každou otázku. Tab. \ref{tab:realisation:listening:bigrams} obsahuje ukázku těchto výsledků. Stejně jako v předchozím případě, správné odpovědi jsou zvýrazněny tučně. Ačkoli jsou nyní v testu dvojice slov, tak dosažené výsledky do značné míry korespondují s výsledky z testu s izolovanými slovy (viz tab. \ref{tab:realisation:listening:isolated}). V prvním případě (\textit{borci + porci}) nebyli účastníci schopni jednoznačně určit význam slov v nahrávce. V druhém případě (\textit{kosa + koza}) všichni posluchači až na jednoho vybrali možnost \textit{d)}, tedy \textit{koza + koza}. Správný výběr jedním účastníkem lze považovat spíše za náhodu, protože u opačného pořadí slov již nikdo správnou odpověď nevybral. Je dobré zmínit, že účastníci v žádném poslechovém testu nebyli omezeni v počtu opětovného přehrání promluvy. Tím pádem je velmi pravděpodobné, že tento konrétní participant opakovaně poslouchal danou nahrávku a hledal rozdíl až nějaký drobný zaznamenal. Otazkou však je, jestli to spíše nebyla sugesce a to již zmíněné štěstí.

Poslední prezentovaný příklad zastupuje množinu odpovědí, kdy účastníci naprosto správně určili význam slov. Průměrná dosažená přesnost člověka, počítána pomocí rovnice (\ref{eq:realisation:accuracy:human}), dosáhla hodnoty $Acc_{p}^{human} = 66,24\ \%$.

\begin{table}[htpb]
  \centering
  \def\arraystretch{1.5}
  \pgfplotstabletypeset[
    col sep=semicolon,
    string type,
    columns/word/.style={column name=Slovní bigram, column type={|l}},
    columns/option1/.style={column name=Mož. \textit{\textbf{a)} [\%]}, column type={|r}},
    columns/option2/.style={column name=Mož. \textit{\textbf{b)} [\%]}, column type={|r}},
    columns/option3/.style={column name=Mož. \textit{\textbf{c)} [\%]}, column type={|r}},
    columns/option4/.style={column name=Mož. \textit{\textbf{d)} [\%]}, column type={|r|}},
    every head row/.style={after row=\hline, before row=\hline},
    every row no 1/.style={after row=\hline},
    every row no 3/.style={after row=\hline},
    every last row/.style={after row=\hline},
  ]{./ch5-construction/tabs/0207-bigrams.csv}
  \caption{Ukázka výsledku poslechového testu na dvojicích slov.}
  \label{tab:realisation:listening:bigrams}
\end{table}

\subsection{Výsledky porovnání}
\label{chap:realisation:comparison}

Výsledky poslechového testu ukázaly jak je na tom člověk. Nyní je potřeba zjistit jak je na tom stroj zastoupený ASR systémem. K tomu je nezbytné natrénovat akustický model a použít vhodný jazykový model. Jako akustický model je použit HMM-DNN model. KOnkrétně jde o DNN síť s $6$ vrstvami ($5$ skrytých vrstev, každá s $4096$ neurony), výstupní vrstva je typu softmax s dimenzí rovnou počtu HMM stavů. Jako parametrizace je použito PLP (12 kepstrálních koeficientů + delta + delta-delta parametry) a pro eliminaci vlivu kanálu \textit{CMN} počítané přes všechny nahrávky v rámci etapy. Výsledný příznakový vektor má dimenzi $40$. Vstupem neuronové sítě je pak parametrizované okénko mající kontext přes $5$ mikrosegmentů, tedy $t-5$ a $t+5$. Vstupní dimenze neuronové sítě je tedy $440$. Oproti dosavadním experimentům je v tomto případě použit vlastní real-time dekodér (více o dekódování v části \ref{chap:asr:decoding}). Tento LVCSR systém je optimalizován pro co nejnižší latenci a je schopný pracovat s velmi velkými slovníky. Ty mohou čítat miliony položek. Tento dekodér byl vyvinut na katedře kybernetiky Fakulty aplikovaných věd.

Z poslechových testů jsou k dispozici dva výsledky. První reprezentuje schopnost určit význam izolovaného slova. Druhý schopnost rozeznat dvě velmi podobná slova. Pro potřeby porovnání, s těmi dosaženými ASR systémem ,jsou vytvořeny celkem $3$ experimenty využívající výše popsaný Kaldi model.

První experiment odpovídá poslechovému testu s izolovanými slovy a jeho základem je zerogramový LM obsahující více než 1 milion slov. Většina předchozích experimentů využívala fonémový zerogramový model, aby bylo možné eliminovat vliv LM na výsledné přesnosti. U těchto experimentů je tento LM nevyhovující, protože cílem je správně rozpoznat celé slovo, a proto je využit slovní jazykový model. U zerogramového modelu mají všechny položky stejnou pravděpodobnost, tím je zaručeno, že nebudou preferována četnější slova. LM je natrénovaný na textech pocházejích z novinových článků, webových zpravodajských serverů, filmových titlků a přepisů televizních pořadů. Využití takto velkého LM vychází z představy, že i člověk má velkou slovní zásobu a dopředu neví co bude obsahem konkrétní promluvy v rámci testu. Tento test je pojmenován jako \uv{onemil}.

Ve skutečnosti však, v rámci poslechového testu, účastníci znají seznam slov zahrnutých v testu a mají tak určitou výhodu oproti \uv{onemil} nastavení. Ke kompenzaci tohoto faktu, je vytvořen druhý experiment, který má redukovaný LM. Obsahuje pouze slova, která se opravdu vyskytla v rámci poslechového testu. Tento experiment je nazván jako \uv{reduced}. Výsledky obou těchto experimentů jsou porovnány s poslechovým testem na izolovaných slovech. Další možností by mohlo být vytvoření speciálních LM pro každou promluvu. Ten by obsahoval pouze komplementární slova. Problémem tohoto postupu je 3 možnost, kterou obsahuje poslechový test (tzv. \uv{nemohu rozhodnout}). V případě, že by LM obsahoval pouze dvě slova, tak by ASR experiment, svými parametry, neodpovídal poslechovému testu.

Poslední experiment odpovídá druhému poslechovému testu. K získání srovnatelných výsledků, je ke každé nahrávce se slovním bigramem vygenerován speciální zerogramový LM. Ten obsahuje vždy pouze všechny 4 kombinace slov. Tím pádem odpovídá dostupným možnostem v rámci poslechového testu. Tento experiment je nazván jako \uv{bigrams}. Jeho výsledky jsou porovnány s druhým poslechovým testem.

Výsledkem rozpoznávače je nejlepší hypotéza (případně \textit{N} nejlepších hypotéz), tudíž slovo. To však není porovnatelné s výsledkem poslechového testu. Z tohoto důvodu jsou všechny výsledky ohodnoceny $1$, pokud bylo výstupem správné slova, a v opačném případě $0$. Násladně byl z tohoto ohodnocení vypočten průměr. Pro upřesnějí je nutné zmínit, že \textit{i/y} na výsledném ohodnocení nehraje roli. Dosažené výsledky jsou pak v tab. \ref{tab:realisation:comparison}. Ty ukazují, že požedovaný úkol je výzvou jak pro člověka, natož pro stroj. V případě experimentu \uv{onemil} je výkon ASR systému významně horší než výkon člověka. To je zejména způsobeno enormní perplexitou jazykového modelu. Ta je přímo rovna velikosti slovníku. Zmenšením slovníku se podařilo získat výsledky srovnatelné s člověkem. Je dobré zdůraznit, že i v případě \uv{reduced} experimentu hrají karty ve prospěch člověka, protože čelí pouze perplexitě $3$, protože se kdykoli může podívat na nabízené možnosti. Řešením by bylo nechat účastníky přepsat obsah promluvy a porovnat ho se skutečným obsahem. Nicméně toto by významně zvýšílo náročnost (zejména časovou) poslechového testu a bylo by velmi komplivané získat kompletní výsledky od relevantního množství účastníků. Už jen podíl odpadlíků mezi prvním a druhým poslechovým testem činí závratných $30\ \%$.

Velmi zajímavé jsou výsledky u \uv{bigrams} experimentu. Na první pohled se může jevit jako snazší, protože úkolem je vybrat z jasně definovaných kombinací slov. Avšak slova jsou si akusticky velmi podobná a v mnoha případech je velmi náročné je od sebe rozeznat. Jak člověk, tak stroj, dosáhli v tomto testu nejhorších výsledků. Při analýze se ukázalo, že rozdíly mezi hypotézami ASR systému jsou velmi malé, což naznačuje velkou podobnost mezi inkriminovanými modely fonémů. Zároveň tyto výsledky s velmi podobnými hypotézami korelují s výsledky poslechového testu, kde posluchači nebyli schopni jednoznačně rozhodnout o významu jednotlivých slov.

\begin{table}[htpb]
  \centering
  \def\arraystretch{1.5}
  \pgfplotstabletypeset[
    col sep=semicolon,
    string type,
    columns/type/.style={column name= , column type={|l}},
    columns/onemil/.style={column name=onemil \textit{[\%]}, column type={|r}},
    columns/reduced/.style={column name=reduced \textit{[\%]}, column type={|r}},
    columns/bigrams/.style={column name=bigrams \textit{[\%]}, column type={|r|}},
    every head row/.style={after row=\hline, before row=\hline},
    every last row/.style={after row=\hline},
  ]{./ch5-construction/tabs/0208-comparison.csv}
  \caption{Porovnání dosažených výsledků člověka a stroje.}
  \label{tab:realisation:comparison}
\end{table}

Z dosažených výsledků je zřejmé, že stroj nedosahuje schopností člověka. Pokud se vezme v úvahu, že byl člověk oproti stroji vždy v malé výhodě, tak dosažené výsledky jsou relativně optimistické. Minimálně v jednom případě se stroj téměř vyrovnal člověku. Samotnou kapitolou je vliv kontextu. Ještě před samotnými ASR experimenty byl ověřen výkon ASR systému na \uv{kontinuální} řeči, zde reprezentované větami z testovací sady. Jazykovým modelem je v tomto případě trigramový slovní model obsahující $1,2$ milionu unikátních slov. Přesnost na slovech (počítaná pomocí rovnice (\ref{eq:asr:decoding:acc})) dosáhla hodnoty $86,10\ \%$. Při porovnání s výsledky z tab. \ref{tab:realisation:comparison} jasně plyne, že pokud je k dispozici dostatečný kontext, tak je ASR schopen správně určit význam slova. Přeci jen slovo \uv{kosa} se většinou vyskytuje v trochu jiném slovním kontextu, než slovo \uv{koza} a toto platí u většiny dvojic.
