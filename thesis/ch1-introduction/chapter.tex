% !TEX root = ../thesis.tex
\chapter{Úvod}
\label{chap:introduction}

Lidská řeč je jedním z hlavních dorozumívacích prostředků užívaných člověkem, proto
ztráta hlasu způsobuje řadu komplikací. Její příčinou může být chirurgický zákrok, který byl proveden za účelem odstranění nádorového onemocnění v oblasti hrtanu, nebo
poškození hrtanu vlivem traumatické nehody. Proto se lékaři již od konce 19.~století snaží o rehabilitaci pacientova hlasu za účelem zvýšit kvalitu jeho života. 

První snahy o navrácení schopnosti mluvit nebyly příliš úspěšné a byly svým
způsobem i životu nebezpečné. Přesto neutuchající snaha lékařů postupně
vyústila nejen ve vyvinutí bezpečných operačních postupů, ale i metod následně využívaných pro rehabilitaci hlasu. Mezi nejpoužívanější přístupy patří chirurgicko-protetické a foniatrické metody.
Nejčastěji postižení pacienti využivají pro rehabilitaci hlasu elektrolarynx, jícnový hlas a tracheoezofageální píštěl. Bohužel žádná z používaných metod není univerzálním řešením pro každého pacienta.
U~nemalého počtu pacientů je navíc snaha začít opětovně komunikovat s okolím pomocí mluvené řeči doprovázena významnou psychickou zátěží mluvčího, který se například může ostýchat mluvit na veřejnosti. 
Z~toho důvodu je problematice rehabilitace hlasu v současnosti věnována nemalá pozornost. Významnou pomoc mohou v tomto ohledu přinést řečové technologie.

V~polovině 20.~století se s rozvojem číslicových počítačů začaly objevovat první
snahy o zpracování přirozené řeči počítačem. Toto úsilí vyústilo ve vyvinutí v dnešní
době hojně užívaných systémů automatického rozpoznávání řeči (zkr. ASR) a systémů pro syntézu
řeči (zkr. TTS). Nejmodernější ASR systémy jsou schopné pracovat s~obrovskými
slovníky v~mnoha rozličných situacích. Největší problémy však stále způsobuje
okolní hluk ovlivňující výkon těchto systémů. O eliminaci jeho vlivu se
výzkumníci snaží už od samých počátků jejich vývoje. V~mnoha případech se
inspirují schopnostmi člověka, protože ten je schopen relativně úspěšně
porozumět promluvě i za velmi ztížených podmínek. 

Tyto snahy velmi často vedou
k vytvoření multimodálních systémů zpracovávajících nejen akustická data, ale
například i data obrazová. %videozáznam artikulace rtů. 
Bohužel multimodální systémy zatím nedosahují požadovaných kvalit, proto se vývoj ASR systémů v současnosti ubírá zejména směrem vývoje komplexnějších modelů.
Běžně využívané systémy rozpoznávání řeči jsou však trénovány na obecných datech a pro uživatele postižené trvalou ztrátou hlasu jsou nepoužitelné. Jako jeden z hlavnich problémů se jeví jiné charakteristiky produkované řeči a ztráta určitého množství informace v ní obsažené. Ke ztrátě části informace dochází v důsledku chybějícího buzení  proudu vzduchu hlasivkami. Nejčastěji využívané metody rehabilitace hlasu se totiž snaží nahradit chybějící buzení  jiným zdrojem buzení, které má ale v podstatě konstantní charakter. Obecné ASR systémy pak nejsou bez adaptace schopné obstojně tuto řeč zpracovávat, proto se většina doposud vyvíjených metod snaží získat tuto ztracenou informaci z~dalšího doprovodného zdroje dat (např. kamerového záznamu artikulace). Výsledné multimodální systémy však zatím nedosahují konkurence schopných výsledků a ve většině případů předpokládají využití dalšího (prozatím) neergonomického zařízení.

Tato práce si klade za cíl prozkoumání možností rozšíření schopností ASR systému tak, aby se výkon vytvořeného systému co možná nejvíce blížil obecnému na řečníkovi nezávislému ASR systému. Velký důraz je kladen na co možná nejmenší požadavky na samotného řečníka, aby bylo možné navržený systém převést do praxe, a tím tak zlepšit v určitých aspektech život lidí postižených trvalou ztrátou hlasivek.

% Překotný vývoj technologií, který je vidět v poslední době, zažehl snahu o
% vytvoření systémů pracujících (výhradně) s jinými než akustickými daty. To by
% umožňovalo použití i v situacích, kdy klasické systémy nedokáží optimálně
% plnit svou funkci. K~tomuto účelu se vyvíjejí technologie zpracovávající tzv.
% \uv{tichou} řeč postavené na záznamu artikulace mluvčího, mozkové aktivitě,
% případně tělem šířené řeči. Schopnost pracovat i v případě absence akustických
% dat předurčuje tyto technologie k využití při rehabilitaci hlasu. Při správné
% funkci mohou poskytnout přirozenější a kvalitnější hlas než klasické metody
% rehabilitace a výrazně tak zlepšit kvalitu života pacientů postižených ztrátou
% hlasu.

% Cílem této práce je přinést kompletní přehled používaných metod rehabilitace
% hlasu. Primárně se zaměřuje na možnost využití řečových technologií v této
% oblasti. Je samozřejmě jasné, že vytvoření funkčních, masově používaných a
% uživatelsky přívětivých systémů, je velmi náročný úkol, který zatím není
% uspokojivě vyřešen. Přesto principiálně nic nebrání, aby mohly být tyto
% systémy použity při rehabilitaci hlasu.
