% !TEX root = ../thesis.tex
\chapter{Úvod}
\label{chap:introduction}

Lidská řeč je jedním z hlavních dorozumívacích prostředků užívaných člověkem.
Její ztráta způsobuje řadu komplikací. Příčinou ztráty může být chirurgický zákrok v důsledku nádorovitého onemocnění nebo
poškození hrtanu vlivem traumatické nehody. S ohledem na zvýšení kvality
života se už od konce 19. století lékaři snaží o rehabilitaci pacientova hlasu.
Mezi nejpoužívanější přístupy patří chirurgicko-protetické a foniatrické metody.

První snahy o navrácení schopnosti mluvit nebyly příliš úspěšné a svým
způsobem i životu nebezpečné. Přesto neutuchající snaha lékařů postupně
vyústila v bezpečné a běžně používané metody. Mezi nejpoužívanější se řadí
využití elektrolarynxu, jícnového hlasu a tracheoezofageální píštěle. Bohužel
žádná z používaných metod není univerzálním řešením pro každého pacienta.
V~mnoha případech je navíc případné používaní spojené s nemalou psychickou
zátěží mluvčího, který se může například ostýchat mluvit na veřejnosti.
Z~tohoto důvodu je této problematice věnována nemalá pozornost a další pomoc
mohou přinést řečové technologie.

V polovině 20. století se s rozvojem číslicových počítačů začaly objevovat
snahy o zpracování přirozené řeči počítačem. Toto úsilí vyústilo v (v dnešní
době) hojně užívané systémy automatického rozpoznávání (zkr. ASR) a syntézy
řeči (zkr. TTS). Nejmodernější ASR systémy jsou schopné pracovat s obrovskými
slovníky v mnoha rozličných situacích. Největší problémy však stále způsobuje
okolní hluk ovlivňující výkon těchto systémů. O eliminaci jeho vlivu se
výzkumníci snaží už od samých počátků jejich vývoje. V mnoha případech se
inspirují schopnostmi člověka, protože ten je schopen relativně úspěšně
porozumět promluvě i za velmi ztížených podmínek. Tyto snahy velmi často vedou
k vytvoření multimodálních systémů zpracovávajících nejen akustická data, ale
například i videozáznam artikulace rtů. Bohužel multimodální systémy zatím nedosahují požadovaných kvalit a tak se vývoj ASR systému ubírá zejména směrem vývoje komplexnějších modelů.

Běžné systémy rozpoznávání řeči jsou však trénovány na obecných datech a pro uživatele postižené trvalou ztrátou hlasu jsou nepoužitelné. Hlavním problémem se jeví jiné charakteristiky produkované řeči a ztráta určitého množství informace v ní obsažené. Tato ztráta je důsledkem aktuálně používaných metod rehabilitace hlasu, které se snaží nahradit chybějící buzení hlasivek jiným, ale ve své podstatě konstantním, zdrojem buzení. Obecné ASR systémy pak nejsou bez adaptace schopné obstojně tuto řeč zpracovávat.

Většina doposud vyvíjených metod se tuto ztracenou informaci snaží získat pomocí zapojení dalšího zdroje dat (např. kamerového záznamu artikulace). Výsledné multimodální systémy však zatím nedosahují konkurence schopných výsledků a ve většině případů předpokládají využití dalšího (prozatím) neergonomického zařízení.

Tato práce si klade za cíl prozkoumání možností rozšíření schopností ASR systému tak, aby se výkon vytvořeného systému co možná nejvíce blížil obecnému na řečníkovi nezávislému ASR systému. Velký důraz je kladen na co možná nejmenší požadavky na samotného řečníka, aby bylo možné navržený systém převést do praxe a tím tak zlepšit v určitých aspektech život lidí postižených trvalou ztrátou hlasivek.

% Překotný vývoj technologií, který je vidět v poslední době, zažehl snahu o
% vytvoření systémů pracujících (výhradně) s jinými než akustickými daty. To by
% umožňovalo použití i v situacích, kdy klasické systémy nedokáží optimálně
% plnit svou funkci. K~tomuto účelu se vyvíjejí technologie zpracovávající tzv.
% \uv{tichou} řeč postavené na záznamu artikulace mluvčího, mozkové aktivitě,
% případně tělem šířené řeči. Schopnost pracovat i v případě absence akustických
% dat předurčuje tyto technologie k využití při rehabilitaci hlasu. Při správné
% funkci mohou poskytnout přirozenější a kvalitnější hlas než klasické metody
% rehabilitace a výrazně tak zlepšit kvalitu života pacientů postižených ztrátou
% hlasu.

% Cílem této práce je přinést kompletní přehled používaných metod rehabilitace
% hlasu. Primárně se zaměřuje na možnost využití řečových technologií v této
% oblasti. Je samozřejmě jasné, že vytvoření funkčních, masově používaných a
% uživatelsky přívětivých systémů, je velmi náročný úkol, který zatím není
% uspokojivě vyřešen. Přesto principiálně nic nebrání, aby mohly být tyto
% systémy použity při rehabilitaci hlasu.
